% !TeX root = ../modelizacion.tex

\chapter{Modelos matriciales discretos}

\section{Modelo discreto unidimensional}

En esta sección trataremos problemas con ecuaciones en diferencias finitas. Podemos dar un ejemplo con el interés de una tarjeta de crédito.

\begin{eg}[Interés en una tarjeta de crédito]
    Tomamos $x_0$ la cantidad que debemos, e $i = 2\%$ el tipo de interés.\\
    $x(t)$ es la cantidad de dinero que debo cuando han pasado $t$ días. Vamos a intentar dar una expresión para $x$.
    \begin{align*}
        x(0) &= x_0\\
        x(1) &= (1 + i)\cdot x(0) = x(0) + i \cdot x(0)\\
        x(2) &= (1+i) \cdot x(1) = (1+i)^2\cdot x(0)
    \end{align*}
    y por inducción hallamos:
    $$
        x(k) = (1+i)^{k} \cdot x(0)
    $$
    Por tanto obtenemos el sistema:
    \begin{align*}
        x(k+1) &= ax(k)\\
        x(0) &= x_0
    \end{align*}
    donde $a = (1+i)$, y estamos ante un sistema de ecuaciones en diferenciadas finitas homogéneas y autónomas.
\end{eg}

%%%%%%%%%%%%%%%%%%%%%%%%%%%%%%%%%%%%%%%%%%%%%%%%%%%%%%%%%%%%%%%%%%%%%%%%%% 05/02
\section{Dinámica discreta de poblaciones}

Vamos a ver como podemos dar distintos modelos de poblaciones.

\begin{eg}[Modelo de población de elefantes hembras de Botswana]
    Consideramos:
    \begin{align*}
        y(k) &= \sdf{\text{Población de elefantes hembras de Botswana}}\\
        y(k+1) &= \sdf{\text{Sobreviven}} + \sdf{\text{Nacen}} - \sdf{\text{Emigrantes}} + \sdf{\text{Inmigrantes}}
    \end{align*}
    Y llamamos $k$ al número de años, $s$ a la tasa de supervivencia, $n$ a la tasa de natalidad y suponiendo que Botswana no deja salir ni entrar a ningún elefante, tenemos que la tasa de inmigración y los puntos de emigración son nulas.\\
    Si hallamos que $s = 0.8$ y $n = 0.3$ tenemos la ecuación:
    $$
        y(k+1) = 0.8\cdot y(k) + 0.3\cdot y(k) = (1.1)\cdot y(k) = ay(k)
    $$
    Entonces, por inducción $y(k) = a^k y(0)$.\\
    En este caso, como $a > 1$, $1.1 = a = 1 + \alpha$ y $\alpha = 0.1$ diremos que la tasa de crecimiento de la población es del $10\%$. Además, el límite cuando $k \to \infty$ es $\infty$.
\end{eg}

Sin embargo, este modelo se puede sofisticar, veamos otro ejemplo.
\begin{eg}[Modelo de población de la mariposa monarca]
    Consideraremos $y(k) = \sdf{\text{Población de mariposas monarca}}$. Vamos a expresar este valor de forma matricial:
    $$
        y(k) = \px{y_1(k)\\y_2(k)} \implies
        \Mx{
            y_1(k)=\sdf{\text{número de crisálidas}} \\
            y_2(k)=\sdf{\text{número de mariposas adultas}}
        }
    $$
    En este sistema, $k$ nos indica el período de semana en que estamos y sabemos que:
    \begin{itemize}
        \item Cada semana maduran el $30\%$ de las crisálidas.
        \item Cada semana sobreviven el $10\%$ de las adultas.
        \item Las crisálidas o se transforman en adultas o se mantienen como crisálidas.
        \item Por cada mariposa adulta, una oruga se transforma en crisálida.
    \end{itemize}
    Entonces tenemos las relaciones:
    \begin{align*}
        y_1(k+1) &= 0.7 \cdot y_1(k) + y_2(k)\\
        y_2(k+1) &= 0.3 \cdot y_1(k) + 0.6 \cdot y_2(k)
    \end{align*}
    Y lo podemos expresar como $y(k+1) = A y(k)$ con la matriz:
    $$
        A = \px{0.7 & 1 \\ 0.3 & 0.6}
    $$
    Con esto, podríamos hacer el análisis de la población. Supongamos que tenemos una población inicial de:
    $$
        y(0) = \px{1000\\ 1000}
    $$
    Para hallar la población en el período $k$ basta usar la expresión:
    $$
        y(k+1) = A y(k) \implies y(k) = A^k \cdot y(0)
    $$
\end{eg}

Nos preguntamos ahora qué pasa con estas poblaciones. Recordemos que una matriz $A$ la podemos descomponer como $A = P J P^{-1}$  (por la descomposición de Jordan). $J$ es algo más fácil de manejar y en algunos estupendos casos es diagonal.
Además, recordemos que una condición suficiente de ser matriz diagonalizable es ser simétrica.\\

De esta forma, podemos expresar:
\begin{align*}
    A &= P J P^{-1}\\
    A^2 &= P J^2 P^{-1}\\
    A^n &= P J^{n} P^{-1}
\end{align*}
y nuestra pregunta se resuelve estudiando el \textbf{espectro} de $A$ (sus autovalores).\\

\begin{eg}[Modelo de población de la mariposa monarca - Continuación]

    Retomando el ejemplo de la población de mariposas, los autovalores son los ceros de:
    $$
        0 = \rho(\lambda) = \det(A - \lambda I) = \lambda^2 - 1.3\cdot \lambda + 0.42\cdot 0.3
    $$
    De donde obtenemos $\lambda_1 \simeq 1.2$ y $\lambda_2 \simeq 0.1$ y por tanto:
    $$
        J = \px{1.2 & 0 \\ 0 & 0.1} \implies J^n = \px{\infty & 0 \\ 0 & 0}
    $$
    También vamos a hallar los autovectores, (para la condición del primer autovector hemos pedido que tenga $\vabs{\vabs\cdot}_1 = 1$).
    \begin{align*}
        \px{-0.5 & 1\\ 0.3 & -0.6}\px{u_1 \\ u_2} = 0 &\implies u_1 = 2u_2,\ u_1+u_2=1\\
        \px{6 & 1\\ 0.3 & 0.5}\px{v_1 \\ v_2} = 0 &\implies v_2 = -0.6 \cdot v_1,\ v_1 = -\frac{5}{3}v_2\\
        &\implies u_1 = \px{\frac{2}{3} \\ \frac{1}{3}},\ u_2 = \px{\frac{5}{3}\\ -1}
    \end{align*}

    Donde tenemos que $\lambda_1$ un autovalor dominante $-0.1 = \lambda_2 < \lambda_1 = 1.2$ ya que
    $$
        \exists \lambda_1 \text{ tal que } \lambda_1 > \vabs{\lambda_i} i\neq 1
    $$
    y también $\lambda_1$ es positivo y por tanto, es el autovector de componentes positivas.\\
    Vamos a hallar una expresión de $y(k)$:
    \begin{align*}
        y(0) &= c_1 u_1 + c_2 u_2,\ y(0) \leq 0\\
        y(k) &= A^k \cdot y(0)\\
             &= A^{k-1} \cdot A \cdot y(0) = A^{k-1} \cdot(c_1 \lambda u_1 + c_2 \lambda_2 u_2)\\
             &= c_1 \lambda_1^k u_1 + c_2 \lambda_2^k u_2\\
             &= \lambda_1^k \px{c_1 u_1 + c_2 \pfrac{\lambda_2}{\lambda_1}^ku_2} \to_{k>>1} \lambda_1^k c_1 u_1 \to \infty
    \end{align*}
    De donde podemos inferir que $\lambda_1 > 1$ hace que la población sea enorme en el infinito y $\lambda_1 < 0$ hace que se extinga la población.
\end{eg}
%%%%%%%%%%%%%%%%%%%%%%%%%%%%%%%%%%%%%%%%%%%%%%%%%%%%%%%%%%%%%%%%%%%%%%%%%% 06/02

\begin{obs}
    La \textbf{tasa de crecimiento} viene dada por el \textbf{autovalor dominante}, y el \textbf{estado estacionario} (el equilibrio de las distintas clases de la población) viene dado por el \textbf{autovector} del \textit{autovalor dominante}.
\end{obs}

Volvamos al ejemplo anterior. Buscamos una tasa de supervivencia para las mariposas adultas de tal manera que la población se mantenga estable. Recordamos que teníamos la matriz del sistema:
$$
\px{0.7 & 1\\ 0.3 & \alpha}
$$
Ahora buscamos $\alpha$ tal que $\lambda_1 = 1$, entonces:
\begin{align*}
    \rho(\lambda) &= (0.7 - \lambda)(\alpha - \lambda) - 0.3\\
     &= \lambda^2 - \alpha \lambda - 0.7\lambda + 0.7\lambda - 0.3\\
     &= \lambda^2 - (0.7 + \alpha)\lambda + 0.7\lambda -0.3
\end{align*}
Entonces, despejando $\alpha$ de $\lambda(\alpha)=1$ obtenemos que $\alpha=0$

\subsection{Modelos de Leslie}
Dividimos la población por el rango de edad en $n$ períodos.

\begin{tikzpicture}[node distance=5cm and 2cm, auto]
    \node[rectangle, draw] (p1) {Período $1 \equiv$ población $1$ };
    \node[rectangle, draw, right of= p1] (p2) {Período $2 \equiv$ población $2$ };
    \node[circle, draw, right = 1cm of p2] (ind) {$\cdots$};
    \node[rectangle, draw, right = 1cm of ind] (p3) {Período $n \equiv$ población $n$ };
    \draw[->] (p1) edge [loop left] node {$a_1$} (p1);
    \draw[->] (p1) to node {$b_1$} (p2);
    \draw[->] (p2) to node {$b_2$} (ind);
    \draw[->] (ind) to node {$b_{n-1}$} (p3);
    \draw[->, out=90, in=90, distance=1cm] (p2) to node {$a_2$} (p1);
    \draw[->, out=90, in=90, distance=1.5cm] (p3) to node {$a_n$} (p1);
\end{tikzpicture}

Donde $b_i$ va a ser la \textit{tasa de supervivencia} de la población $i$ y $a_i$ a a ser la \textit{tasa de descendientes/reproducción} de la población $i$.\\
Sea
\begin{align*}
    y_i(k) &= \sdf{\text{número de individuos $i$ en la etapa $k$}}\\
    y_1(k+1) &= a_1y_1(k) + \cdots + a_ny_n(k)\\
    y_2(k+1) &= b_1y_1(k)\\
    y_n(k+1) &= b_ny_{n-1}(k)\\
\end{align*}
Entonces construimos la matriz
$$
    A = \px{
    a_1 & a_2 & \cdots & a_{n-1} & a_n\\
    b_1 & 0 & \cdots & 0 & 0\\
    0 & b_2 & \ddots & 0 & 0\\
    0 & 0 & \ddots & 0 & 0\\
    0 & 0 & \cdots & b_{n-1} & 0
    }
$$
De la que podemos hallar su polinomio característico.
$$
    \rho(\lambda) = \mathrm{det}(A - \lambda I) = (\cdots) = \lambda^n - a_1\lambda^{n-1} - b_1a_2\lambda^{n-2} - b_1b_2a_3\lambda^{n-3} - \cdots - (b_1b_2\ldots b_{n-1}a_n)
$$
%%%%%%%%%%%%%%%%%%%%%%%%%%%%%%%%%%%%%%%%%%%%%%%%%%%%%%%%%%%%%%%%%%%%%%%%%% 10/02
%TODO: Revisar todo este día
Podemos reescribirlo como
$$
    \rho(\lambda) = \lambda^n - (A_1\lambda^{n-1} + \cdots + A_n)
$$
donde
\begin{align*}
    A_1 &= a_1\\
    A_i &= b_1\cdots b_{i-1}a_i,\ i=2,\ldots, n
\end{align*}
De esta forma, vamos a ver ciertos aspectos:
\begin{enumerate}[1)]
    \item Existe un autovalor positivo $\lambda_1$.\\
    Comenzamos con $\rho(\lambda)=0$. Queremos ver que existe un $\lambda > 0$, entonces:
    \begin{align*}
        \lambda^n &= A_1\lambda^{n-1} + \cdots + A_n\\
        1 &= \underbracket{\frac{A_1}{\lambda} + \cdots + \frac{A_n}{\lambda^n}}_{q(\lambda)}
    \end{align*}
    Además, vemos que:
    \begin{enumerate}[i]
        \item $\lim_{\lambda\to\infty} q(\lambda) = 0$
        \item $\lim_{\lambda\to 0} q(\lambda) = \infty$
        \item $q'(\lambda) < 0,\ \lambda \in (0, \infty)$
    \end{enumerate}
    Notamos que $A_i \geq 0$ y además todos los $A_i = 0 \iff a_i = 0 \forall i$.
    %TODO: preguntar a Santorum por la línea de x(k)
    El caso en que $a_i = 0 \forall i$ es trivial ya que estamos ante una extinción.
    Además la hipótesis de que exista algun $a_j \neq 0$ es \textit{natural}. Por tanto:
    \begin{align*}
        \exists \lambda > 0 \mid \rho(\lambda) = 0 \iff \exists \lambda \mid q(\lambda) = 1
    \end{align*}
    Y por continuidad de $q$ esto ocurre.
    \item Existe un único autovalor positivo.\\
    $$q'(\lambda) = - \frac{A_1}{\lambda^2} - \frac{2A_2}{\lambda^3} - \cdots - \frac{nA_n}{\lambda^{n+1}}$$
    y por tanto $q'(\lambda) < 0 \forall \lambda \in (0, \infty)$, es decir, $q(\lambda)$ es una función decreciente. Como $\lambda_1$ es tal que $q(\lambda_1) = 1$ y entonces $\forall \lambda > \lambda_1\; q(\lambda) < 1$.
    \item $\lambda_1$ es simple.\\
    $\lambda_1$ no es simple $\iff \rho'(\lambda_1) = 0$. Recordemos que $\rho(\lambda) = \lambda^n(1-q(\lambda)) \forall \lambda \neq 0$.\\
    Si $\lambda$ no es simple entonces $0 = n\lambda_1^{n-1}(1-q(\lambda_1)) - \lambda^nq'(\lambda_1)$ y sabemos que $q(\lambda_1) = 1$, entonces
    $$
        0 = -\lambda^nq'(\lambda_1) \iff q'(\lambda_1) = 0
    $$
    pero hemos visto que $q'(\lambda) < 0$, por lo que el supuesto es falso y $\lambda_1$ es simple.
    \item Sean $\lambda_j$ los autovalores de $L$ (la matriz de Leslie), tenemos que:
    $$
        \vabs{\lambda_j} \leq \lambda_1
    $$
    \begin{enumerate}[i]
        \item Por reducción al absurdo cuando $\lambda_j \in \R$. Supongamos que existe $\lambda_j$ tal que $-\lambda_j > \lambda_1$, $\lambda_j < 0$.\\
        \begin{align*}
            0 = \rho(\lambda_j) &\iff 1 = q(\lambda_j)\\
            0 = \rho(\lambda_1) &\iff 1 = q(\lambda_1)
        \end{align*}
        entonces
        $$
            0 = q(\lambda_j) - q(\lambda_1) = A_1\px{\frac{1}{\lambda_j} - \frac{1}{\lambda_1}} + \cdots + A_n\px{\frac{1}{\lambda_j^n - \frac{1}{\lambda_1^n}}}
        $$
        Tenemos que $\frac{1}{\lambda_j} - \frac{1}{\lambda_1} < 0$, $\frac{1}{\lambda_j^2} - \frac{1}{\lambda_1^2} < 0$ y también $q(\lambda_j) - q(\lambda_1) < 0$.
        Y esto contradice que $\lambda_j, \lambda_1$ son raíces de $\rho(\lambda) = 0$.
        \item Supongamos $\lambda_j \in \C$ tal que $\mu = \vabs{\lambda_j} > \lambda_1$, con $\lambda_j = \mu e^{i\theta}$.
        %% TODO: Pedir a Santorum
    \end{enumerate}
    \item Si existen dos coeficientes $a_i$ contiguos distintos de cero, entonces
    $$
        \vabs{\lambda_1} < \lambda_1 \forall j \neq 1
    $$
    es decir, $\lambda_1$ es dominante.
    \begin{enumerate}[i]
        \item Supongamos que $-\lambda_1$ es autovalor.
        $$
            0 = q(-\lambda_1) - q(\lambda_1)
        $$
        entonces
        $$
            0 = A_1\px{\frac{1}{\lambda_1} - \frac{1}{\lambda_1}} + 2A_3\pfrac{-1}{\lambda_1} + \cdots
        $$
        con todos los coeficientes impares, y como algún impar es distinto entonces llegamos a que $0 < 0$.
        %% TODO: Pedir a Santorum línea de q(x) - q(y)
        \item Supongamos que $\lambda_1e^{i\theta}, \lambda_1e^{-i\theta}$ son raíces del polinomio.\\
        \begin{align*}
            1 = q(\lambda_1) &= \frac{A_1}{\lambda_1} + \cdots + \frac{A_n}{\lambda_1^n}\\
            1 = q(e^{i\theta}\lambda_1) &= \frac{A_1}{\lambda_1}e^{-i\theta} + \cdots + \frac{A_n}{\lambda_1^n}e^{-ni\theta}\\
            1 = q(e^{-i\theta}\lambda_1) &= \frac{A_1}{\lambda_1}e^{i\theta} + \cdots + \frac{A_n}{\lambda_1^n}e^{ni\theta}\\
        \end{align*}
        y tenemos
        $$
            q(e^{i\theta}\lambda_1) + q(e^{-i\theta}\lambda_1) - 2q(\lambda_1) = 0
        $$
        aplicando identidades trigonométricas
        $$
            2 \frac{A_1}{\lambda_1}(\cos \theta - 1) + 2 \frac{A_2}{\lambda_1^2}(\cos(2\theta) - 1) + \cdots + 2 \frac{A_n}{\lambda_1^n}(\cos(n\theta) - 1) = 0
        $$
    \end{enumerate}
\end{enumerate}
