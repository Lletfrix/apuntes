% !TeX root = ../modelizacion.tex

\chapter{Modelos matriciales discretos}

\section{Modelo discreto unidimensional}

En esta sección trataremos problemas con ecuaciones en diferencias finitas. Podemos dar un ejemplo con el interés de una tarjeta de crédito.

\begin{eg}[Interés en una tarjeta de crédito]
    Tomamos $x_0$ la cantidad que debemos, e $i = 2\%$ el tipo de interés.\\
    $x(t)$ es la cantidad de dinero que debo cuando han pasado $t$ días. Vamos a intentar dar una expresión para $x$.
    \begin{align*}
        x(0) &= x_0\\
        x(1) &= (1 + i)\cdot x(0) = x(0) + i \cdot x(0)\\
        x(2) &= (1+i) \cdot x(1) = (1+i)^2\cdot x(0)
    \end{align*}
    y por inducción hallamos:
    $$
        x(k) = (1+i)^{k} \cdot x(0)
    $$
    Por tanto obtenemos el sistema:
    \begin{align*}
        x(k+1) &= ax(k)\\
        x(0) &= x_0
    \end{align*}
    donde $a = (1+i)$, y estamos ante un sistema de ecuaciones en diferenciadas finitas homogéneas y autónomas.
\end{eg}

%%%%%%%%%%%%%%%%%%%%%%%%%%%%%%%%%%%%%%%%%%%%%%%%%%%%%%%%%%%%%%%%%%%%%%%%%% 05/02
\section{Dinámica discreta de poblaciones}

Vamos a ver como podemos dar distintos modelos de poblaciones.

\begin{eg}[Modelo de población de elefantes hembras de Botswana]
    Consideramos:
    \begin{align*}
        y(k) &= \sdf{\text{Población de elefantes hembras de Botswana}}\\
        y(k+1) &= \sdf{\text{Sobreviven}} + \sdf{\text{Nacen}} - \sdf{\text{Emigrantes}} + \sdf{\text{Inmigrantes}}
    \end{align*}
    Y llamamos $k$ al número de años, $s$ a la tasa de supervivencia, $n$ a la tasa de natalidad y suponiendo que Botswana no deja salir ni entrar a ningún elefante, tenemos que la tasa de inmigración y los puntos de emigración son nulas.\\
    Si hallamos que $s = 0.8$ y $n = 0.3$ tenemos la ecuación:
    $$
        y(k+1) = 0.8\cdot y(k) + 0.3\cdot y(k) = (1.1)\cdot y(k) = ay(k)
    $$
    Entonces, por inducción $y(k) = a^k y(0)$.\\
    En este caso, como $a > 1$, $1.1 = a = 1 + \alpha$ y $\alpha = 0.1$ diremos que la tasa de crecimiento de la población es del $10\%$. Además, el límite cuando $k \to \infty$ es $\infty$.
\end{eg}

Sin embargo, este modelo se puede sofisticar, veamos otro ejemplo.
\begin{eg}[Modelo de población de la mariposa monarca]
    Consideraremos $y(k) = \sdf{\text{Población de mariposas monarca}}$. Vamos a expresar este valor de forma matricial:
    $$
        y(k) = \px{y_1(k)\\y_2(k)} \implies
        \Mx{
            y_1(k)=\sdf{\text{número de crisálidas}} \\
            y_2(k)=\sdf{\text{número de mariposas adultas}}
        }
    $$
    En este sistema, $k$ nos indica el período de semana en que estamos y sabemos que:
    \begin{itemize}
        \item Cada semana maduran el $30\%$ de las crisálidas.
        \item Cada semana sobreviven el $10\%$ de las adultas.
        \item Las crisálidas o se transforman en adultas o se mantienen como crisálidas.
        \item Por cada mariposa adulta, una oruga se transforma en crisálida.
    \end{itemize}
    Entonces tenemos las relaciones:
    \begin{align*}
        y_1(k+1) &= 0.7 \cdot y_1(k) + y_2(k)\\
        y_2(k+1) &= 0.3 \cdot y_1(k) + 0.6 \cdot y_2(k)
    \end{align*}
    Y lo podemos expresar como $y(k+1) = A y(k)$ con la matriz:
    $$
        A = \px{0.7 & 1 \\ 0.3 & 0.6}
    $$
    Con esto, podríamos hacer el análisis de la población. Supongamos que tenemos una población inicial de:
    $$
        y(0) = \px{1000\\ 1000}
    $$
    Para hallar la población en el período $k$ basta usar la expresión:
    $$
        y(k+1) = A y(k) \implies y(k) = A^k \cdot y(0)
    $$
\end{eg}

Nos preguntamos ahora qué pasa con estas poblaciones. Recordemos que una matriz $A$ la podemos descomponer como $A = P J P^{-1}$  (por la descomposición de Jordan). $J$ es algo más fácil de manejar y en algunos estupendos casos es diagonal.
Además, recordemos que una condición suficiente de ser matriz diagonalizable es ser simétrica.\\

De esta forma, podemos expresar:
\begin{align*}
    A &= P J P^{-1}\\
    A^2 &= P J^2 P^{-1}\\
    A^n &= P J^{n} P^{-1}
\end{align*}
y nuestra pregunta se resuelve estudiando el \textbf{espectro} de $A$ (sus autovalores).\\

\begin{eg}[Modelo de población de la mariposa monarca - Continuación]

    Retomando el ejemplo de la población de mariposas, los autovalores son los ceros de:
    $$
        0 = \rho(\lambda) = \det(A - \lambda I) = \lambda^2 - 1.3\cdot \lambda + 0.42\cdot 0.3
    $$
    De donde obtenemos $\lambda_1 \simeq 1.2$ y $\lambda_2 \simeq 0.1$ y por tanto:
    $$
        J = \px{1.2 & 0 \\ 0 & 0.1} \implies J^n = \px{\infty & 0 \\ 0 & 0}
    $$
    También vamos a hallar los autovectores, (para la condición del primer autovector hemos pedido que tenga $\vabs{\vabs\cdot}_1 = 1$).
    \begin{align*}
        \px{-0.5 & 1\\ 0.3 & -0.6}\px{u_1 \\ u_2} = 0 &\implies u_1 = 2u_2,\ u_1+u_2=1\\
        \px{6 & 1\\ 0.3 & 0.5}\px{v_1 \\ v_2} = 0 &\implies v_2 = -0.6 \cdot v_1,\ v_1 = -\frac{5}{3}v_2\\
        &\implies u_1 = \px{\frac{2}{3} \\ \frac{1}{3}},\ u_2 = \px{\frac{5}{3}\\ -1}
    \end{align*}

    Donde tenemos que $\lambda_1$ un autovalor dominante $-0.1 = \lambda_2 < \lambda_1 = 1.2$ ya que
    $$
        \exists \lambda_1 \text{ tal que } \lambda_1 > \vabs{\lambda_i} i\neq 1
    $$
    y también $\lambda_1$ es positivo y por tanto, es el autovector de componentes positivas.\\
    Vamos a hallar una expresión de $y(k)$:
    \begin{align*}
        y(0) &= c_1 u_1 + c_2 u_2,\ y(0) \leq 0\\
        y(k) &= A^k \cdot y(0)\\
             &= A^{k-1} \cdot A \cdot y(0) = A^{k-1} \cdot(c_1 \lambda u_1 + c_2 \lambda_2 u_2)\\
             &= c_1 \lambda_1^k u_1 + c_2 \lambda_2^k u_2\\
             &= \lambda_1^k \px{c_1 u_1 + c_2 \pfrac{\lambda_2}{\lambda_1}^ku_2} \to_{k>>1} \lambda_1^k c_1 u_1 \to \infty
    \end{align*}
    De donde podemos inferir que $\lambda_1 > 1$ hace que la población sea enorme en el infinito y $\lambda_1 < 0$ hace que se extinga la población.
\end{eg}
