% !TeX root = ../modelizacion.tex

\chapter{An\'alisis dimensional}

El análisis dimensional es una herramienta que nos permite simplificar el estudio de cualquier fenómeno que involucre varias magnitudes físicas para tratarlas como variables independientes. Esto nos ayudará a simplificar los modelos matemáticos de lo que queramos estudiar.\\

Vamos a comenzar con un ejemplo introductorio a la asignatura, con el que se busca de alguna forma introducir conceptos que si bien no son del todo matemáticos o formales serán de utilidad en el desarrollo del curso.

\begin{eg}[Segunda Ley de Newton - Ley física]\label{eg:1}
    La segunda ley de Newton se puede escribir como la ecuación diferencial:
    $$
        m \ddot x(t) = F(x, t),\ t\in[0, T]
    $$
    donde $m$ representa la masa de un objeto, $x(t)$ la posición del mismo respecto del tiempo, $F(x, t)$ la fuerza que se ejerce sobre él y $T$ es el tiempo final.\\
    Para completar el problema daremos un par de condiciones iniciales:
    \begin{align*}
        x(0) = x_0 \text{ la posición inicial }\\
        \dot x(0) = v_0 \text{ la velocidad inicial}
    \end{align*}

    En el análisis dimensional analizaremos que \textbf{magnitudes} entran en juego en la \textbf{ley}. En este caso tenemos:
    \begin{itemize}
        \item $m$ - masa.
        \item $x$ - posición.
        \item $F$ - fuerza.
        \item $T$ - tiempo final.
        \item $x_0$ - posición inicial.
        \item $v_0$ - velocidad inicial.
    \end{itemize}

    Por tanto nuestra función final será de la forma:
    $$
        f(m, x, F, T, x_0, v_0) = 0
    $$
    que es otra forma de expresar la \textbf{ley}. Además, querremos ver de qué \textbf{magnitudes} dependen estos $6$ parámetros. Esto lo expresaremos con la notación:
    $$
        \mx{p} = M
    $$
    donde $p$ representa un parámetro y $M$ una magnitud (también puede ser un producto de ellas). En nuestro caso tenemos:
    \begin{itemize}
        \item $\mx{m} = M$. Masa, una magnitud elemental.
        \item $\mx{x} = L$. Longitud, una magnitud elemental.
        \item $\mx{T} = \tau$. Tiempo, una magnitud elemental.
        \item $\mx{x_0} = L$. Longitud.
        \item $\mx{v_0} = L \cdot \tau^{-1}$. Velocidad, longitud $\times$ tiempo$^{-1}$.
        \item $\mx{F} = \mx{m \cdot \ddot x} = M \cdot L \cdot \tau^{-2}$. Fuerza, masa $\times$ longitud $\times$ tiempo$^{-2}$.
    \end{itemize}
\end{eg}

\section{Magnitudes. Teorema $\Pi$.}

Vamos a suponer la existencia de $\sucn{L}$ magnitudes elementales con $n < \infty \in \N$, es decir, cada $L_i$ es independiente de cada magnitud de $\mtc L \setminus L_i$. Diremos que una colección de magnitudes conforman un sistema.

\begin{dfn}[Dimensión de una magnitud]
    Sea $a \in \R$ una medida de una magnitud $A$ en un sistema $\sucn{L}$. Si cambiamos a un sistema $\sucn{L'}$ con $L'_i = \lambda_i L_i$ y sea $a'$ la medida de $A$ en el nuevo sistema, entonces si se cumple que:
    $$
        a' = a \cdot \lambda_1^{a_1} \cdot \cdots \cdot \lambda_n^{a_n}
    $$
    para una serie de escalares $\sucn{a}$, entonces diremos que la magnitud $A$ tiene \textbf{dimensión} $L_{1}^{a_1} \cdots L_{n}^{a_n}$ y lo expresamos por:
    $$
        \mx{A} = L_{1}^{a_1} \cdots L_{n}^{a_n}
    $$

    Cuando el sistema $\sucn{L}$ esté fijado podremos identificar la dimensión de la magnitud $A$ con el vector de escalares $(\sucn{a})$.
\end{dfn}

Recordando el \hyperref[eg:1]{ejemplo de la segunda ley de Newton}, donde teníamos tres magnitudes elementales $(L, \tau, M)$, si consideramos que nuestro sistema $\sucn{L}$ es dicha $3$-tupla, entonces podemos expresar las magnitudes no elementales como $3$-tuplas (o vectores de $\R^3$):
\begin{itemize}
    \item $\mx{v_0} = L \cdot \tau^{-1} = (1, -1, 0)$
    \item $\mx{F} = \mx{m \ddot x} = M \cdot L \cdot \tau^{-1} =  (1, -2, 1)$
\end{itemize}

\begin{eg}[Dimensión de una magnitud]
    Sea $L_1 = \sdf{m} $ y $L_2=\sdf{s}$ un sistema de magnitudes (longitud en metros y tiempo en segundos  respectivamente), consideramos la magnitud de la velocidad $V$ que tiene dimensión:
    $$
        \mx{V} = L_1 \cdot L_2^{-1} = (1, -1)
    $$
    entonces, si tenemos una medida $v = 30 \sfrac{m}{s}$ y queremos ver su medida $L'$ en el sistema:
    \begin{align*}
        L_1' &= 10^{-3} L_1 \text{ con $L_1'$ longitud en km }\\
        L_2' &= \frac{1}{3600} L_2 \text{ con $L_2'$ tiempo en h }
    \end{align*}
    entonces:
    $$
        v' = v \cdot \lambda_1 \cdot \lambda_2^{-1} = \frac{30 \cdot 3.6 \cdot \cancel{10^3}}{\cancel{10^3}} = 108 \sfrac{km}{h}
    $$
\end{eg}

\begin{pro}[Expresión de una magnitud dependiente]
    Sean $A, B$ dos magnitudes tales que:
    \begin{align*}
        \mx{A} &= \Multn{L}{a}\\
        \mx{B} &= \Multn{L}{b}
    \end{align*}
    Sea $C$ otra magnitud dependiente de $A$ y $B$, tal que si $a, b$ son medidas de $A, B$ y $C$ y $\exists p, q, d$ tales que $c = d\cdot a^p + b^q$ con $p, q, d$ independientes de las unidades $\sucn{L}$, entonces:
    $$
        \mx{C} = L_1^{a_1p + b_1q} \cdots L_n^{a_n p + b_n q}
    $$
\end{pro}

\begin{proof}
    Sean $L'_i = \lambda_i L_i$ un nuevo sistema, entonces:
    $$
        a' = a \cdot \Multn{\lambda}{a},\ \ b' = b \cdot \Multn{\lambda}{b}
    $$
    y por tanto $c'$ es:
    \begin{align*}
        c' &= da'^p + b'^q = d(a\Multn{\lambda}{a})^p + (b\Multn{\lambda}{b})\\
          &= (da^p + b^q) \cdot (\lambda_1^{a_1p+b_1q} \cdots \lambda_n^{a_np+b_nq})\\
          &=  c \cdot (\lambda_1^{a_1p+b_1q} \cdots \lambda_n^{a_np+b_nq}) \implies\\
          \mx{C} &= L_1^{a_1p + b_1q} \cdots L_n^{a_n p + b_n q}
    \end{align*}
\end{proof}

\begin{dfn}[Matriz de dimensiones]
    Dados $\sucm{q}$ magnitudes, tales que su dimensión es:
    $$
        \mx{q_i} = \Multn{L}{{a_i}}
    $$
    llamamos \textbf{matriz de dimensiones} a la matriz ($n \times m$):
    $$
        \dmx{
            a_{11} & a_{21} & \cdots & a_{m1}\\
            \vdots & \vdots & \ddots & \vdots\\
            a_{1n} & a_{2n} & \cdots & a_{mn}
            }
    $$
    que tiene $n$ filas (una por cada \textit{magnitud elemental} $L_i$) y $m$ columnas (una por cada magnitud del problema $q_i$)
\end{dfn}

\begin{obs}
    Los índices de los elementos de la matriz quedan permutados respecto de la notación habitual.
\end{obs}

Retomando el \hyperref[eg:1]{ejemplo de la segunda ley de Newton} tendríamos la \textit{matriz de dimensiones}:
$$
    \Dmx{~ & x_0 & v_0 & T & F & M & x \cr
        L & 1 & 1 & 0 & 1 & 0 & 1\cr
        \tau & 0 & -1 & 1 & -2 & 0 & 0\cr
        M & 0 & 0 & 0 & 1 & 1 & 0
    }
$$

\begin{dfn}[Magnitud adimensional]
    Una magnitud $\Pi$ se dice adimensional si $\mx{\Pi} = 1$.
\end{dfn}

Hallar magnitudes adimensionales en los distintos problemas nos ayuda a simplificar el estudio de los mismos. Retomemos el \hyperref[eg:1]{ejemplo de la segunda ley de Newton}, vamos a intentar reducir la dimensión del problema.

\begin{eg}[Reduciendo la dimensión del ejemplo \ref{eg:1}]
    Recordemos que teníamos $6$ parámetros ($x, x_0, v_0, T, F, m$). Una forma de reducir los parámetros es intentar \textit{enmascarar} los valores iniciales en nuevas variables.\\ Recordemos que tanto $x$ como $x_0$ tenían la misma dimensión. Gracias a ello podemos definir un nuevo parámetro $y$ sin dimensión:
    $$
        y = \frac{x}{x_0}
    $$
    Además, como $x(0) = x_0$ tendremos que $y(0) = 1$ y como $\dot x(0) = v_0$ entonces $\dot y(0) = \frac{v_0}{x_0}$.\\
    Podemos también hacer lo mismo con el tiempo, recordemos que en la fórmula original la variable $t$ pertencía a $[0, T]$. Podemos definir entonces:
    $$
        \frac{t}{T} = \tau
    $$
    y por tanto $\tau \in [0, 1]$, con lo que hemos \textit{eliminado} $T$. Sin embargo, cambiar $t$ tiene consecuencias debido que es la variable respecto de la que se diferencia $x$ (y por tanto $y$), tenemos que ver como afectan estos cambios a nuestras variables.\\
    Usaremos la notación $\dot x$ para referirnos a $\pd{x}{t}$ y $x'$ para referirnos a $\pd{x}{\tau}$.\\
    Entonces obtenemos:
    $$
        y' = \pd{y}{\tau} = \frac{1}{x_0} \pd{x}{\tau} = \frac{T}{x_0} \pd{x}{t} = \frac{T}{x_0} \dot x \implies y'(0) = \frac{T}{x_0} \cdot v_0 = \tilde q
    $$
    y de nuevo $\mx{\tilde q} = 1$.\\\\

    Recordemos que nuestro problema comenzaba con $m\ddot x = F$, vamos a usar esto para encontrar otro parámetro adimensional.
    \begin{align*}
        &\pd{x}{t} = \frac{x_0}{T} \pd{y}{\tau}\\
        &\pdn{x}{t}{2} = \frac{x_0}{T} \Pd{t} \left(\pd{y}{\tau}\right) = \frac{x_0}{T} \pd{\tau}{t} \pdn{y}{\tau}{2} = \frac{x_0}{T} \cdot y''\\
        & \frac{mx_0}{T^2} \cdot y'' = F \implies y'' = \frac{T^2}{mx_0}\cdot F = f
    \end{align*}
    y podemos comprobar que $\mx{f} = 1$. Recapitulado, hemos conseguido encontrar nuevos parámetros adimensionales $y$, $\tilde q = y'$, $f = y''$ haciendo algunos cambios en el problema. Con esto, podemos reescribir el problema de valores iniciales con los nuevos parámetros adimensionales:
    \begin{align*}
        &y'' = f = \frac{T^2}{mx_0}\cdot F\\
        &y(0) = 1\\
        &y'(0) = \tilde q = \frac{T v_0}{x_0}
    \end{align*}
\end{eg}

\begin{dfn}[Ley invariante]
    Sea una ley $f(\sucm{q}) = 0$, se dice que es \textbf{invariante} frente al cambio de unidades $L'_1 = \lambda_1 L_1, \ldots L'_n = \lambda_n L_n$ si verifica que $f(\sucm{q'}) = 0$ para $\sucm{q'}$ las medidas de $\sucm{q}$ en las nuevas unidades $\sucn{L'}$. Informalmente:\\
    \begin{center}
        Una ley es invariante cuando sigue siendo cierta tras el cambio de variables del problema
    \end{center}
\end{dfn}

\begin{thm}[Teorema $\Pi$]\label{thm:pi}
    Sea $f(\sucm{q}) = 0$ una ley invariante con $\sucm{q}$ magnitudes con matriz de dimensiones:
    $$
        D = \dmx{
            a_{11} & \cdots & a_{m1}\\
            \vdots & \ddots & \vdots\\
            a_{1n} & \cdots & a_{mn}
        }
    $$
    tal que $n < m$ y el rango de $D$ es $r \leq n$. Entonces existen $m - r$ cantidades $\suc{\Pi}{m-r}$ que van a ser magnitudes adimensionales tales que la ley invariante es equivalente a una relación $F(\suc{\Pi}{m-r}) = 0$.
\end{thm}

\begin{proof}$ $
    \begin{enumerate}[(1)]
        \item Vamos a demostrar que existen $\suc{\Pi}{m-r}$ magnitudes adimensionales independientes entre sí. Partimos de la matriz de dimensiones:
        $$
            \GDmx{a}
        $$
        Y entonces tenemos:
        $$\mx{\Pi} = 1 \text{ con } \Pi = \Multm{q}{\alpha}$$
        $$
            \mx{\Pi} = (\Multn{L}{{a_1}})^{\alpha_1} \cdot (\Multn{L}{{a_2}})^{\alpha_2} \cdots (\Multn{L}{{a_m}})^{\alpha_m} = L_1^{\alpha_1 a_{11} + \cdot + \alpha_m a_{m1}} \cdots L_n^{\alpha_1 a_{1n} + \cdots \alpha_m a_{m1}}
        $$
        De donde surge el sistema:
        \begin{align*}
            \alpha_1 a_{11} + \alpha_2 a_{21} + \cdots + \alpha_m a_{m1} &= 0\\
            \alpha_1 a_{12} + \alpha_2 a_{22} + \cdots + \alpha_m a_{m2} &= 0\\
            (\cdots)\\
            \alpha_1 a_{1n} + \alpha_2 a_{2n} + \cdots + \alpha_m a_{mn} &= 0\\
        \end{align*}
        Y la demostración se reduce a resolver un sistema homogéneo de $n$ ecuaciones con $m$ incógnitas. Como la matriz $D$ es de rango $r$, entonces existen $m-r$ soluciones linealmente independientes.

        \item La demostración de que la ley invariante es equivalente a otra que solo comprende a las magnitudes adimensionales para el caso general resulta muy difícil de escribir. Un caso particular para $m=4, n=2$ y $r=2$ se encuentra en el siguiente ejemplo.
    \end{enumerate}
\end{proof}

\begin{eg}[Cálculo de las magnitudes adimensionales y la ley invariante asociada]
    Vamos a hacer un caso particular del sistema de la demostración. Sea $m=4$, $n=2$ y $r=2$, es decir:
    \begin{align*}
        \alpha_1 a_{11} + \alpha_2 a_{21} + \alpha_3 a_{31} + \alpha_4 a_{41} &= 0\\
        \alpha_1 a_{12} + \alpha_2 a_{22} + \alpha_3 a_{32} + \alpha_4 a_{42} &= 0\\
    \end{align*}
    Como el rango es $2$, sin perdida de generalidad puedo suponer que:
    $$
        \det \mx{a_{11} & a_{21}\\
                 a_{12} & a_{22}} = 0
    $$
    Entonces por Rouche-Frobenius existen $C_{34}, C_{32}, C_{41}, C_{42}$ tales que:
    \begin{align*}
        \alpha_1 = - \alpha_3 C_{31} - \alpha_4 C_{41} &\text{ con } \alpha_3 = 1,\ \alpha_4=0\\
        \alpha_2 = - \alpha_3 C_{32} - \alpha_4 C_{42} &\text{ con } \alpha_3 = 0,\ \alpha_4=1
    \end{align*}

    Y entonces puedo encontrar las magnitudes adimensionales:
    \begin{align*}
        \Pi_1 = q_1^{-C_{31}} q_2^{-C_{32}} q_3^{1} q_4^{0} \implies q_3 = \Pi_1 q_1^{C_{31}} q_2^{C_{32}}\\
        \Pi_2 = q_1^{-C_{41}} q_2^{-C_{42}} q_3^{0} q_4^{1} \implies q_4 = \Pi_2 q_1^{C_{41}} q_2^{C_{42}}
    \end{align*}

    Recordemos que partíamos de una ley:
    $$
        f(q_1, q_2, q_3, q_4) = 0
    $$
    Y además, hemos encontrado que:
    $$
        f(q_1, q_2, q_3, q_4) = f(q_1, q_2, \Pi_1 q_1^{C_{31}} q_2^{C_{32}}, \Pi_2 q_1^{C_{41}} q_2^{C_{42}}) = G(q_1, q_2, \Pi_1, \Pi_2)
    $$
    Además, vamos a hacer un cambio del sistema de magnitudes de la forma:
    $$
        L_1' = \lambda_1 L_1,\ L_2' = \lambda_2 L_2
    $$
    para conseguir:
    \begin{align*}
        q_1' &= q_1 \lambda_1^{a_11} \lambda_2^{a_12}\\
        q_2' &= q_2 \lambda_1^{a_21} \lambda_2^{a_22}\\
        \Pi'_1 &= \Pi_1\\
        \Pi'_2 &= \Pi_2
    \end{align*}
    Entonces tenemos una nueva ley invariante al cambio de escala:
    $$
        0 = G(q_1, q_2, \Pi_1, \Pi_2) = G(q_1', q_2', \Pi_1, \Pi_2)
    $$
    Finalmente, querremos hacer un cambio de variables tal que $q_1 = q_2 = 1$ en nuestra nueva ley.
    \begin{align*}
        L_nq_1 + a_{11} L_n \lambda_1 + a_{12} L_n + \lambda_2 &= 0\\
        L_nq_2 + a_{21} L_n \lambda_1 + a_{22} L_n + \lambda_2 &= 0
    \end{align*}
    En el sistema anterior hacemos el cambio: $y_i = L_n \lambda_i$ y tomando logaritmos obtenemos:
    \begin{align*}
        a_{11} y_1 + a_{12} y_2 = - \log(q_1)\\
        a_{21} y_1 + a_{22} y_2 = - \log(q_2)
    \end{align*}
    Y sabemos que dichos $q_1, q_2$ existen porque la matriz es invertible, y por tanto:
    $$
        0 = f(q_1, q_2, q_3, q_4) = G(1, 1, \Pi_1, \Pi_2) = F(\Pi_1, \Pi_2)
    $$
\end{eg}

\begin{eg}[H1.4]$ $
    \begin{enumerate}[a)]
        \item Magnitudes. Magnitudes elementales. Magnitudes adimensionales.\\\\
        Nuestras magnitudes $q_i$ serán ($h, t, g, R, v$). Podemos analizarlas en función de las magnitudes elementales:
        \begin{itemize}
            \item $\mx{h} = L = $ \{longitud\}.
            \item $\mx{t} = T = $ \{tiempo\}.
            \item $\mx{g} = L T^{-2}$.
            \item $\mx{v} = L T^{-1}$.
            \item $\mx{R} = L$.
        \end{itemize}
        Que tiene como matriz de dimensiones:
        $$
            \dmx{
                1 & 0 &  1 &  1 & 1\\
                0 & 1 & -2 & -1 & 0
            }
        $$
        Donde podemos encontrar las siguientes $3$ magnitudes adimensionales (ya que el rango de la matriz es $2$):
        $$
            \frac{h}{R},\ \frac{tv}{R},\ \frac{v}{\sqrt{gR}}
        $$
        \item Relación de la altura máxima a alcanzar el proyectil con respecto a $v$, $g$, $R$.\\\\
        Partimos de $0 = f(h, t, v, g, R)$ y por el \hyperref[thm:pi]{teorema \Pi} sabemos que tenemos una ley equivalente:
        $$
            0 = F\left(\frac{h}{R},\ \frac{tv}{R},\ \frac{v}{\sqrt{gR}}\right)
        $$
        Además, sabemos que la altura máxima $h_{max}$ se va a tomar en un tiempo determinado $t_{max}$. Por tanto tenemos la relación:
        $$
            0 = F\left(\frac{h_{max}}{R},\ \frac{t_{max}v}{R},\ \frac{v}{\sqrt{gR}}\right)
        $$
        Por el Teorema de la función implícita (TFI):
        $$
            \frac{h}{R} = G\left( \frac{tV}{R}, \frac{V}{\sqrt{gR}} \right)
        $$
        Como alcanzamos la altura máxima:
        $$
            h'(t_{max}) = 0 \implies \pd{G}{\Pi_2}\left( \frac{t_{max}V}{R}, \frac{V}{\sqrt{gR}} \right) = 0
        $$
        Aplicando de nuevo el TFI:
        $$
            \frac{t_{max} v}{R} = \varphi\left( \frac{V}{\sqrt{gR}} \right)
        $$
        Y llegamos a la relación que nos pedían:
        $$
            \frac{h_{max}}{R} = G\left( \varphi\left( \frac{V}{\sqrt{gR}} \right), \frac{V}{\sqrt{gR}} \right) = \varphi^{\star}\left( \frac{V}{\sqrt{gR}} \right)
        $$
    \end{enumerate}
\end{eg}
