% !TeX root = ../logica-matematica.tex

\chapter{Teoremas de Finitud y de la Deducci\'on.}

\section{Teorema de finitud}

\begin{thm}[Teorema de finitud]\label{thm:tf}
    Sea $L$ un lenguaje, $T$ una $L$-teoría y $F \in \for(L)$ una fórmula. Si $T \vdash F$ entonces existe $T_0 \subseteq T$ finito tal que $T_0 \vdash F$.
\end{thm}

\begin{cor}[Corolario primero al teorema \ref{thm:tf}]
    Si para $T_0 \subseteq T$, $T_0$ es coherente, entonces $T$ es coherente.
\end{cor}

\begin{cor}[Corolario segundo al teorema \ref{thm:tf}. Unión de teorías coherentes]
    Sea $L$ un lenguaje, sea $\sdf{T_i, i\in I}$ una familia de $L$-teorías, tal que $\forall i, j\in I$, $T_i \subseteq T_j$ o $T_j \subseteq T_i$. Si $T_i$ es coherente, $\forall i \in I$, entonces $\bigcup_{i\in I} T_i$ es una $L$-teoría coherente.
\end{cor}

\section{Teorema de la deducci\'on}

\begin{thm}[Teorema de la deducción]
    Sea $L$ un lenguaje, $T$ una $L$-teoría, $F$ un enunciado de $L$ y $G \in \for(L)$. Si $T \bigcup \sdf{F} \vdash G$ entonces $T \vdash F \then G$.
\end{thm}

\begin{proof}
    A completar. %%TODO: Página 29, día 17/10/2019
\end{proof}
