% !TeX root = ../logica-matematica.tex

\chapter{Teorema de Validez.}

\section{Introducción}

A lo largo del curso hemos estado interesados en las consecuencias semánticas, donde veíamos que $T \vDash F$ representaba que un enunciado $F$ se satisfacía en todos los modelos de $T$. A continuación, introdujimos la relación sintáctica ($T \vdash F$) como una versión más potente de la consecuencia semántica ($\vdash F \implies \vDash F$). Sin embargo, tenemos que comprobar que lo que hemos introducido es válido, es decir, que los teoremas son fórmulas válidas.

En esta sección vamos a ver las herramientas necesarias para demostrar que si $F$ es un teorema de $T$ ($T \vdash F$), es decir, existe una demostración de $F$ en $T$, entonces $F$ se satisface en $T$ ($T \vDash F$), es decir, $F$ se satisface en todos los modelos $\mtc A$ de $T$. Esto se enuncia formalmente con el teorema de validez.

\begin{thm}[Teorema de validez]\label{thm:tv}
    Sea $L$ un lenguaje, $T$ una $L$-teoría y $F \in \for(L)$ un enunciado.
    \begin{center}
        Si $T \vdash F \implies T \vDash F$. Informalmente:\\
        \textit{Los teoremas de $T$, que sean enunciados, se satisfacen en todos los modelos de $T$}.
    \end{center}
\end{thm}
\begin{proof}
    Más adelante. \hyperref[proof:tv]{Ir a la demostración}.
\end{proof}

Recordemos que una lógica consta de dos partes:
\begin{itemize}
    \item Una parte \textbf{sintáctica}: axiomas y teoremas ($T \vdash F$).
    \item Una parte \textbf{semántica}: estructuras y relaciones de satisfacción ($\mtc A \vDash F$, $T \vDash F$)
\end{itemize}

\begin{dfn}[Validez de una lógica]
    Decimos que una lógica es \textbf{válida} si satisface el teorema de validez (\ref{thm:tv}).
\end{dfn}

\begin{dfn}[Completitud de una lógica]
    Decimos que una lógica es \textbf{completa} si satisface el teorema de validez (\ref{thm:tv}) y su recíproco, es decir:
    $$
        T \vdash F \iff T \vDash F
    $$
\end{dfn}
\begin{obs}[Conjuntos de enunciados en una lógica completa]
    Si una lógica es completa, entonces sea $F$ un enunciado:
    $$
        \sdf{F \mid T \vdash F} = \sdf{F \mid T \vDash F}
    $$
\end{obs}

\begin{lm}[Validez de los axiomas lógicos]\label{lm:axval}
    Sea $L$ un lenguaje, los axiomas lógicos de $L$ son fórmulas válidas.
\end{lm}
\begin{proof}$ $
    \begin{enumerate}[(I)]
        \item \textsc{Tautologías}\\
        Ya lo vimos en la proposición \ref{pro:1.3}.
        \item \textsc{Axiomas de cuantificadores}
        \begin{itemize}
            \item[(2.1)] Axioma de $\forall$\\
                Sean $F, G \in \for(L)$ fórmulas, $F$ es $F(\sucn{x})$ y $G$ es $G(\sucn{x})$. Si $x$ no está libre en $G$:
                $$
                    H:\ \forall x (G \then F) \then G \then \forall x F
                $$
                Tenemos que ver que $H$ se satisface en cualquier estructura. Queremos llegar a:
                \begin{equation*}
                    \mtc A \vDash \forall x_1 \ldots \forall x_n H \text{ para toda $L$-estructura } \mtc A \tag{$\star$}
                \end{equation*}
                \begin{align*}
                    (\star) &\iff \text{ para todo } \sucn{a} \in A \implies \mtc A \vDash H(\sucn{a})\\
                            &\iff \mtc A \vDash (\forall x G \then F) (\sucn{a}) \tag{$1$} \\
                            &\iff \mtc A \vDash (G \then \forall x F) (\sucn{a}) \tag{$+$} \\
                    (+)     &\iff \text{Si } \mtc A \vDash G(\sucn{a}) \text{ entonces } \mtc A \vDash (\forall x F)(\sucn{a})
                \end{align*}
                Supongamos entonces:
                \begin{equation*}
                    \mtc A \vDash G(\sucn{a}) \tag{$2$}
                \end{equation*}
                y veamos que llegamos a
                \begin{equation*}
                    \mtc A \vDash (\forall x F)(\sucn{a}) \tag{$+++$}
                \end{equation*}
                Sea $x \neq x_i$, $(+++) \implies \forall b \in A$, $\mtc A \vDash F(b, \sucn{a})$. Llamemos $(3)$ a $\forall b \in A$, $\mtc A \vDash F(b, \sucn{a})$.\\
                Fijamos $b \in A$, por $(1)$, para este $b$ sabemos $\mtc A \vDash (G \then F)(b, \sucn{a})$.\\
                Por $(2)$, $\mtc A \vDash G(\sucn{a}) \implies \mtc A \vDash G(b, \sucn{a})$. Finalmente, como $x$ no aparece libre en $G$ podemos concluir que $(3)$ es cierto y por tanto, $(2) \implies (+++)$ con lo que se satisfacen $(+)$ y $(\star)$.

            \item[(2.2)] Axioma de sustitución\\
                Sea $F \in \for(L)$, $x \in V$m $t \in \ter(L)$ y $x$ sustituible por $t$ en $F$. Sea $\mtc A$ una $L$-estructura y $\sucn{a} \in A$.\\
                Veamos que si $\mtc A \vDash (\forall x F)(\sucn{a})$ entonces $\mtc A \vDash F(\quox{t})(\sucn{a})$.\\
                Por la proposición \ref{pro:1.2}, $\mtc A \vDash F(\quox{t})(\sucn{a}) \iff \mtc A \vDash F(t^{\mtc A}(\sucn{a}), \sucn{a})$.\\
                Como $t^{\mtc A}(\sucn{a}) \in A$, por $(1)$ tenemos que $\mtc A \vDash F(b, \sucn{a})$. En particular, también para $b = t^{\mtc A}(\sucn{a})$.
        \end{itemize}
        \item \textsc{Axiomas de igualdad}
        \begin{itemize}
            \item[(3.1)] Variables\\
            $$
                \mtc A \vDash (x=x)(a)\ \forall a \in A,\ \text{ entonces para todo } a \in A,\ a = a
            $$
            \item[(3.2)] Funciones\\
            $$
                \mtc A \vDash \forall x_1 \ldots \forall x_n \forall y_1 \ldots \forall y_n\ \ x_1 = y_1 \then \cdots \then x_n = y_n \then f x_1 \ldots x_n = f y_1 \ldots y_n
            $$
            Entonces es claro que: $\forall \sucn{a} \in A,\ \forall \sucn{b} \in A$ si $a_i = b_i$ entonces $f a_1 \ldots a_n = f b_1 \ldots b_n$.
            \item[(3.3)] Relaciones\\
            $$
                \mtc A \vDash \forall x_1 \ldots \forall x_m \forall y_1 \ldots \forall y_m\ \ x_1 = y_1 \then \cdots \then x_m = y_m \then R x_1 \ldots x_m = R y_1 \ldots y_m
            $$
            Entonces es claro que: $\forall \suc{b}{m} \in A,\ \forall \suc{b}{m} \in A$ si $a_i = b_i$ entonces $(\suc{a}{m}) \in R^{\mtc A}$ y $(\suc{b}{m} \in R^{\mtc A}) \implies (\sucn{a}) = (\suc{b}{m})$.
        \end{itemize}
    \end{enumerate}
\end{proof}

\section{Teorema de validez}

Una vez hemos demostrado que los axiomas son fórmulas válidas ya disponemos de las herramientas necesarias para demostrar el teorema de validez (teorema \ref{thm:tv}).

\begin{proof}[Demostración al teorema de validez]\label{proof:tv}
    La demostración sigue por inducción en la longitud de una demostración de $F$ en $T$.
    \begin{itemize}
        \item[$(n=1)$] Si la longitud de la demostración es $1$ o $F \in T$ o $F$ es un axioma lógico.\\
        Si $F \in T \implies \mtc A \vDash F,\ \forall \mtc A \vDash T$ y por tanto $T \vDash F$.
        Si $F$ es un axioma lógico, por el lema \ref{lm:axval} sabemos que $ \vDash F$ pues $F$ es una fórmula válida. Entonces $\mtc A \vDash F \forall \mtc A \implies T \vDash F$.

        \item[$(n>1)$]
        Supongamos que $F_n$ es $F$ y $(\sucn{F})$ es una demostración de $F$ en $T$.
        \begin{itemize}
            \item Si $F_n \in T$ o $F_n$ es un axioma lógico (como en $n=1$) entonces $T \vDash F$.
            \item Si $F_n$ es $\forall x F_i$ (aplicando \textit{Generalización} a algún $F_i$), por hipótesis de inducción, $T \vDash F_i (\iff T \vDash \overline{F_i} \text{ cualquier cierre universal de }F_i)$. Como todo cierre universal de $F_i$ también es cierre universal de $\forall x F_i$ entonces $T \vDash \forall x F_i$.
            \item Si $F_n$ se obtiene de $F_i$ y $F_j: F_i \then F_n$ aplicando \textit{Modus Ponens}, con $i, j < n$, por hipótesis de inducción:
            \begin{equation}
                T \vDash F_i\\
                T \vDash F_i \then F_n
            \end{equation}
            Veamos que $T \vDash F_n$. Sean $F_n:\ F_n(\sucn{x})$ y $F_i:\ F_i(\sucn{x})$, y $\mtc A \vDash T$, vamos a ver que $\mtc A \vDash \forall x_1 \ldots \forall x_n\ F_n$.\\
            Sean $(\sucn{a})\in A$, por (1) tenemos $T \vDash \forall x_1 \ldots \forall x_n F_i$. Para este $(\sucn{a})$ se tiene $\mtc A \vDash F_i(\sucn{a})$.
            Análogamente $\mtc A \vDash (F_i \then F_n)(\sucn{a}) \iff \mtc A (\sucn{a})$ entonces $\mtc{A} \vDash F_n(\sucn{a})$.\\
            Como tenemos el antecedente $\mtc A (\sucn{a})$, y también $\mtc A \vDash (F_i \then F_n)(\sucn{a})$ por H.I, obtenemos $\mtc{A} \vDash F_n(\sucn{a})$.
        \end{itemize}
    \end{itemize}
\end{proof}

\begin{cor}[Corolario primero al teorema \ref{thm:tv}]\label{cor:tv1}
    Sea $L$ un lenguaje, $T$ una $L$-teoría. Si existe $\mtc A \vDash T$ entonces $T$ es coherente.
\end{cor}

\begin{proof}
    Si $T$ es incoherente entonces existe un enunciado $F \in \for(L)$ tal que $T \vdash F \land \neg F$.\\
    Por el \hyperref[thm:tv]{teorema de validez}, $T \vDash F \land \neg F \implies T$ no puede tener modelos. Esto demuestra el contrarrecíproco.
\end{proof}

\begin{cor}[Corolario segundo al teorema \ref{thm:tv}]\label{cor:tv2}
    El corolario \ref{cor:tv1} es equivalente al \hyperref[thm:tv]{teorema de validez}.
\end{cor}

\begin{proof}
    No se vió.
\end{proof}
