% !TeX root = ../logica-matematica.tex

\chapter{Teorema de Igualdad.}


\section{Teorema de igualdad}

En esta sección vamos a ver algunos resultados acerca de igualdades. Vamos a apoyarnos en la proposición \ref{pro:2.1} por lo que convendría revisarla.

\begin{lm}[$=$ en las consecuencias sintácticas]\label{lm:igualsintactico}
    Sea $L$ un lenguaje, $t_i \in \ter(L)$ con $i = 1, 2, 3$, entonces:
    \begin{gather*}
        \vdash t_1 = t_1\\
        \vdash t_1 = t_2 \then t_2 = t_1\\
        \vdash t_1 = t_2 \land t_2 = t_3 \then t_1 = t_3
    \end{gather*}
\end{lm}

\begin{proof}
    Vamos a proceder con una demostración abreviada haciendo uso de la proposición \ref{pro:2.1}.
    \begin{align}
        &x=x &\text{Axioma de igualdad 3.1}\\
        &\forall x\ x=x&\text{Gen(1)}\\
        &\forall x\ x=x \then (x=x)(\quo{t_1}{x})&\text{Axioma de sustitución}\\
        &(x=x)(\quo{t_1}{x}): t_1=t_1&\text{MP(3, 1)}
    \end{align}
    \begin{align}
        &x=y \then y=x   &\vdash x=y \then y=x\\
        &\forall x\ F_1   &   \text{Gen(1)}\\
        &\forall x\ F_1 \then (x=y \then y=x)(\quo{t_1}{x})   &   \text{Axioma de sustitución}\\
        &t_1 = y \then y = t_1   &   \text{MP(3,2)}\\
        &\forall y\ F_4   &   \text{Gen(4)}\\
        &\forall y\ F_4 \then F_4(\quo{t_2}{y})   &   \text{Axioma de sustitución}\\
        &t_1 = t_2 \then t_2 = t_1   &   \text{MP(6, 5)}
    \end{align}
    \begin{align} %TODO: Demostrar
        &x=y \then y=x   &\text{se demuestra de forma análoga}
    \end{align}
\end{proof}

\begin{thm}[Teorema de igualdad]\label{thm:igualdad}
    Sea $L$ un lenguaje, $t, s_1, s_2 \in \ter(L)$, $x \in V$, $F \in \for(L)$ tal que $x$ es sustituible por $s_i$ en $F$. Sea $T$ una $L$-teoría. Entonces:
    \begin{enumerate}[(a)]
        \item $T \vdash s_1 = s_2 \implies T \vdash t(\quo{s_1}{x}) = t(\quo{s_2}{x})$
        \item $T \vdash s_1 = s_2 \implies T \vdash F(\quo{s_1}{x}) = F(\quo{s_2}{x})$
    \end{enumerate}
\end{thm}

\begin{proof}$ $
    \begin{enumerate}[(a)]
        \item Se da una demostración abreviada de $t(\quo{s_1}{x}) = t(\quo{s_2}{x})$ en $T$. Por inducción sobre la complejidad de $t$.\\\\

            Si $t\in V$:\\
                \begin{itemize}
                    \item $t$ es $x \implies t(\quo{s_i}{x})$ es $s_i$, y por tanto $T \vdash s_1 = s_2$ por hipótesis.
                    \item $t$ es $v \neq x \implies t(\quo{s_1}{x})$ es $t$, y por tanto $T \vdash t = t$ por el lema \ref{lm:igualsintactico}.
                    \item $t$ es $c$ se demuestra igual que el caso $v \neq x$
                \end{itemize}
            Si $t$ es $f t_1 \ldots t_m$. Por hipótesis de inducción se cumple que $T \vdash t_i(\quo{s_1}{x}) = t_i(\quo{s_2}{x})$, vamos a dar una demostración abreviada.\\
            \begin{align*}
                & t_1(\quo{s_1}{x}) = t_1(\quo{s_2}{x})                                                                                  &\ T\vdash \tag{$F_1$}\\
                & \vdots                                                                                                                 & \tag{$\cdots$}\\
                & t_m(\quo{s_1}{x}) = t_m(\quo{s_2}{x})                                                                                  &\ T\vdash  \tag{$F_m$}\\
                & x_1 = y_1 \then \cdots \then x_m = y_m \then f x_1 \ldots x_m = f y_1 \ldots y_m                                       & \text{Axioma de igualdad 3.2} \tag{$F_{m+1}$}\\
                & \forall x_1\ F_{m+1}                                                                                                   & \text{Gen($F_{m+1}$)} \tag{$F_{m+2}$}\\
                & \forall x_1\ F_{m+1} \then F_{m+1}(\quo{t_1(\quo{s_1}{x})}{x_1})                                                       & \text{Axioma de sustitución} \tag{$F_{m+3}$}\\
                & F_{m+1}(\quo{t_1(\quo{s_1}{x})}{x_1})                                                                                  & \text{MP($m+3$, $m+2$)}\tag{$F_{m+4}$}\\
                & \forall y_1\ F_{m+4}                                                                                                   & \text{Gen($F_{m+4}$)} \tag{$F_{m+5}$}\\
                & \forall y_1\ F_{m+4} \then F_{m+4}(\quo{t_1(\quo{s_2}{y})}{y_1})                                                       & \text{Axioma de sustitución} \tag{$F_{m+6}$}\\
                & F_{m+4}(\quo{t_1(\quo{s_2}{y})}{y_1})                                                                                  & \text{MP($m+6$, $m+5$)}\tag{$F_{m+7}$}\\
                & x_2 = y_2 \then \cdots \then x_m = y_m \then & \text{MP($m+7$, $1$)}\tag{$F_{m+8}$}\\
                &\then f t_1(\quo{s_1}{x}) x_2 \ldots x_m = t_1(\quo{s_2}{y}) f y_2 \ldots y_m &\\
                &\text{repetir el proceso m veces}&
            \end{align*}
        \item Demostramos por inducción sobre la complejidad de $F$. Como $F$ es $R t_1 \ldots t_m$ siendo $R$ un símbolo de relación $m$-aria (incluyendo $=$), basta demostrar:
            \begin{equation*}
                \tag{$+$} T \vdash F(\quo{s_1}{x}) \then F(\quo{s_2}{x})
            \end{equation*}
            ya que $T \vdash s_1 = s_2 \implies T \vdash s_2 = s_1$. De $(+)$ se obtiene:
            \begin{equation*}
                \tag{$++$} T \vdash F (\quo{s_2}{x}) \then F(\quo{s_1}{x})
            \end{equation*}
            y finalmente de $(+)$ y $(++)$ acabamos por obtener:
            $$
                T \vdash F(\quo{s_1}{x}) \bicond F(\quo{s_2}{x})
            $$
            por tautologías.
            \begin{itemize}
                \item Si $F$ es atómica, la demostración es análoga al apartado (a) usando el axioma de igualdad 3.3 en vez de 3.2.
                \item Si $F$ es $\neg G$.\\
                      Por hipótesis de inducción $T \vdash s_1 = s_2 \implies T \vdash G(\quo{s_1}{x}) \bicond G(\quox{s_2})$. De aquí podemos hallar $T \vdash \neg G(\quox{s_1}) \bicond \neg G(\quox{s_2})$ por tautologías.
                      Como $\neg(G(\quox{s_1}))$ es $(\neg G)(\quox{s_1})$ hemos terminado.
                \item Si $F$ es $G \then H$, por hipótesis de inducción y observando que $(G \then H)(\quox{s_i})$ es $(G(\quox{s_i}) \then H(\quox{s_i}))$ habríamos terminado.
                \item Si $F$ es $\forall y G$ distinguimos dos casos. Si $x$ no aparece libre, entonces $F(\quox{s_i})$ es $F$ y $T \vdash F \bicond F$ es una tautología.\\
                      Si $x$ aparece libre, entonces $x \neg y$ y como $x$ es sustituible por $s_i$ con $i = 1,2$ también tenemos que $y$ no aparece en $s_i$.\\
                      Además $(\forall y G)(\quox{s_i})$ es $\forall y (G(\quox{s_i}))$. Por hipótesis de inducción, $T \vdash s_1 = s_2 \implies T \vdash G(\quox{s_1}) \bicond G(\quox{s_2})$. Y como ya vimos en clase, podemos generalizar a ambos lados del bicondicional, con lo que obtenemos:
                      $$
                            T \vdash \forall y (G\quox{s_1}) \bicond \forall y (G\quox{s_2})
                      $$
                      es decir,
                      $$
                            T \vdash F \bicond F
                      $$
                      que es lo que queríamos demostrar
            \end{itemize}
    \end{enumerate}
\end{proof}

\begin{cor}[al teorema de igualdad (teorema \ref{thm:igualdad})]
    Con las mismas hipótesis que el teorema:
        \begin{enumerate}[(a')]
            \item $T \vdash s_1 = s_2 \then t(\quox{s_1}) \then t(\quox{s_2})$
            \item $T \vdash s_1 = s_2 \then F(\quox{s_1}) \then F(\quox{s_2})$
        \end{enumerate}
\end{cor}

\begin{proof}
    Si $s_1 = s_2$ fuera un enunciado, (a') y (b') se obtienen de (a) y (b) usando el teorema de la deducción (teorema \ref{thm:td}). En general, si $s_i$ es $s_i(\sucn{x})$, construimos $s_i'$ como $s_i'(\sucn{c})$ sustituyendo cada $x_j$ por el $c_j$ correspondiente.\\
    Por la proposición \ref{pro:2.3}, sea $L' = L \cup \sdf{\sucn{c}}$ y $s_i' \in \ter(L')$, tenemos que:
    \begin{itemize}
        \item[] (a')$\iff T \vdash s'_1 = s'_2 \then t(\quox{s'_1}) \then t(\quox{s'_2})$
        \item[] (b')$\iff T \vdash s'_1 = s'_2 \then F(\quox{s'_1}) \then F(\quox{s'_2})$
    \end{itemize}
    También sabemos que $T \vdash s_1 = s_2 \iff T \vdash s_1' = s_2'$. Como ahora son enunciados basta aplicar el teorema \ref{thm:td} y obtenemos el resultado que queremos demostrar.\\
    Además, también hemos demostrado que:
        \begin{gather*}
            T \vdash t(\quox{s_1}) = t(\quox{s_2}) \implies T \vdash t(\quox{s_1'}) = t(\quox{s_2'})\\
            T \vdash F(\quox{s_1}) = F(\quox{s_2}) \implies T \vdash F(\quox{s_1'}) = F(\quox{s_2'})
        \end{gather*}
\end{proof}
