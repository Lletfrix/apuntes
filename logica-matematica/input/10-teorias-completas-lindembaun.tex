% !TeX root = ../logica-matematica.tex

\chapter{Teorías completas y teorema de Lindembaun.}

\section{Introducción}

Hasta ahora habíamos visto que una teoría completa era aquella que cumplía el teorema de validez y su recíproco. Podemos enunciar así una primera forma del teorema de Completitud.

\begin{thm}[Teorema de completitud. Primera forma.]\label{thm:tc1}
    Sea $L$ un lenguaje, $T$ una $L$-teoría y $F \in \for(L)$ entonces $T \vDash F \implies T \vdash F$.
\end{thm}

Sin embargo, podemos enunciarlo de otra forma, que es el recíproco al corolario \ref{cor:tv1}, ya que según el corolario 2 al teorema de validez, son equivalentes.
\begin{thm}[Teorema de completitud. Segunda forma.]\label{thm:tc2}
    Sea $L$ un lenguaje y $T$ una $L$-teoría, si $T$ es coherente entonces $T$ tiene un modelo.
\end{thm}


\begin{proof}[Demostración de que las dos formas del teorema de completitud son equivalentes]
    Vamos a ver que la primera forma del TC $\iff$ la segunda forma del TC.
    \begin{itemize}
        \item[$\implies$] Vamos a ver que si $T$ no tiene modelos, entonces $T$ es incoherente.\\
        Sea $F \in \for(L)$, $F$ un enunciado entonces, como $T$ no tiene modelos, trivialmente $T \vDash F \land \neq F$. Por la primera forma (teorema \ref{thm:tc1}) $T \vDash F \land \neq F \implies T \vdash F \land \neq F$ y entonces $T$ es incoherente.

        \item[$\implied$] Vamos a ver que sea $F \in \for(L)$, si $T \cancel\vdash F$ entonces $T \cancel\vDash F$, o expresado de otra forma:
        $$
            T \cancel\vdash F \implies T \cancel\vDash G, \text{ con $G$ el cierre universal de F}
        $$
        Puesto que $G$ es un enunciado podemos hacer uso del \hyperref[cor:td1]{corolario 1 al TD}. Si $T \cancel\vdash G$ entonces $T \cup \sdf{\neq G}$ es coherente.\\
        Si es coherente, por la segunda forma (teorema \ref{thm:tc2}) $T \cup \sdf{\neq G}$ tiene un modelo. Sea $\mtc A \vDash T \cup \sdf{\neq G}$, entonces $\mtc A \vDash T$ y $\mtc A \cancel\vDash G \implies T \cancel\vDash G \implies T \cancel\vDash F$.
    \end{itemize}
\end{proof}
Sin embargo, la demostración del teorema se verá más adelante en el curso. Se encuentra en \hyperref[proof:tc2]{el capítulo 12}.

\section{Teorías completas}

Vamos a ver que para cualquier teoría podemos añadir un conjunto de axiomas que la haga completa. Por ejemplo podemos pasar de la teoría de grupos (que ya vimos que es incompleta)
a grupos abelianos divisibles y sin torsión. GADST es una teoría completa.

\begin{eg}[Teorías completas e incompletas]
    \begin{enumerate}[(1)]
        \item Teoría de grupos:\\
        La teorías de grupos no es completa. Si $T \vdash \forall x \forall y\ xy = yx \implies T \vDash \forall x \forall y\ xy=yx$ por el teorema de validez, sin embargo esto no es cierto. Además, su negación tampoco lo es. Por tanto la teoría de grupos no es completa (no hay modelos).
        \item $T = \sdf{F \land \neq F}$ con $F$ un enunciado:\\
        Es una teoría completa. Toda teoría incoherente es trivialmente completa.
        \item Teoría de una estructura:\\
        La teoría de una estructura es completa. Sea $\mtc A$ una $L$-estructura, $T = \te(\mtc A) = \sdf{ F \in \for(L) \mid F \text{ enunciado y } \mtc A \vDash F}$.\\
        Sea $F\in \for(L)$ un enunciado, entonces $A \vDash F$ o $A \vDash \neq F \implies F \in \te(A)$ o $\neq F \in \te(A)$ y por tanto $\te(A) \vdash F$ o $\te(A) \vdash \neq F$ con una demostración de longitud $1$.
    \end{enumerate}
\end{eg}

\begin{obs}
    Por lo general es muy difícil encontrar teorías completas.
\end{obs}

\begin{pro}\label{pro:3.2}
    Sea $L$ un lenguaje, $T$ una $L$-teoría completa. Entonces:
    \begin{enumerate}
        \item Todos los modelos de $T$ satisfacen los mismos enunciados.
        \item Todos los modelos de $T$ son elementalmente equivalentes.
    \end{enumerate}
\end{pro}

\begin{proof}
    \begin{enumerate}
        \item Queremos ver que $\mtc A,\ \mtc B\ \vDash T$, $F \in \mathrm{Enun}(L)$.\\
        $T$ completa $\implies$ $T \vdash F$ o $T \vdash \neg F$.
        \begin{itemize}
            \item Si $T \vdash F$ entonces por el teorema de validez $T \vDash F \implies \mtc A,\ \mtc B \vDash F$.
            \item Si $T \vdash \neg F$ entonces $T \vDash \neg F \implies \mtc A \vDash \neg F \land \mtc B \vDash \neg F$.
        \end{itemize}
        \item Directa desde 1 por definición.
    \end{enumerate}
\end{proof}

\begin{pro}\label{pro:3.3}
    Sea $L$ un lenguaje, $T$ una $L$-teoría \textit{coherente}. Las siguientes propiedades son equivalentes:
    \begin{enumerate}[(a)]
        \item $T$ es completa.
        \item $\forall F,G \in \mathrm{Enun}(L)$, si $T \vdash F \lor G$ entonces $T \vdash F$ o $T \vdash G$.
        \item $T' = \sdf{F \in \mathrm{Enun}(L) \mid T \vdash F}$ es una $L$-teoría coherente maximal.
    \end{enumerate}
\end{pro}

\begin{proof}$ $
    \begin{enumerate}
        \item[(a) $\implies$ (b)] Sabemos que $T \vdash F \lor G \iff T \vdash \neg F \then G$. Además, supongamos que $T \cancel\vdash G \implies T \vdash \neg F$ (por ser teoría completa), entonces por \textit{Modus Ponens} $T\vdash G$.\\
        Si por el contrario $T \vdash F$ entonces ya estaría demostrado.
        \item[(b) $\implies$ (c)] Vamos a ver primero que $T'$ es coherente. Sea $T'$ como en el teorema, si no fuera coherente entonces $T' \vdash F \land \neg F$ para algún $F$ enunciado de $L$.\\
        Sea $(\sucn{F})$ una demostración de $F \land \neg F$ en $T'$, como $F_i \in T'$ entonces $T \vdash F_i$. Por tanto $(\sucn{F})$ también es una demostración de $F \land \neg F$ en $T$.
        Pero $T$ es coherente por hipótesis, con lo que llegamos a una contradicción. Vamos a ver ahora que es maximal.\\

        Sea $T''$ una $L$-teoría $T' \subset T''$. Sea $F \in T''$ tal que $F \notin T' \implies T \cancel\vdash F$. Como $T \vdash F \lor \neg F$, por (b), $T \vdash \neg F \implies \neg F \in T' \implies F,\neg F \in T'' \implies T''$ incoherente, que entra en contradicción con lo que acabamos de demostrar.

        \item[(c) $\implies$ (a)]
        Sea $F$ un enunciado de $L$. $T \cancel\vdash F \implies F \notin T' \implies T' \cup \sdf{F} \supset T' \implies T' \cup \sdf{F} \text{ incoherente } \implies \text{ (por colorario al TD) } T' \vdash \neg F \implies T \vdash F \implies T$ es completa.
    \end{enumerate}
\end{proof}

\section{Teorema de Lindembaum}

\begin{thm}[Teorema de Lindembaum]\label{thm:lindembaun}
    Sea $L$ un lenguaje, $T$ una $L$-teoría coherente. Entonces existe una $L$-teoría $T'$ que contiene a $T$ y es completa y coherente.
\end{thm}
\begin{obs}
    Si $T$ es coherente $\implies T$ tiene un modelo (Teorema de completitud). Usando esto la demostración del teorema sería trivial:
    $$
        T \text{ coherente } \implies \exists \mtc A \vDash T \implies T \subseteq \te(A)
    $$
    donde $\te(A)$ es completa y coherente (ya que existe un modelo). Sin embargo, como aún no hemos probado el teorema de completitud vamos a probarlo de otra forma.
\end{obs}

\begin{proof}
    Sea $\Gamma' = \sdf{T_i,\ L-\text{teoría } \mid T \subseteq T_i \text{ y } T_i \text{ es coherente}}$. Veamos que estamos ante la hipótesis del lema de Zorn:
    \begin{enumerate}
        \item $T \in\Gamma$ con lo que $\Gamma \neq \varnothing$.
        \item $\subseteq$ define un orden parcial en $\Gamma$.
        \item Sea $\sdf{T_i}_{i\in I}$ una cadena en $\Gamma$, $T \subseteq T_i \forall i \in T$. Sea $T'' = \bigcup_{i\in I} T_i$. Por el corolario al teorema de finitud, la unión de una cadena de teorías coherentes es coherente. Es decir, toda cadena tiene una cota superior.
    \end{enumerate}
    Entonces por el lema de Zorn existe $T' \in \Gamma$ un elemento maximal. Veamos que $T'$ es coherente, $T \subset T'$ y que es completa. Sea $F$ un enunciado de $L$:
    $$
        T' \cancel\vdash F \implies F \notin T' \implies T' \cup \sdf{F} \supset T' \implies T' \cup{F}\notin \Gamma \implies T' \cup \sdf{F} \text{ incoherente } \implies T' \neg F \implies T'\text{ completa}
    $$
\end{proof}
