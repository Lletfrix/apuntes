% !TeX root = ../galois.tex

\chapter{Anillos, polinomios y cuerpos}

\section{Anillos}

A lo largo de este curso se supondrán conocidos los contenidos de la asignatura \textit{Estructuras Algebraicas}, se pueden encontrar unos apuntes de los mismos en: \url{https://github.com/knifecake/apuntes/raw/master/ea/apuntes-ea.pdf}.

\begin{dfn}[Anillo]
	Un \textbf{anillo} es una terna $(A, +, \cdot)$ donde $+: A \times A \to A$ es una operación a la que llamamos suma, $\cdot: A \times A \to A$ es otra operación a la que llamamos producto y se verifican las siguientes propiedades
	\begin{enumerate}
		\item El par $(A, +)$ es un grupo abeliano
		\item El producto $\cdot$ es asociativo
		\item Se cumplen las propiedades distributivas:
		\begin{align}
			\forall a, b , c \in A,\ a\cdot (b + c) = a\cdot b + a \cdot c \\
			\forall a, b , c \in A,\ (a + b) \cdot c = a\cdot c + b \cdot c
		\end{align}
	\end{enumerate}
\end{dfn}

Con la operación $+$ tenemos las siguientes propiedades
\begin{enumerate}
	\item Asociatividad: $(a+b)+c = a+(b+c)$
	\item Elemento neutro aditivo: $\exists! \0 \in A \mid \0+a = a$
	\item Elemento inverso aditivo: $\forall a \in A, \exists -a \in A \mid a + (-a) = \0$
	\item Conmutatividad aditiva: $\forall a, b \in A,\ a+b = b+a$
\end{enumerate}

Con la operación $\cdot$ tenemos las siguientes propiedades
\begin{enumerate}
	\item Asociatividad: $a\cdot (b \cdot c) = (a \cdot b) \cdot c$
	\item No siempre existe el neutro multiplicativo: $\1 \in A \mid a\cdot 1 = 1 \cdot a = a$
	\item No siempre el producto es conmutativo.
	\item No siempre existe inverso multiplicativo: $\inv{a} \mid a\cdot \inv{a} = \1$
	\item No siembre se da la conmutatividad multiplicativa: $a \cdot b = b\cdot a$
\end{enumerate}

\begin{pro}[Producto con 0 en anillos]
	$\forall a \in A,\ a\cdot \0 = \0$
\end{pro}

\begin{proof}
	$a \cdot \0 = a \cdot(\0 + \0) = a\cdot \0 + a\cdot \0 \implies \0 = a\cdot \0$
\end{proof}

Además, a lo largo de este curso vamos a referirnos únicamente a los anillos conmutativos con unidad (o unitario), que cumplen las siguientes definiciones.

\begin{dfn}[Anillo con unidad o anillo unitario]
	Sea $(A, +, \cdot)$ un anillo. Decimos que es un anillo con unidad o un \textbf{anillo unitario} si tiene elemento neutro multiplicativo, es decir, si $\exists \1 \in A \mid \forall a \in A, \1 a = a \1 = a$.
\end{dfn}

\begin{dfn}[Anillo conmutativo]
    Sea $(A, +, \cdot)$ un anillo. Decimos que es un \textbf{anillo conmutativo} si se cumple que:
    $$
        r\cdot s = s\cdot r,\ \forall r,s \in A
    $$
\end{dfn}

\begin{eg}[Ejemplos de anillos]
    $\Z$, $\Q$, $\R$ y $\C$ con la suma y producto usual verifican todas las definiciones de anillo, anillo conmutativo y anillo unitario.
\end{eg}

Vamos a considerar además el concepto de anillo de polinomios:
\begin{dfn}[Anillo de polinomios]
    Sea $R$ un anillo, definimos el \textbf{anillo de polinomios} $R[x]$ como:
    $$
        R[x] = \left\{ \sum_{i > 0}^{n} a_i \cdot x^i \mid a_i \in R,\ n \in \N \right\}
    $$
    Es fácil ver que $R[x]$ es un anillo pues la suma y el producto son transitivas y asociativas.
\end{dfn}
\begin{obs}
    Vamos a considerar algunas definiciones y convenciones menores.\\
    \begin{enumerate}
        \item Sea $p \in R[x]$, $p$ es un polinomio y escribimos:
        $$
            p(x) = a_0 + a_1 x + \ldots + a_n x^n
        $$
        Donde llamamos \textit{coeficientes} del polinomio a los $a_i$.
        \item Sea $p \in R[x] = \sum_{i > 0} a_i x^i$, denominamos grado de $p$ a:
        $$
            \delta(p) = \max \left\{ i \mid a_i \neq 0 \right\}
        $$
        \item Sea $p \in R[x] = a_0 + a_1 x + \ldots + a_n x^n$, llamamos \textit{coeficiente director} al coeficiente del término de mayor grado ($a_n$).
        \item  Sea $p \in R[x] = a_0 + a_1 x + \ldots + a_n x^n$, llamamos \textit{termino independiente} al coeficiente libre ($a_0$).
        \item Sea $p \in R[x]$ con todos los coeficientes nulos, entonces $p$ es el \textit{polinomio cero}.
        $$
            0 = \sum_{i > 0} 0 \cdot x^n
        $$
        Por convención, $\delta(0) = -\infty$.
    \end{enumerate}
\end{obs}

\begin{dfn}[Polinomio mónico]
    Sea $R[x]$ un anillo de polinomios, decimos que $p \in R[x]$ es \textbf{mónico} si y sólo si su \textit{término director} es $1$.
\end{dfn}

\begin{dfn}[Divisor de cero]
    Sea $R$ un anillo, decimos que $r \in R$ es un \textbf{divisor de cero} si satisface:
    $$
        \exists s \in R,\ s\neq 0\ :\ r\cdot s = \0
    $$
\end{dfn}
\begin{dfn}[Unidad de un anillo]
    Sea $R$ un anillo, decimos que $r \in R$ es una \textbf{unidad} si satisface:
    $$
        \exists s \in R\ :\ r\cdot s = \1
    $$
    Decimos entonces que $r \in \uds{R}$, con $\uds{R} = \left\{a \mid a\text{ es una unidad}\right\}$
\end{dfn}

\begin{dfn}[Dominio de integridad]
    Sea $R$ un anillo, $R$ es un \textbf{dominio de integridad} si no tiene divisores de $\0$.
\end{dfn}

\begin{dfn}[Cuerpo]
    Sea $(A, +, \cdot)$ un anillo conmutativo con unidad. Diremos que $A$ es un cuerpo si $A^{\times} = A\setminus \{\0\}$ es cerrado por la segunda operación (el \textit{producto}).
\end{dfn}

\begin{obs} $ $
    \begin{itemize}
        \item $R$ es un cuerpo si $\uds{R} = R$.
        \item $\1 \in \uds{R}$, para todo $R$ anillo unitario.
    \end{itemize}
\end{obs}

\begin{pro}[Cuerpo y dominio de integridad]\label{pro:cuerpoDI}
    Sea $R$ un cuerpo, entonces $R$ es un dominio de integridad.
\end{pro}
\begin{proof}
    Vamos a ver que $R$ no tiene divisores de $\0$. Sea $r \in R^{\times} = R\setminus \{\0\}$, supongamos $\exists s \in R^{\times}$ tal que:
    $$
        r \cdot s = 0
    $$
    Como $r \in \uds{R} = R^{\times}$ pues $R$ es un cuerpo, entonces, $\exists t \in R$ tal que $t\cdot r = r\cdot t = \1$. Por tanto:
    $$
        \0 = t \cdot (r \cdot s) = (t \cdot r) \cdot s = \1 \cdot s = s
    $$
    Y $s = \0$ contradice la hipótesis. Concluimos con que $\not \exists r, s \in R$ tal que $r \cdot s = \0$
\end{proof}

\begin{pro}[Dominio de integridad en anillos de polinomios]
    Sea $R$ un anillo. Si $R$ es un dominio de integridad, entonces $R[x]$ es un dominio de integridad.
\end{pro}
\begin{proof}
    Sean $f, g \in R[x]^{\times}$, y $a_m, b_k$ sus términos directores respectivamente. Como $R$ es un dominio de integridad, $a_m \cdot b_k \neq \0$, que coincide con el término director de $f \cdot g$ y no es nulo. Por tanto, $R[x]$ es un dominio de integridad.
\end{proof}

\begin{pro}[Propiedad de cuerpo en anillos de polinomios]
    $R[x]$ nunca es un cuerpo.
\end{pro}
\begin{proof}
    Solo hay que comprobar que aunque $f(x) = x \in R[x]$, $f(x) \notin \uds(R[x])$. Y por tanto $\uds(R[x]) \neq R[x]$, lo que nos dice que $R[x]$ no es un cuerpo.
\end{proof}

\begin{pro}[Unidades en anillos de polinomios]
    Sea $R$ un anillo, si $R$ es un dominio de integridad, entonces $\uds{R} = \uds{R[x]}$.
\end{pro}

\begin{obs}
    Podemos definir anillos como \textit{extensión} de otros, al igual que hicimos con los anillos de polinomios:\\
    \begin{itemize}
        \item $\Z[\sqrt{d}] = \left\{a + b\sqrt{d} \mid a,b \in \Z \right\}$, con $d \neq e^2,\ \forall e \in \Z$ es un anillo y un dominio de integridad, pero no es un cuerpo.
        \item $\Q[\sqrt{d}] = \left\{a + b\sqrt{d} \mid a,b \in \Q \right\}$, con $d \neq e^2,\ \forall e \in \Z$ es un cuerpo. Decimos que $\left\{ 1, \sqrt{d} \right\}$ es una $\Q$-base de $\Q[\sqrt{d}]$, pues todos los elementos de $\Q[\sqrt{d}]$ se pueden expresar como combinación lineal de los elementos de la $\Q$-base.
    \end{itemize}
\end{obs}

\begin{dfn}[Subanillo]
    Sea $R$ un anillo, $S \subseteq R$, $\1 \in S$. Decimos que $S$ es un \textbf{subanillo} si:
    \begin{itemize}
        \item $S$ es cerrado por suma y producto.
        \item Todo elemento tiene opuesto, es decir, $\forall a \in S, \exists b \in S\ :\ a + b = \0$.
    \end{itemize}
\end{dfn}

\begin{dfn}[Subcuerpo]
    Sean $R$ un cuerpo, $S \subseteq R$. Decimos que $S$ es un \textbf{subcuerpo} si:
    \begin{itemize}
        \item $S$ es un subanillo de $R$
        \item Todo elemento no nulo tiene inverso, es decir, $\forall a \in S^{\times}, \exists b \in S^{\times}\ :\ a \cdot b = \1$
    \end{itemize}
\end{dfn}

\begin{eg}[Ejemplos de subanillos y subcuerpos]$ $
    \begin{itemize}
        \item $\Z$ es subanillo de $\Q$
        \item $\Q$ es subcuerpo de $\R$ y $\C$
        \item $\Z[\sqrt{d}]$ es subanillo de $\Q[\sqrt{d}]$
    \end{itemize}
\end{eg}

\section{Ideales}

\begin{dfn}[Ideal]
    Sea $R$ un anillo, e $I \subseteq S$. $I$ es un \textbf{ideal} si:
    \begin{enumerate}
        \item $\forall a, b \in I,\ a - b \in I$
        \item $\forall r \in R,\ \forall a \in I$ se satisface: $r\cdot a \in I$
    \end{enumerate}
    Los ideales triviales son $\{\0\}$ y $R$.
\end{dfn}

\begin{obs}
    Sea $R$ un anillo, denotamos al ideal generado por $a \in R$ como
    $
        \gen{a}
    $
\end{obs}

\begin{pro}[Ideal propio]
    Sea $R$ un anillo, $I$ un ideal:
    $$
        I \subsetneq R \iff \1 \in I \iff I \cap \uds{R} \neq \varnothing
    $$
\end{pro}

\begin{obs}
    Sea $R$ un anillo, $I \leq R$ un ideal:
    $$
        I \leq R \iff I \cap \uds{R} = \varnothing
    $$
    $$
        I = R \iff I \cap \uds{R} \neq \varnothing
    $$
\end{obs}


\begin{pro}[Ideales y cuerpos]
    Sea $R$ un cuerpo, y sea $I$ un ideal de $R$ (escribimos $I \leq R$), entonces $I = \{\0\}$ o $I = R$, ($I$ es impropio).\\
    El recíproco también es cierto.
\end{pro}

\begin{proof}$ $
    \begin{itemize}
        \item $\left( \implies \right)$
        $$
            R \text{ cuerpo} \implies \uds{R} = R^{\times} \implies I = \uds{R} \cup \{\0\} \text{ o trivialmente } I = \{\0\}
        $$
        \item $\left( \Longleftarrow \right)$\\
        Sea $a \in R^{\times}$, $a \in \gen{a} \leq R$
        $$
            \{\0\} \neq I = \gen{a},\text{ entonces } I = R \implies \exists u \in I \cap \uds{R} \neq \varnothing \implies u \in \gen{a} \implies u = a\cdot r, \text{ con } r \in R
        $$
        y por tanto:
        $$
            1 = u \cdot u^{-1} = a \cdot r \cdot u^{-1} \implies a \in \uds{R} \implies R \text{ es un cuerpo}
        $$
    \end{itemize}
\end{proof}

\begin{eg}[Ejemplos de ideales]$ $
    \begin{enumerate}
        \item $n\Z \leq \Z$
        \item $I = \left\{ f \in \Z[x] \mid \text{el termino independiente de f es par} \right\}$
    \end{enumerate}
\end{eg}

\begin{dfn}[Ideal principal]
    Sea $R$ un anillo, $a \in R$ un elemento. El ideal generado por $a$:
    $$
        \gen{a} = \left\{a\cdot r \mid r \in R\right\} = aR
    $$
    se denomina \textbf{ideal principal} generado por $a$.
\end{dfn}

\begin{pro}[Propiedades de ideales] Sea $ R $ un anillo e $ I \leq R $ un ideal.
    \begin{enumerate}
        \item Sean $I, J \leq R$ ideales, entonces $I+J = \left\{a+b \mid a \in I,\ b \in J\right\} \leq R$ es un ideal.
        \item Sea $\mbf{a} \in \R^n$, entonces $I = \gen{\mbf{a}} = \left\{ a_1r_1 + \cdots + a_nr_n \mid r_i \in R \right\} \leq R$ es un ideal.
        \item $\quo{R}{I} = \left\{r + I \mid r \in R\right\}$ es un anillo.
        \item \label{pro:correspondencia}(Teorema de correspondencia) Existe una biyección de la forma:
        \begin{align*}
            \{J \leq R \mid I \subseteq J \subseteq R\} &\longrightarrow \{\quo{J}{I} \leq \quo{R}{I}\}\\
             J \appad{80mu} &\longmapsto \appad{4mu} \{r + I \mid r \in J\}
        \end{align*}
    \end{enumerate}
\end{pro}

\begin{obs}
    En particular, si en $R$ todo ideal es principal e $I \leq R$, en $\quo{R}{I}$ todo ideal es principal.
\end{obs}

\begin{ex}[H1.5]
    Sea $n$ un número natural. Prueba que $\Z_n = \ZnZ$ es un cuerpo si y sólo si $n$ es primo.\\\\
    \begin{itemize}
        \item $\left(\ \Longleftarrow\ \right)$\\\\
        $ n $ primo $ \implies \forall k:\ 0 < k < n \text{ se cumple que } \mathrm{mcd}(k, n) = 1$, y por Bezout:
        $$
            1 = ka + nb,\ \text{ con } a, b \in \Z
        $$
        Donde el término $nb \equiv 0$ en $\ZnZ$ y por tanto queda $1 = ka$, lo que quiere decir que $k$ es el inverso de $a$ en $\ZnZ$.
        \item $\left(\implies\right)$\\\\
        Partimos de que $\ZnZ$ es cuerpo, por la proposición \ref{pro:cuerpoDI} sabemos que $\ZnZ$ es un dominio de integridad. Supongamos $n$ no primo, entonces $n = a\cdot b$, entonces:
        $$
            n \equiv \0 (\mathrm{mod}\ n) \implies \0 = (a + n\Z)(b + n\Z)
        $$
        Pero es imposible, ya que $a$ y $b$ serían divisores de $\0$ pero estamos en un dominio de integridad. Por tanto, $n$ es necesariamente primo.
    \end{itemize}
\end{ex}

\begin{ex}[H1.12]
    ¿Cuántos elementos tiene el anillo $\quo{\Z[i]}{\gen{2i}}$?¿Se trata de un cuerpo?\\\\
    Comenzamos escribiendo los conjuntos que forman parte del cociente:
    $$
        \Z[i] = \setdef{a + bi \mid a, b \in \Z}
    $$
    $$
        \gen{2i} = \gen{2} = 2\Z[i] = \setdef{2(a+bi) \mid a,b \in \Z} = (2\Z)[i] = \setdef{a + bi \mid a, b \in 2\Z}
    $$
    El conjunto cociente es por tanto:
    $$
        \quo{\Z[i]}{\gen{2i}} = \quo{\Z[i]}{2\Z[i]} = \left\{ a+bi + 2\Z[i] \mid a, b \in \Z \right\}
    $$
    Donde se tiene que:
    $$
        a + bi + 2\Z[i] = a_1 + b_1i + 2\Z[i] \iff a-a_1 \in 2\Z \text{ y } b-b_1 \in 2\Z \iff \setdef{a+bi+2\Z[i] \mid a, b \in \setdef{0, 1}} = \setdef{0, 1, i, 1+i}
    $$
    De esta forma vemos que el anillo tiene $4$ elementos y además no es un cuerpo ya que $i$ no tiene inverso.
\end{ex}

\begin{dfn}[Ideal primo]
    Sea $R$ un anillo e $I \leq R$ un ideal, diremos que $I$ es un \textbf{ideal primo} si:
    $$
        a\cdot b \in I \implies a \in I \text{ ó } b \in I
    $$
\end{dfn}

\begin{dfn}[Ideal maximal]
    Sea $R$ un anillo e $I \leq R$ un ideal, diremos que $I$ es un \textbf{ideal maximal} si:
    $$
        I \subseteq J \leq R \implies J = I \text { ó } J = R
    $$
\end{dfn}
\begin{thm}[Cociente de ideales primos y maximales]\label{thm:cocienteideales}
    Sea $R$ un anillo, $I \leq R$ un ideal:\\
    \begin{enumerate}
        \item $I$ es primo $\iff \quo{R}{I}$ es un dominio de integridad.
        \item $I$ es maximal $\iff \quo{R}{I}$ es un cuerpo.
        \item $I$ ideal maximal $\implies$ $I$ ideal primo.
    \end{enumerate}
\end{thm}
\begin{proof}$ $
    \begin{enumerate}
        \item Se deja como ejercicio. Es directa usando definiciones.
        \item $I$ es maximal $\iff \quo{R}{I}$ no tiene ideales propios (por el teorema de correspondencia \ref{pro:correspondencia}). Y ya sabemos que $\quo{R}{I}$ no tiene ideales propios $\iff \quo{R}{I}$ es un cuerpo.
        \item Se sique de los apartados anteriores junto a la proposición $\ref{pro:cuerpoDI}$ que nos dice que un cuerpo es un dominio de integridad.
    \end{enumerate}
\end{proof}

\section{Homomorfismos} \label{section:homomorfismos}

\begin{dfn}[Homomorfismo de anillos]
    Sean $R, S$ anillos, $\varphi: R \to S$ es un \textbf{homomorfismo de anillos} si:\\
    \begin{enumerate}
        \item $\varphi$ es homomorfismo de grupos, es decir, $\varphi(0) = 0$ y $\varphi(a - b) = \varphi(a) - \varphi(b)$.
        \item $\varphi(1) = 1$.
        \item $\varphi(ab) = \varphi(a)\varphi(b)$.
    \end{enumerate}
\end{dfn}

\begin{obs}$ $
    \begin{itemize}
        \item $\ker{\varphi} = \setdef{a \in R \mid \varphi(a) = 0} \leq R$.
        \item $\varphi(R) \subseteq S $ es un subanillo. (No es ideal en general).
        \item $\varphi$ sobreyectiva, es decir, $\varphi$ es un epimorfismo $\iff \varphi(R) = S$.
    \end{itemize}
\end{obs}

\begin{obs}
    Si $R$ y $S$ son cuerpos y $\varphi: R \to S$ es un homomorfismo de anillos, llamaremos a $\varphi$ homomorfismo de cuerpos. Además $\varphi$ es inyectivo pues:
    $$
        1 \notin \ker{\varphi} \leq R \text{ cuerpo} \implies \ker{\varphi} = 0
    $$
\end{obs}

\begin{eg}[Proyección canónica]
    Sea $R$ un anillo, $I \leq R$ un ideal, es fácil ver que $\pi: R \to \quo{R}{I};\ r \mapsto r + I$ es un epimorfismo de anillos con $\ker{\pi} = I$.
\end{eg}

\begin{obs}
    $$\quo{R}{\ker{\varphi}} = \quo{R}{I}$$
\end{obs}

\begin{thm}[Primer teorema de isomorfía]\label{thm:1erisomorfia}
    Sea $\varphi: R \to S$ un homomorfismo de anillos, se tiene que:

    \begin{align*}
        \overline{\varphi}: \quo{R}{\ker \varphi} &\longrightarrow \varphi(S)\\
        r + \ker \varphi &\longmapsto \overline{\varphi}(r + \ker \varphi) = \varphi(r)
    \end{align*}
    es un isomorfismo de anillos.
\end{thm}

\begin{proof}
    Se deja como ejercicio.
\end{proof}

\begin{obs}
    Sea $\pi$ la proyección canónica, $\overline{\pi} = id_{\quo{R}{I}}$
\end{obs}


\begin{ex}[H1.14]
    Demuestra que si $\varphi: R \to S$ es un homomorfismo de anillos y $a \in \uds{R}$, entonces $\varphi(a) \in \uds(S)$. ¿Es cierto el recíproco?.\\\\
    Si $a \in \uds{R}$, tntonces $\exists b \in R$ tal que $\1 = a\cdot b$. Por tanto:
    $$
        \1 = \varphi(\1) = \varphi(a \cdot b) = \varphi(a) \cdot \varphi(b) \implies \varphi(a) \in \uds(S)
    $$
    El recíproco solo es cierto si $\varphi$ es un isomorfismo, pero en general no. Como contraejemplo consideramos el homomorfismo identidad $\iota : \Z \to \Q;\ a \mapsto a$. Es fácil ver que es un homomorfismo de anillos, sin embargo: $\iota(2) = (2)$ pero $\iota(2) \in \uds{\Q}$ y $2 \notin \uds{\Z}$.
\end{ex}

\begin{ex}[H1.16]
    Demuestra que:
    \begin{enumerate}
        \item No existe ningún homomorfismo de anillos $\varphi: \Q \to \Z_p$ para $p \in \Z$ primo.
        \item No existe ningún homomorfismo de anillos $\varphi: \R \to \Q$.
    \end{enumerate}
    Solución:
    \begin{enumerate}
        \item Sea $\varphi: \Q \to \Z_p;\ \1 \mapsto \1 + p\Z$.
        $$
            \varphi(p) = \varphi\left(\sum_1^p 1 \right) = \sum_1^p (\1 + p\Z) = p + p\Z = 0.
        $$
        y como $p \in \uds{\Q}$, es imposible que la imagen de una unidad no sea otra por medio de un homomorfismo, por tanto, dicho homomorfismo no existe.
        \item Sea $\varphi: \R \to \Q;\ \sqrt{2} \mapsto a$
        $$
            2 = \varphi(1 + 1) = \varphi(2) = \varphi(\sqrt{2}^2) = \varphi(\sqrt{2})^2 = a^2,\ a \in \Q
        $$
        que es una contradicción pues no existe dicho $a$, con lo que no existe el homomorfismo.
    \end{enumerate}
\end{ex}

\begin{ex}[H1.21]
    Fijado un entero $n \in \Z$ con $n \geq 2$, demuestra que el anillo cociente $\quo{\Z[x]}{n\Z[x]}$ es isomorfo a $\Z_n[x]$. Conclute que el ideal $n\Z[x]$ es primo si y sólo si $n$ es un número primo.\\\\
    Vamos a dar una guía de como proceder con el ejercicio:
    $$
        \text{Sea } \varphi: \Z[x] \to \Z_n[x];\ (a_0 + \ldots + a_n x^n) \mapsto (\overline{a_0} + \ldots + \overline{a_n} x^n)
    $$
    donde $\overline{a_i} = a_i + n\Z \in \quo{\Z}{n\Z}$.
    \begin{itemize}
        \item Comprobar que $\varphi$ es un homomorfismo de anillos.
        \item Comprobar que $\varphi$ es sobreyectiva.
        \item Ver que $\ker \varphi = n\Z[x]$.
        \item Aplicar el teorema de isomorfía.
    \end{itemize}
\end{ex}

\begin{eg}[Homomorfismo de evaluación]
    Sea $R$ un anillo, $a \in R$.
    \begin{align*}
        \mathcal{E}_a : R[x] &\longleftarrow R\\
                     f(x) &\longmapsto f(a)
    \end{align*}
    es un homomorfismo de anillos sobreyectivo.
\end{eg}

\begin{obs}
    Si $R = K$ es un cuerpo:
    $$
        \quo{K[x]}{\ker \mathcal{E}_a} \isom K \implies \ker\mathcal{E}_a \text{ es maximal.}
    $$
\end{obs}

\section{Anillos de polinomios}

\begin{pro}[Algoritmo de la división]
    Sea $R$ un anillo, $f, g \in R[x]^{\times}$ polinomios con coeficientes en $R$. Si el coeficiente director de $g$ es una unidad de $R$, entonces $\exists d,r \in R[x]$ únicos tales que:
    $$
        f = g \cdot d + r \text{ con } \delta(r) < \delta(g)
    $$
    Diremos que $g \divides f$ si $r = \0$.
\end{pro}
\begin{dfn}[Raíz de un polinomio]
    Sea $R$ un anillo, $f \in R[x]^{\times}$ un polinomio, decimos que $a \in R$ es una \textbf{raíz} de $f$ si $\mathcal{E}_a(f) = f(a) = \0$
\end{dfn}
\begin{cor}[Ruffini]\label{cor:ruffini}
    Sea $R$ un anillo, $f \in R[x]^{\times}$ un polinomio:
    $$
        a \text{ es raíz de } f \iff f(x) = (x-a)\cdot g(x)
    $$
\end{cor}
\begin{proof}$ $
    \begin{itemize}
        \item $\left(\  \Longleftarrow\ \right)$
        $$
            \mathcal{E}_a(f) = \mathcal{E}_a(x-a) \cdot \mathcal{E}_a(g) = \0
        $$
        \item $(\implies)$
        $$
            f(x) = (x - a)\cdot d(x) + r(x);\ \delta(r) \leq \delta(x-a) \implies r \in R;\ f(a) = r = 0 \implies g(x) = d(x)
        $$
    \end{itemize}
\end{proof}

\begin{eg}[Uso de Ruffini]
    Sea $f(x) = x^2 + x + 1$, $f(x) \in \Z_3[x]$.\\ Es fácil ver que $f(1) = 0$, según Ruffini (corolario \ref{cor:ruffini}) $(x-1) \divides f$. Y es cierto, de hecho: $f(x) = (x - 1)(x - 1)$.
\end{eg}

\begin{thm}[Raíces y dominio de integridad]
    Sea $R$ un dominio de integridad, $f \in R[x]^{\times}$ un polinomio y $\alpha_1, \ldots, \alpha_n$ raíces distintas de $f$, entonces $n \leq \delta(f)$.
\end{thm}

\begin{proof}
    Vamos a probarlo por inducción sobre $\delta(f)$
    \begin{itemize}
        \item Caso base: $\delta(f) = 1$. Entonces $f(x) = ax + b$. Sea $\alpha_1$ raíz de $f(x)$, entonces $a \alpha_1 = - b$. Si $\alpha_2$ es raíz de $f$, entonces $a \alpha_2 = -b \implies a \alpha_1 = a \alpha_2 \implies a (\alpha_1 - \alpha_2) = 0$ y como $a \neq 0$ y $R$ es dominio de integridad $\alpha_1 - \alpha_2 = 0 \implies \alpha_1 = \alpha_2$

        \item $\delta(f) = m > 1$. Sea $\alpha_1$ raíz de $f$, por Ruffini $f(x) = (x - \alpha_1) d(x)$. Por hipótesis, $\alpha_2 \ldots \alpha_n$ son raíces de $f$ distintas de $\alpha_1$, por lo que necesariamente $d(\alpha_i) = 0\ \forall i \in [2, n]$. Por la hipótesis de inducción $n - 1 \leq \delta(d) = \delta(f) - 1 \implies n \leq \delta(f)$
    \end{itemize}
\end{proof}

\begin{obs}
    La hipótesis de que $R$ sea un dominio integridad es necesaria. Se puede comprobar que en $\Z_8[x]$, el polinomio $f(x) = x^2 - 1$ con $\delta(f) = 2$, tiene $4$ raíces: $\overline 1, \overline 3, \overline 5, \overline 7$. Sin embargo, no supondrá un problema a lo largo del curso ya que trabajaremos con cuerpos.
\end{obs}

\begin{ex}[H1.27]
    Demuestra que si $K$ es un cuerpo finito y $f, g \in K[x]$ tales que $f(a) = g(a)$ para todo $a \in K$, entonces $f = g$. ¿Qué ocurre si $K$ es finito?.\\\\

    Supongamos $h = f - g \in K[x]$. Entonces $ h(a) = 0\ \forall a \in K$ con $K[x]$ un cuerpo infinito implica necesariamente que $h = 0$ y por tanto $f = g$.\\
    Si $K$ es finito, por ejemplo $K = \Z_p = \{\overline 0, \overline 1, \ldots, \overline{p-1}\}$ con $p$ un primo, consideramos el polinomio $f(x) = x^p - x$. En este caso $f(x) \neq 0$ pero se anula en todo elemento de $\Z_p$ ya que $a^p \equiv a \mod p$ por el pequeño teorema de Fermat.
\end{ex}

\begin{thm}[Pequeño teorema de Fermat]
    Si $p$ es un número primo, entonces, para cada número natural $a$, con $a > 0$, $a^p \equiv a \mod p$.
\end{thm}

\begin{thm}[Ideales principales]
    Sea $K$ un cuerpo, $I \leq K[x]$ un ideal tal que $I \neq \setdef{0}$, entonces existe un $p \in K[x]$ tal que $I = \gen{p}$.
\end{thm}

\begin{proof}
    Sea $p \in I$ con el menor grado finito posible, es decir, sea $0 \neq g \in I \implies \delta(g) \geq \delta(p)\ \forall g \in I$. Entonces por el algoritmo de la division, para $f \in I$, $f = pd + r$, con $d, r \in K[x]$ y $\delta(r) \leq \delta(p) \implies r \in I$. Por la elección de $p$, la única opción es que $ r = 0 $ entonces $f = pd \implies f \in \gen{p}$.
\end{proof}

\begin{ex}[H1.25]
    Hallar un generador de $I = \gen{x^3 + 1, x^2 + 1}$ en $\Z_2[x]$.\\\\

    Basta observar que en $\Z_2[x]$, $(x^2 + 1) = (x + 1)^2$ y $(x^3 + 1) = (x + 1) \cdot (x^2 + x + 1)$, por tanto, $I = \gen{x + 1}$.
\end{ex}

\begin{dfn}[Dominio de ideales principales]
    Un anillo $R$ en el que todo ideal es principal y es un dominio de integridad se llama \textbf{dominio de ideales principales} (o DIP para abreviar).
\end{dfn}

\begin{dfn}[Elemento irreducible]
    Sea $R$ un anillo y $a \neq 0 \in \uds{R}$, decimos que $a$ es \textbf{irreducible} si $a = b \cdot c \implies$ tiene que ocurrir que $b\in \uds{R}$ o que $c \in \uds{R}$
\end{dfn}

\begin{thm}[Irreducibilidad en DIP]
    Sea $R$ un dominio de ideales principales, entonces:
    $$
        a \in R \text{ irreducible } \iff \gen{a} \text{ es maximal.}
    $$
\end{thm}

\begin{proof}$ $
    \begin{itemize}
        \item [$\implies$] Sea $\gen{a} \subseteq J \leq R$, con $J = \gen{b}$. Si $J \neq R$ entonces $b \notin \uds{R}$. Falta ver que $\gen{a} = \gen{b}$. Como $\gen{a} \subseteq \gen{b}$, $a = b \cdot c$ con $c \in R$. Además como $b \notin \uds{R}$ y $a$ es irreducible, entonces $c \in \uds{R}$ y con ello $\gen{a} = \gen{b c} = \gen{b}$.

        \item[$\ \Longleftarrow \ $] Sabemos que $\gen{a} \leq R$ es maximal. Sea $a = bc$ con $b, c \in R$, entonces:
        $$
            \gen{a} \subseteq \gen{b} \leq R \implies \text{ o bien } (\gen{a} = \gen{b} \implies c \in \uds{R}) \text{ o bien } (\gen{b} = R \implies b \in \uds{R})
        $$
        y por tanto $a$ es irreducible.
    \end{itemize}
\end{proof}

\begin{cor}\label{cor:irreduciblecociente}
    Sea $0 \neq f \in K[x]$, con $K$ un cuerpo.
    $$
        f \text{ es irreducible } \iff \quo{K[x]}{\gen{f}} \text{ es un cuerpo}.
    $$
\end{cor}

\begin{proof}
    La prueba es directa sabiendo que $K[x]$ es un DIP, el teorema anterior y el teorema \ref{thm:cocienteideales}.
\end{proof}

\begin{obs}
    $f \in K[x]$ es irreducible si $\delta(f) > 1$ y $f \neq gh$ con $g, h \in K[x]$, $\delta(g) < \delta(f)$ y $\delta(h) < \delta(f)$.
\end{obs}

\begin{obs}
    En $K[x]$ los polinomios de grado $1$ son irreducibles por definición. Los de grado $2$ y grado $3$ son irreducibles $\iff$ no tienen raíces en $K$ (por Ruffini).
\end{obs}

\begin{cor}[Euclides]
    Sea $0 \neq f \in K[x]$ irreducible, si $f \divides gh$ entonces $f\divides g$ o $f \divides h$.
\end{cor}

\begin{proof}
    $$
        f \text{ irreducible} \implies \gen{f} \text{ es maximal} \implies \gen{f} \text{ es primo}.
    $$
    Por definición de ideal primo:
    $$
        f \divides gh \iff gh \in \gen{f} \iff g \in \gen{f} \lor h \in \gen{f}
    $$
\end{proof}

\begin{eg}
    Este corolario nos permite construir cuerpos finitos distintos a los $\F_p$.\\
    $E = \quo{\F_2[x]}{\gen{x^2+x+1}}$ es un cuerpo. Veamos su caracterización.
    \begin{itemize}
        \item Primero comprobamos que $f(x) = x^2 + x + 1 \in \F_2[x]$ es irreducible. Como es un polinomio de grado $2$, es irreducible si no tiene raíces en $\F_2$, y es cierto ya que $f(0) = f(1) = 1$.
        \item Los elementos de $E$ son de la forma: $g + \gen{f}$, y además $g + \gen{f} \neq 0$ en $E \iff g \notin \gen{f}$.\\
        $g = fq + r$, $\delta(r) < \delta(f) = 2$ y como $g$ no es múltiplo, $0 \leq \delta(r) \leq 2$.\\
        $g + \gen{f} = r + fq + \gen{f} = r + \gen{f}$. Por tanto, todo elemento en $E$ tiene un representante con grado  menor a $2$.
    \end{itemize}
    Y por tanto:
    $$
        E = \setdef{a + bx + \gen{f} \mid a, b \in \F_2} \implies E = \setdef{0, 1, x, x+1}
    $$
\end{eg}

\begin{thm}[Máximo común divisor]
    Sean $K$ cuerpo, $0 \neq f, g \in K[x]$ polinomios, existe un único polinomio mónico $d \in K[x]$ tal que:
    $$
        \gen{f} + \gen{g} = \gen{d} \text{ es decir, } \exists a,b \in K[x] :\ d = af + bg
    $$
    Además, $d \divides f$ y $d \divides g$ y si $\exists :\ e \divides f$ y $e \divides g \implies e \divides d$ en $K[x]$.
    Denotamos al polinomio $d$ por $mcd_K(f, g)$.
\end{thm}
\begin{proof}
    Se deja como ejercicio.
\end{proof}

\begin{pro}[Máximo común divisor en subcuerpos]
    Sean $E, K \subseteq E$ cuerpos, y $0 \neq f, g \in K[x]$ polinomios.
    $$
        mcd_K(f, g) = mcd_E(f, g)
    $$
\end{pro}

\begin{proof}
    Sea $d = mcd_K(f, g)$, $e = mcd_E(f, g)$. Entonces, $d = af + bg$ en $K[x] \subseteq E[x] \implies e\divides d $ en $E[x]$. Como $d\divides f$ y $d \divides g$ en $K[x]$ (y en particular también en $E[x]$), entonces $d \divides e$ en $E[x]$. Por tanto:
    $$
        (d \divides e) \land( e \divides d) \land \text{ e, d mónicos} \implies e=d \in K[x]
    $$
\end{proof}

\begin{cor}
    Sea $0 \neq f, g \in K[x]$ con $f$ irreducible.\\
    \begin{enumerate}
        \item $mcd(f, g) = 1$ o $f \divides g$.
        \item Si tenemos $K \subseteq E$, y $f, g$ tienen una raíz común en $E$, entonces $f \divides g$ en $K[x]$.
    \end{enumerate}
\end{cor}

\begin{proof}$ $
    \begin{enumerate}
        \item Si $d = mcd(f, g) \neq 1 \implies \delta(d) > 1 \implies f = ad \implies d = a \cdot f,\ a \in K^\times$. Sea $b \in K[x]$, $g = b\cdot d = b a f \implies f \divides g$.
        \item Sea $a \in E$ la raíz común, por Ruffini $(x - a) \divides f, g$ en $E[x] \implies mcd_E(f, g) = mcd_K(f, g) > \implies f \divides g$.
    \end{enumerate}
\end{proof}

\begin{cor}[Descripción de $\uds{\quo{K[x]}{\gen{f}}}$]
    Sea $0 \neq f \in K[x]$, $R = \quo{K[x]}{\gen{f}}$. Entonces:
    $$
        \bar g = g + \gen{f} \in \uds{R} \iff mcd(f, g)=1
    $$
    Es decir, $\exists a, b \in K[x]$ tal que $1 = af + bg$ y por tanto $(g + \gen{f})^{-1} = b + \gen{g}$.
\end{cor}

\begin{ex}[H1.24]
    Sea $p \in \Q[x]$ dado por $p(x) = (x^2+1)(x^4+2x+2)$. Escribimos $R = \quo{\Q[x]}{\gen{p}}$ y $\bar f = f + \gen{p}$.
    \begin{enumerate}
        \item Describe los ideales en $R$. ¿Es $R$ un cuerpo?.
        \item Decide justificadamente si $\bar x$ y $\overline{x+1}$ son divisores de cero en $R$.
        \item Decide si $\bar x$ y $\overline{x+1}$ son elementos invertibles en $R$ y, en caso afirmativo, encuentra sus inversos.
    \end{enumerate}

    El primer apartado se resuelve por el teorema de correspondencia.\\
    En el segundo apartado tenemos que ver que $mcd(x, p) = 1 = mcd(x+1, p)$. Con ello vemos que $\bar x$ y $\overline{x+1} \in \uds{R}$ y por tanto no pueden ser divisores de cero.\\
    En el tercer apartado faltaría calcular los inversos con la identidad de Bezout.
\end{ex}

\begin{pro}[Cociente de cuerpo e ideal de polinomio irreducible]
    Sea $K$ un cuerpo, $f \in K[x]$ irreducible, $\quo{K[x]}{\gen{f}}$ es un cuerpo.
\end{pro}
\begin{proof}
    Ver el corolario \ref{cor:irreduciblecociente}.
\end{proof}
\begin{thm}[Factorización única]
    Sea $K[x]$ un dominio de factorización única (DFU), sea $0 \neq f \in K[x]$, de la forma:
    $$
        f(x) = ap_1(x) \cdot \cdots \cdot p_r(x),\ a\in K^\times,\ p_i \text{ irreducibles mónicos no necesariamente distintos}
    $$
    entonces la expresión es única (salvo el orden de los factores).
\end{thm}
\begin{proof}
    Se deja como ejercicio.
\end{proof}

\section{Criterios de irreducibilidad}

\begin{lm}[de Gauss]
    Sea $f(x) \in \Z[x]$ un polinomio con $\delta(f) \geq 2$, entonces:
    $$
        f(x) \text{ irreducible en } \Z[x] \implies f(x) \text{ irreducible en } \Q(x)
    $$
\end{lm}
\begin{proof}
    Se deja como ejercicio.
\end{proof}


\begin{lm}[Reducción módulo $p$]
    Sea $f$ un polinomio entero mónico, y $\varphi_p: \Z[x] \to \Z_p[x]$; $\sum a_jx^d \mapsto \sum \overline{a_j}x^d$. Si existe algún primo $p$ de forma que $\varphi_p(f)$ es irreducible en $\Z_p[x]$, entonces $f$ es irreducible en $\Z[x]$.
\end{lm}
\begin{proof}
    Se deja como ejercicio.
\end{proof}

\begin{ex}[H1.34 (c)]
    Demuestra que $f(x) = x^3+x+1$ es irreducible en $\Q[x]$.\\\\
    Usamos reducción módulo $p$ con $p = 2$.
    $$
        f(0) = 1,\ f(1) = 1
    $$
    Como es un polinomio de grado $3$ sin raíces, es irreducible en $\Z_2[x]$ y por tanto es reducible en $\Q[x]$.
\end{ex}

\begin{thm}[Criterio de Einsestein]\label{thm:Einsestein}
    Sea $f(x) = a_0 + a_1 x + \ldots + a_n x^n \in \Z[x]$. Si existe un primo $p$ tal que:
    \begin{enumerate}
        \item $p \not \divides a_n$.
        \item $p^2 \not \divides a_0$.
        \item $p \divides a_i,\ \forall i \in \setdef{0, \ldots, n-1} $.
    \end{enumerate}
    Entonces $f$ es irreducible en $\Q[x]$.
\end{thm}
\begin{proof}
    Se deja como ejercicio.
\end{proof}

\begin{pro}[Raíces racionales de un polinomio]\label{pro:raicesracionales}
    Sea $f(x) = a_0 + a_1 x + \ldots + a_n x^n \in \Z[x]$. Si $\frac{r}{s} \in \Q$ con $mcd(r, s)=1$ es una raíz de $f$, entonces: $s \divides a_n$ y $r \divides a_0$. En particular, si $f \in \Z[x]$ es mónico, las raíces racionales están contenidas en los enteros.
\end{pro}
\begin{proof}
    \begin{align*}
        0 &= f(\frac{r}{s}) = a_0 + a_1 \frac{r}{s} + \ldots + a_n \frac{r^n}{s^n}\\
        0 &= a_0 s^n + a_1 r s^{n-1} + \ldots + a_n r^n \\
        -a_0 s^n &= a_1 r s^{n-1} + \ldots + a_n r^n = r (s^{n-1} a_1 + \ldots + a_n r^{n-1}) \implies r \divides a_0 s^n \implies r \divides a_0\\
        -a_n r^n &= s (a_0 s^{n-1} + \ldots + a_{n-1}r^{n-1}) \implies s \divides a_n r^n \implies s \divides a_n
    \end{align*}
\end{proof}

\begin{eg}[Irreducibilidad cuando fallan otros criterios]
    ¿Es $x^3 + x + 6$ irreducible en $\Q[x]$?.\\\\
    Si intentamos comprobarlo con el criterio de Einsestein o por reducción módulo $p$ no llegamos a nada. Podemos utilizar la proposición \ref{pro:raicesracionales} para hallar que las únicas raíces racionales del polinomio son los divisores de $6$, es decir, $\setdef{\pm 1, \pm 2, \pm 3, \pm 6}$ y si evaluamos el polinomio en los posibles valores ninguno resulta $0$. Por tanto, es un polinomio de grado $3$ sin raíces y entonces es irreducible en $\Q[x]$.
\end{eg}

\begin{lm}[Irreducibilidad evaluando en $x+a$]\label{lema:irreducibilidadeval}
    Sea $f \in K[x]$ ($K$ cuerpo), $a \in K$.
    $$
        f(x) \text{ irreducible} \iff f(x+a) \text{ irreducible}
    $$
\end{lm}
\begin{proof}
    La demostración se sigue de demostrar que $\varphi_a: K[x] \to K[x]$; $f(x) \mapsto f(x+a)$ es un isomorfismo de anillos (cuerpos).
\end{proof}

\begin{thm}[Irreducibilidad de polinomios ciclotómicos]
    Sea $p$ primo, $\Phi_p(x) = x^{p-1} + \ldots + x + 1$ es irreducible en $\Q[x]$.
\end{thm}
\begin{proof}
    Partimos de $(x-1)\Phi_p(x) = x^p - 1$. Aplicamos el lema \ref{lema:irreducibilidadeval} con $a = 1$. Tenemos por tanto:
    $x \Phi(x+1) = (x+1)^p - 1$. Desarrollando con el binomio de newton llegamos a la expresión:
    $$
        \Phi(x+1) = x^{p-1} + \binom{p}{1} x^{p-2} + \ldots + \binom{p}{p-1}
    $$
    Ahora aplicamos Einsestein para el primo $p$, donde vemos que $p \divides \binom{p}{i}$ y $p^2 \not \divides \binom{p}{p-1} = p$, por lo que $\Phi_p(x+1)$ es irreducible y también lo es $\Phi_p(x)$.
\end{proof}

\subsection{Raices múltiples e irreducibilidad}

\begin{dfn}[Raíz múltiple]
    Sea $0 \neq f(x) \in K[x]$ un polinomio, $a \in K$ un raíz de f, existe un $m \in \N > 0$ tal que $f(x) = (x - a)^m g(x)$ con $g(a) \neq 0$ aplicando Ruffini y siendo $K[x]$ un DFU.\\
    Decimos que $a$ es \textbf{raíz múltiple} s i $m > 1$.
\end{dfn}

\begin{dfn}[Derivada formal]
    Sea $f(x) = a_0 + a_1 x + \ldots a_n x^n \in K[x]\setminus K$. Se define $f'(x) = a_1 + 2a_2 x + \ldots n a_n x^{n-1} \in K[x]$ como \textbf{derivada formal}. Si $f \in K$, $f'(x) = 0$.
\end{dfn}

\begin{pro}[Propiedades de derivada formal]
    Sean $f, g \in K[x]$
    \begin{enumerate}
        \item $(f + g)' = f' + g'$; $(af)' = a\cdot f',\ \forall a \in K$.
        \item $(fg)' = f' \cdot g + f \cdot g'$.
        \item Si $f(x) = (x - a)^m,\ m\geq 1$ entonces $f'(x) = m(x - a)^{m-1}$.
    \end{enumerate}
\end{pro}

\begin{pro}[Raíz de derivadas]
    Sean $K \subseteq E$ un subcuerpo, $f(x), f'(x) \in K[x]$ polinomios,  $a \in E$ una raíz múltiple (con multiplicidad $m$) de $f$, entonces:
    $$
        f(a) = f'(a) = 0 \iff m > 1
    $$
\end{pro}
\begin{proof}
    En ambos supuestos: $f(x) = (x - a)^m g(x)$, con $m \geq 1$, $g(a) = 0$ y $f'(x) = m (x - a)^{m-1} g(x) + (x-a)^{m} \cdot g'(x)$.
    \begin{itemize}
        \item[$\implies$] $m > 1 \implies m-1 \geq 1$ y $(f(a) = f'(a) = 0)$.
        \item[$\ \Longleftarrow\ $] $0 = f'(a) = m \cdot 0^{m-1} \cdot g(a) \implies m > 1$. ($g(a) \neq 0$).
        Si tuvieramos $m = 1$ entonces $f'(x) = g(x) + (x-a) g'(x)$, con lo que llegaríamos a $0 = f'(a) = g(a) \neq 0$ lo que es imposible y la desigualdad es necesariamente estricta.
    \end{itemize}
\end{proof}

\begin{thm}[Irreducibilidad y raíces múltiples]
    Sea $K \subseteq E$ un subcuerpo, $f(x) \in K[x]$ un polinomio con $f'(x) \neq 0$.
    \begin{enumerate}
        \item $mcd(f, f') = 1 \implies f$ no tiene raíces múltiples en $E$.
        \item Si $f$ es irreducible, entonces $f$ no tiene raíces múltiples en $E$.
    \end{enumerate}
\end{thm}

\begin{proof}$ $
    \begin{enumerate}
        \item $mcd(f, f') = 1 \implies \exists g, h \in K[x]:\ 1 = fg + hf'$. Por reducción al absurdo, si supones que un cierto $a \in E$ es raíz múltiple de f, entonces $f(a) = f'(a) = 0$ y llegaríamos a $1 = 0$.
        \item Como $f, f' \neq 0$ y $f$ es irreducible, por el lema de Euclides o bien son coprimos o bien $f \divides f'$. Si $f \divides f'$, entonces $\delta(f) < \delta(f')$ pero $\delta(f') = \delta(f) - 1$ y llegamos a una contradicción, por tanto $mcd(f, f')=1$ ya que son coprimos.
    \end{enumerate}
\end{proof}

\begin{ex}[H1.30 (parte)]
    Enumera los polinomios irreducibles en $\F_2$ de grado 1, 2, y 3.\\\\
    \begin{itemize}
        \item[$\delta(f) = 1$] $f(x) = x$, $f(x) = x + 1$.
        \item[$\delta(f) = 2$] $f(x) = x^2 + x + 1$.
        \item[$\delta(f) = 3$] $f(x) = x^3 + x^2 + 1$, $f(x) = x^3 + x + 1$.
    \end{itemize}
\end{ex}


\begin{ex}[H1.35]
    Discute la irreducibilidad de $f(x) = x^5 + 11 x^2 + 15$ en $\Q[x]$.\\\\

    Vamos a ver que es irreducible por medio de reducción módulo $p$ con $p = 2$. $\varphi_2(f) = f_2(x) = x^5 + x^2 + 1$.
    \begin{enumerate}
        \item Vemos que no tiene raíces: $f_2(0) = 1$, $f_2(1) = 1$. Por tanto, no existe una forma de factorizarlo en un producto de dos polinomios de grado $1$ y grado $4$.
        \item Faltaría ver que no se puede factorizar en un producto de polinomios de grado $2$ y grado $3$. Si fuera posible:
        $f_2(x) = g(x) \cdot h(x)$. Además, $g$ y $h$ han de ser irreducibles ya que no existe un polinomio de grado $1$ en sus factores. Con el ejercicio anterior, basta ver que $f_2(x)$ no es el resultado de multiplicar los posibles polinomios irreducibles de grados $2$ y $3$.
        $$
            (x^2 + x + 1)(x^3 + x^2 + 1) \neq f_2(x) \neq (x^2 + x + 1)(x^3 + x + 1)
        $$
    \end{enumerate}
\end{ex}

\begin{ex}[H1.39]
    Factoriza $x^4 -1$ como producto de irreducibles mónicos en: $\Q[x]$, $\R[x]$, $\C[x]$, $\F_2[x]$ y $\F_3[x]$.\\\\
    \begin{itemize}
        \item $\R[x]$ y $\Q[x]$: $x^4 - 1 = (x - 1)(x + 1)(x^2 + 1)$.
        \item $\C[x]$: $x^4 - 1 = (x - 1)(x + 1)(x - i)(x + i)$.
        \item $\F_2[x]$: Como $f'(x) = 0$, $1$ es una raíz con multiplicidad $4$, y por tanto $x^4 - 1 = (x-1)^4$.
        \item $\F_3[x]$: $f'(x) = 4x = x \in \F_3[x] \implies $ las raíces son simples. Se pueden comprobar a mano y obtenemos $x^4 - 1 = (x-1)(x-2)(x^2+1)$.
    \end{itemize}
\end{ex}

\section{Cuerpos}
\begin{dfn}[Cuerpo primo]
    Sea $K$ un cuerpo, $\mathcal{A} = \setdef{L \subseteq K \text{ subcuerpos}}$. Sea $F = \bigcap_{L \in \mathcal{A}} L$, es un subcuerpo de $K$ (se puede comprobar). Llamamos a $F$ el \textbf{cuerpo primo} de $K$, y tiene la característica de ser el menor subcuerpo contenido en $K$, es decir, si $E \subseteq K$ y $E \subseteq F$, entonces $E = F$.
\end{dfn}

\begin{thm}[Isomorfías del cuerpo primo]
    Sea $K$ un cuerpo y $F$ su cuerpo primo, entonces $F$ es isomorfo a:
    \begin{itemize}
        \item $\Q \iff \sum_1^n \1 \neq 0 \forall n \in \Z^\times$
        \item $\F_p \iff \sum_1^n \1 = \0 \iff p \divides n$ para algún primo $p$.
    \end{itemize}
\end{thm}

\begin{obs}
    Vamos a abreviar $\sum_1^n \1$ por $n \1$, donde $\1 \in F$ y $n \in \Z^\times$.
\end{obs}

\begin{proof}
    Consideramos el homomorfismo $\alpha: \Z \to F$; $n \mapsto n \1$. Si $I = \ker(\alpha) = \setdef{0} \iff n \1 \neq \0\ \forall n \in \Z^\times$, $\alpha$ se puede extender a  $\tilde\alpha: \Q \to F$; $\frac{n}{m} \mapsto (n \1) \cdot (m \1)^{-1}$.\\

    Tenemos que comprobar que $\tilde\alpha$ está bien definida. Partimos de $\frac{n}{m} = \frac{a}{b} \in \Q$:
    \begin{align*}
        nb = ma &\implies \alpha(nb) = \alpha(ma) \implies \left(n \1\right)\left(b \1\right) = \left(m \1\right)\left(a \1\right) \implies (*)\\
        (*) &\implies \left(n \1\right)\left(m \1\right)^{-1} = \left(a \1\right)\left(b \1\right)^{-1} \implies \tilde\alpha\left(\frac{n}{m}\right) = \tilde\alpha\left(\frac{a}{b}\right)
    \end{align*}

    Concluimos con que $\tilde\alpha$ está bien definida y es un homomorfismo de grupos inyectivo. Por el primer teorema de isomorfía (teorema \ref{thm:1erisomorfia}):
    $$
        \tilde\alpha(\Q) \subseteq F \subseteq K \text{ donde además } \tilde\alpha(\Q) \isom \Q
    $$
    y por la definición de cuerpo primo: $\Q \isom \tilde\alpha(\Q) = F$.\\\\

    Consideremos ahora el caso en que $I = \ker(\alpha) \neq \setdef{0} \iff (\alpha(n) = 0 \iff p \divides n)$. Entonces, existe $p$ primo tal que $I = \gen{p}$.\\
    Como $\Z$ es un dominio de ideales principales, e $I \leq \Z$, existe $0 \neq m \in \Z$ que cumple $I = \gen{m} = m\Z$. Supongamos $m = a \cdot b$, entonces $0 = \alpha(m) = \alpha(a) \alpha(b) \implies \alpha(a) = 0 \text{ ó } \alpha(b) = 0 \implies a \text{ ó } b \in \gen{m} \implies m\divides a \text{ ó } m \divides b \implies m = p $ primo.\\

    Entonces, de nuevo por el primer teorema de isomorfía:
    $$
        \quo{\Z}{\ker(\alpha)} = \quo{\Z}{p\Z} = \F_p \isom \alpha(\Z) \subseteq F \subseteq K
    $$
    y por la definición de cuerpo primo: $\F_p \isom \alpha(\Z) = F$
\end{proof}

\begin{dfn}[Característica de un cuerpo]
    Sea $K$ un cuerpo y $F$ su cuerpo primo, decimos que su \textbf{característica} $\car{K}$, es $\car{K} = 0$ si $F \isom \Q$ y $\car{K} = p$ si $F \isom \F_p$.
\end{dfn}

\begin{ex}[H1.40]
    Sean $K$ y $E$ dos cuerpos de distinta característica, demuestra que no existe $\varphi: K \to E$ tal que $\varphi$ sea un homomorfismo de cuerpos.\\\\

    Supongamos $\car{E}=0$ y $\car{K} = p > 0$. Entonces:
    $$
        \0 = \varphi(p \1) = p \varphi(\1) = p \1 \neq 0 \text{(en E)}
    $$
    Faltaría ver el caso en que $\car{E} = p \neq q = \car{K}$ con $p, q$ primos:
    $$
        \0 = \varphi(q \1) = q \varphi(\1) = q \1 \neq 0 \text{(en E)}
    $$
    Con lo que llegamos a una contradicción en ambos casos, y no existe dicho homomorfismo.
\end{ex}

\begin{obs}
    En un cuerpo de característica $p$, $(a \pm b)^p = a^p \pm b^p$.
\end{obs}

\begin{dfn}[Cuerpo perfecto]
    Sea $K$ un cuerpo de característica $p$, y el monomorfismo $Frob: K \to K$; $a \mapsto a^p$. Decimos que $K$ es \textbf{perfecto} si $Frob$ es sobreyectivo, es decir, $Frob$ es un isomorfismo de cuerpos.
\end{dfn}

\begin{pro}[Endomorfismo y cuerpo primo]
    Sea $K$ un cuerpo, $F$ su cuerpo primo y $\sigma: K \to K$ un endomorfismo de cuerpos, entonces;
    $$
        \sigma(a) = a,\ \forall a \in F
    $$
\end{pro}
\begin{proof}
    Se deja como ejercicio.
\end{proof}

\begin{ex}[H1.42 (parte)]
    Si $n > 0$ no es un cuadrado, demuestra que:
    \begin{enumerate}
        \item $\F_3[\xi] = \setdef{a + b\xi \mid a, b \in \F_3,\ \xi^2 = -1}$ es un cuerpo.
        \item No existe un homomorfismo de anillos $\varphi: \Q[i] \to \Q[\sqrt{2}]$.
    \end{enumerate}

    $ $\\\\
    \begin{enumerate}
        \item Sea el polinomio $f(x) = x^2 + 1$, de forma que $f(\xi) = 0$. Otra forma de describir $\F_3[\xi]$ es:
        $$
            \F_3[\xi] = \setdef{a + b\xi \mid a, b \in \F_3,\ f(\xi) = 0} \isom \quo{\F_3[x]}{\gen{f(x)}}
        $$
        por lo que es un cuerpo ya que $x^2+1$ es irreducible en $\F_3[x]$ al no tener raíces.
        \item El problema surge de la imagen de $i$. $\varphi(i)$ será de la forma: $a + b \sqrt{2}$, y entonces:
        $$
            -1 = \varphi(-1) = \varphi(i^2) = (a + b \sqrt{2})^2 = a^2 + 2b^2 + 2ab\sqrt{2}
        $$
        de donde podría deducirse que $\sqrt{2} \in \Q$ y es imposible. Por tanto no existe dicho homomorfismo.
    \end{enumerate}
\end{ex}
