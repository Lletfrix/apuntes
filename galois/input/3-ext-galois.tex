% !TeX root = ../galois.tex

\chapter{Extensiones de Galois}

\section{Cuerpos de escisión}

\begin{dfn}[Escisión de un polinomio]
    Sea $f \in K[x]$, $K$ cuerpo, decimos que $f$ se \textbf{escinde} si $\exists a, a_i\in K$ si:
    $$
        f(x) = a(x - a_1) \ldots (x - a_n)
    $$
\end{dfn}

\begin{obs}
    Si $f(x) \in K[x]$, $\delta f = 1$, entonces $f$ se escinde en $K$.
    $$
        f(x) = ax + b = a \left(x - \left(- \frac{b}{a}\right)\right),\ \ \ a, -\frac{b}{a} \in K
    $$
\end{obs}

\begin{obs}
    Si $f \in K[x]$, se escinde en $E$, sea $E/K$ una extensión:
    $$
        f(x) = a (x - a_i) \ldots (x - a_n) \in E[x]
    $$
    Entonces las raíces de $f$ en cualquier extensión de $E$ son $a_1, \ldots, a_n$
\end{obs}

\begin{dfn}[Cuerpo de escisión]
Sea $f \in K[x]$, entonces $E$ es un cuerpo de escisión de $f$ sobre $K$ si:
    \begin{itemize}
        \item $E/K$ es una extensión.
        \item $f$ se escinde en $E$.
        \item Si $f$ se escinde en $L$, con $L$ un cuerpo intermedio $K \subseteq L \subseteq E$, entonces $L=E$, es decir, $E = K(f)$. (Es el menor subcuerpo con la propiedad de que $f$ se escinda en él).
    \end{itemize}
\end{dfn}

\begin{eg}
    $x^2 + 1$ se escinde en $\C$ pero su cuerpo de escisión es $\Q(i)$.
\end{eg}

\begin{obs}
    Si $f \in K[x]$ se escinde en $E$, entonces $E$ tiene todas las raíces de $f$. Para construir su cuerpo de escisión basta adjuntar todas las raíces al cuerpo sobre el que está definido:
    $$
        f(x) = a (x - a_1) \ldots (x - a_n) \in E[x], \text{ entonces su cuerpo de escisión es } K(f) = K(a_1, \ldots, a_n)
    $$
\end{obs}

\begin{lm}[Escisión de polinomios no constantes]\label{lm:3.1}
    Sea $K$ un cuerpo:\\
    \begin{enumerate}
        \item[($i$)] Si $f \in K[x]$ no constante se escinde en $K$ si y sólo si todos los polinomios en la descomposición en términos de factores irreducibles tienen grado $1$.
        \item[($ii$)] Si $f \in K[x]$ no constante que se escinde en $K$ y sea $p \in K[x]$ tal que $p \divides f$, entonces $p$ se escinde en $K$.
    \end{enumerate}
\end{lm}

\begin{proof}
    La demostración se deja como ejercicio.
\end{proof}

\begin{lm}[Cuerpos de escisión y cuerpos intermedios]\label{lm:3.2}
    Sean $E/L$ y $L/K$ extensiones, y $f \in K[x]$ un polinomio no constante. Si $E$ es el cuerpo de escisión de $f$ sobre $K$, entonces $E$ es el cuerpo de escisión de $f$ sobre $L$.
\end{lm}

\begin{proof}
    Como $E = K(f)$, tenemos que $f(x) = a (x - a_1) \ldots (x - a_n) \in E[x]$ y $E = K(a_1, \ldots, a_n)$. Además, $f \in K[x] \subseteq L[x]$. Si el cuerpo de escisión de $f$ sobre $L$ es $E'$, fácilmente llegamos a que $E' = L(f) \subseteq E$ y por tanto: $E = E'$ ya que $E$ era el menor cuerpo de escisión de $f$ sobre $K$.
\end{proof}

Si $f$ es un polinomio sobre $\Q[x]$ de grado $n$, \textit{sabemos} que $\exists \alpha_i \in \C$ con $i \in \setdef{1, \ldots n}$ tal que $f(x) = a (x - \alpha_1) \ldots (x - \alpha_n)$, y hemos visto para calcular su cuerpo de escisión basta con adjuntar las raíces de $f$, es decir, $\Q(f) = \Q(\alpha_1, \ldots, \alpha_n)$.\\

Vamos a ver resultados para generalizar este proceso.

\begin{lm}[Teorema de Kronecker]\label{lm:3.3}
    Sea $p \in K[x]$ un polinomio irreducible, entonces existe $E/K$ tal que $p$ tiene una raíz en $E$.
\end{lm}

\begin{proof}
    Sea $L = \quo{K[x]}{\gen{p}}$, ya vimos que $L$ es un cuerpo por ser $p$ irreducible. Además se puede comprobar que:
    $$
        \bar x = x + \gen{p} \in L \text{ es raíz de } \bar p \in \bar K[y]
    $$
    Donde $\bar p = \overline{a_0} + \ldots + \overline{a_n}x^n$ y $K \isom \bar K = \setdef{\bar K \mid k \in K}$.

    Hay que comprobar que $\bar K \subseteq L$. Tras ello, habría que ver que $L = \bar K(\bar x)$.
\end{proof}

\begin{thm}[Existencia de cuerpos de escisión]\label{thm:3.4}
    Sea $K$ un cuerpo, $f \in K[x]$ un polinomio no constante. Entonces existe un cuerpo de escisión $E$ de $f$ sobre $K$.
\end{thm}

\begin{proof}
    Basta construir una extensión en la que $f$ se escinda. La demostración sigue por inducción sobre $\delta(f)$.
    \begin{itemize}
        \item[$\delta(f) = 1$] Entonces por el lema \ref{lm:3.1}, $K$ es un cuerpo de escisión de $f$ sobre $K$.
        \item[$\delta(f)>1$] Sea $p \divides f$ un factor irreducible de $f$ en $K[x]$, por el lema \ref{lm:3.3}, sabemos que existe $E/K$ en la que $p$ tiene una raíz $\alpha \in E$, en particular $f(\alpha) = 0$ y por Ruffini:
        $$
            f(x) = (x - \alpha) g(x),\text{ con } g(x) \in K(\alpha)[x]
        $$
        Como $1 \leq \delta(g) \leq \delta(f) - 1 \leq \delta(f)$, por inducción sabemos que existe un cuerpo de escisión $L$ de $g$ sobre $K(\alpha)$. Si $\alpha_1, \ldots \alpha_n \in L$ son las raíces de $g$, entonces:
        $$
            L = K(\alpha)(\alpha_1, \ldots \alpha_n) = K(\alpha, \alpha_1, \ldots, \alpha_n)
        $$
        y por tanto $f(x) = (x - \alpha) g(x) = (x - \alpha) p (x - \alpha_1) \ldots (x - \alpha_n) \in L[x]$, así que $f$ se escinde sobre $L$, y de hecho $L$ es un cuerpo de escisión de $f$ sobre $K$.
    \end{itemize}
\end{proof}

\begin{ex}[Cálculo del cuerpo de escisión]
    Describir el cuerpo de escisión de $f(x) = x^4 - 4x^2 + 2$ sobre $\Q$.\\\\
    Comenzamos hallando las raíces, que resulta fácil pues $f(x) = 0$ es una ecuación bicuadrática. Tenemos entonces las raíces:
    $$
        \alpha = \sqrt{2 + \sqrt 2},\ \beta = \sqrt{2 - \sqrt{2}},\ -\alpha,\ -\beta
    $$
    De donde vemos que $f$ se escinde en $R$ ya que todas las raíces son reales. Para ver su cuerpo de escisión vamos a adjuntar las raíces a $\Q$.
    $$
        E = \Q(\alpha,  \beta, -\alpha, -\beta) = \Q(\alpha, \beta)
    $$
    Sin embargo, podemos reducirlo aún más si vemos que $\Q(\alpha, \beta) = \Q(\alpha)$.
    $$
        \alpha\beta = \sqrt{(2 + \sqrt 2)(2 - \sqrt 2)} = \sqrt 2,\ \alpha^{-1} = \left(\frac{\beta}{\sqrt 2}\right) \in \Q(\alpha) \implies \beta = \frac{\beta}{\sqrt 2} \cdot \sqrt 2 \in \Q(\alpha)
    $$
    Con lo que el cuerpo de escisión es: $\Q(\sqrt{2 + \sqrt 2})$.
\end{ex}

\begin{obs}
    Si $E = K(f)$, entonces $E/K$ es finita por el teorema \ref{thm:extalg}. Anteriormente, vimos que si $f \in K[x]$, entonces existe un cuerpo de escisión de $f$ sobre $K$.\\
    Sea $p$ un polinomio irreducible tal que $p \divides f$ y $\delta p > 1$, entonces consideramos $\quo{K[x]}{\gen{p}}$ y en ese cuerpo hay una raíz de $p$ (que también es raíz de $f$), y después aplicamos inducción.
\end{obs}

\begin{thm}\label{thm:3.5}
    Sea $K_1, K_2$ cuerpos, y $\sigma: K_1 \to K_2$ un isomorfismo de cuerpos. Sea $f_1 \in K_1[x]$ no constante y $\delta(f_1) = f_2$.\\ Supongamos que $E_i$ son cuerpos de escisión de $f_1$ sobre $K_i$ con $i \in \setdef{1, 2}$. Entonces existe un isomorfismo $\tau: E_1 \to E_2$ que extiende a $\sigma$, es decir, $\left. \tau \right|_{K_1} = \sigma$.
\end{thm}

\begin{proof}
    Pedir a Santorum. %% TODO
\end{proof}

\begin{cor}[Unicidad de los cuerpos de escisión]
    Sea $K$ un cuerpo y $f \in K[x]$ no constante. Si $E_1$ y $E_2$ son cuerpos de escisión de $f$ sobre $K$ entonces $\exists \tau$ un isomorfismo $\tau: E_1 \to E_2$ tal que $\tau(a) = a$ $\forall a \in K$.
\end{cor}
\begin{proof}
    Basta tomar $\sigma = id_K$ en el teorema \ref{thm:3.5}.
\end{proof}

%% TODO: COMPLETAR EL DIA 22/10. Santorum.

%%%%%%%%%%%%%%%%%%%%%%%%%%%%%%%%%%%%%%%%%%%%%%%%%%%%%%%%%%%%%%%%%%%%%%%%%% 24/10

\begin{lm}[Normalidad de extensiones en cuerpos intermedios]\label{lm:3.7}
    $K \subseteq L \subseteq E$ extensiones, entonces:
    $$
        E/K \text{ normal } \implies E/L \text{ normal.}
    $$
\end{lm}

\begin{proof}
    Es consecuencia directa del lema \ref{lm:3.2} y de la definición.
\end{proof}


\begin{cor}[al teorema \ref{thm:3.5}]
    Sea $E/K$ una extensión normal, $K \subseteq M_1, M_2 \subseteq E$ con $M_1, M_2$ cuerpos intermedios.\\\\
    Si existe un isomorfismo $\sigma: M_1 \to M_2$ que fija a $K$, es decir, $\sigma(a) = a,\ \forall a \in K$, entonces $\sigma$ se extiende a un isomorfismo $\tau: E \to E$ ($\left. \tau \right|_{M_1} = \sigma$).
\end{cor}
\begin{proof}
    Sea $E = K(f)$ con $f \in K[x]$ no constante. En particular por el lema \ref{lm:3.2}, $E = M_1(f) = M_2(f)$. Como $\sigma(f) = f$, el resultado se sigue del teorema \ref{thm:3.5}.
\end{proof}
\begin{obs}
    La hipótesis $\sigma(a) = a,\ \forall a \in K$ es necesaria. La idea es que sin esta hipótesis:
    $$
        E = M_1(f) = M_2(f)
    $$
    pero no sabríamos si $E = M_2(\sigma(f))$.
\end{obs}

\begin{thm}[Condición necesaria y suficiente para la normalidad de una extensión]\label{thm:3.9}
    Sea $E/K$ una extensión finita, $E/K$ es normal si y sólo si todo polinomio $p \in K[x]$ irreducible con una raíz en $E$ se escinde en $E$.
\end{thm}

\begin{proof}$ $
    \begin{itemize}
        \item[$\ \Longleftarrow\ $] Sabemos que $E/K$ es finita, digamos $|E:K| = n$, podemos tomar una $K$-base $\sdf{a_1, \ldots, a_n}$. En particular $E=K(a_1, \ldots, a_n)$.\\
        Como $E/K$ es finita, por el teorema \ref{thm:2.2}, los $a_i$ en son algebraicos sobre $K$. Sea $p_i = Irr(k, a_i) \in K[x]$, por hipótesis cada $p_i$ se escinde en $E$.\\

        Tomamos $f = \prod_{i=1}^{n} p_i \in K[x]$ se escinde en $E$. De hecho, $E = K(f)$. De esta forma hemos probado que:
        $$
            K(f) \subseteq E = K(a_1, \ldots, a_n) \subseteq K(f)
        $$
        \item[$\implies$] Ahora partimos de que $E/K$ es finita y normal. Entonces $\exists f \in K[x]$ tal que $E = K(f)$.\\
        De hecho, si $f(x) = c(x - b_1) \ldots (x-b_n) \in E[x]$, $E = K(b_1, \ldots, b_n)$.\\
        Sea $p\in K[x]$ irreducible con una raíz $a\in E$, por Ruffini (corolario \ref{cor:ruffini}) $p(x) = (x-a)g(x)\in E[x]$. Además, por el teorema \ref{thm:3.4}, existe $M/E$ un cuerpo de escisión de $p$ sobre $E$, por tanto:
        $$
            p(x) = d(x-d_1) \cdot \ldots \cdot (x-d_r), \text{ podemos suponer que $a = d_1$}
        $$
        Dado $d_i$ con $i > 1$, queremos ver que $d_i \in E$. Llamemos $b = d_i$ para algún $i$. Como $a$ y $b$ son raíces de $p \in K[x]$ irreducible, por el teorema \ref{thm:2.5} $\exists \sigma: K(a) \to K(b)$ tal que $\sigma(k) = k,\ \forall k \in K$.

        %TODO
        %\sigma: K(a) --------> K(b)
        %         ^              ^
        %         |              |
        %         v              v
        %id:      K  --------->  K

        Como $K(a) \subseteq E$, en particular $E = K(a)(f)$. Por otro lado $K(b)(f) = E(b)$. Las raíces de $f$ en $E(b)$ son $b_1, \ldots, b_m$, así que:
        $$
            K(b)(f) = K(b)(b_1, \ldots, b_m) = K(b_1, \ldots, b_m) = E(b)
        $$
        Por el teorema \ref{thm:3.5} existe un isomorfismo $\tau: E \to E(b)$ que extiende a $\sigma$.
        %TODO
        %\tau:  E=K(a)(f) ----> E(b) = K(b)(f)
        %         |              |
        %         |              |
        %\sigma: K(a) --------> K(b)
        %         ^              ^
        %         |              |
        %         v              v
        %id:      K  --------->  K

        Por el ejercicio $H2.15$, $|E:K| = |E(b):K|$ y $|K(a):K| = |K(b):K|$. Además $|E:K(a)| = |E(b):K(b)|$.
        Por tanto:
        $$
            |E:K| = |E:K(a)||K(a):K| = |E(b):K(b)||K(b):K| = |E(b):K| = |E(b):E||E:K| \text{ usando el teorema \ref{thm:transgr}.}
        $$
        Con lo que concluimos:
        $$
            |E(b):E| = 1 \implies E(b) = E \implies b\in E
        $$
    \end{itemize}
\end{proof}


\begin{ex}[H3.5]
    Decide si son normales las siguientes extensiones:
    \begin{enumerate}
        \item $\Q(\sqrt{5}i)/\Q$.\\\\

        Es normal ya que $\Q(\sqrt{5}i) = \Q(x^2+5)$ es el cuerpo de escisión de $x^2+5$.
        \item $\Q(\sqrt{5})/\Q$.\\\\

        Es normal ya que $\Q(\sqrt{5})/\Q = \Q(x^2-5)$ es el cuerpo de escisión de $x^2-5$.

        \item $\Q(\sqrt[4]{5}/\Q)$.\\\\

        No es normal. Sabemos que $\Q(\sqrt[4]{5})/\Q$ es finita, y como $p(x) = x^4 - 5$ es irreducible en $\Q[x]$, tiene una raíz en $\Q(\sqrt[4]{5})$ pero no escinde, por la caracterización de normalidad (teorema \ref{thm:3.9}), concluimos con que $\Q(\sqrt[4]{5})/\Q$ no es normal.
    \end{enumerate}
\end{ex}

\begin{ex}[H3.6]
    Demuestra que $\Q(\sqrt[3]{2})/\Q$ no es normal. Encuentra una extensión normal que contenga $\Q(\sqrt[3]{2})$ como subcuerpo.\\\\

    Sabemos que no es normal porque sea $p(x) = x^3 - 2 \in \Q[x]$, se puede comprobar que $p(x)$ es irreducible y tiene una raíz en $\Q(\sqrt[3]{2})$ pero no se escinde. Por tanto, por el teorema \ref{thm:3.9}, el cuerpo no es normal.\\
    Una extensión normal que contiene a $\Q(\sqrt[3]{2})$ puede ser $\Q(\sqrt[3]{2}, \omega)$ con $\omega \in \C,\ o(\omega) = 3$.
\end{ex}

\begin{obs}
    Si $E/K$ es normal y $f\in K[x]$ no necesariamente irreducible que tiene una raíz en $E$, $f$ no tiene por que escindirse.
\end{obs}

%%%%%%%%%%%%%%%%%%%%%%%%%%%%%%%%%%%%%%%%%%%%%%%%%%%%%%%%%%%%%%%%%%%%%%%%%% 28/10
\section{El grupo de Galois de una extensi\'on}

\subsection{Acción de un grupo}

Vamos a ver un repaso de algunos conceptos de la acción de un grupo sobre un conjunto.

\begin{dfn}[Acción de un grupo]
    Una \textbf{acción} de un grupo $(G, \star)$ sobre un conjunto $X$ es una aplicación: $\phi: G \times X \to X$ que cumple:
    \begin{enumerate}
        \item $ \forall x \in X,\ \phi(e, x) = x$ donde $e$ es el elemento neutro del grupo.
        \item $ \forall x \in X,\ g, h \in G,\ \phi(g \star h, x) = \phi(g, \phi(h \star x))$
    \end{enumerate}
\end{dfn}
\begin{obs}$ $
    \begin{itemize}
        \item El cumplimiento de las dos condiciones hace que la aplicación $\phi(g, \cdot): X \to X$ sea biyectiva para cada $g \in G$. Con esto se puede dar una definición alternativa de la acción de un grupo:
        $$
            \phi: G \to \sdf{\text{funciones biyectivas } X \to X}
        $$
        \item Una notación alternativa usada para las acciones es:
        $$
            (g, x) \mapsto x \cdot g
        $$
        Con lo que las dos condiciones se reescriben como:
        \begin{enumerate}
            \item $x \cdot e = x$, con $e$ el neutro del grupo.
            \item $(x \cdot g) \cdot h = x \cdot (g \cdot h),\ \forall g,h \in G,\ x \in X$
        \end{enumerate}
    \end{itemize}
\end{obs}

\begin{pro}[Una acción es un homomorfismo]
    Sea $(G, \cdot)$ un grupo, $\Omega \neq \varnothing$ un conjunto. Si $G$ actúa sobre $\Omega$ (lo denotamos por $\Omega^{\actar G}$), la aplicación:
    $$
        \rho: G \to S_{\Omega};\ g \mapsto f \circ g: \Omega \to \Omega;\ \omega \mapsto \omega \cdot g
    $$
    es un homomorfismo de grupos, en particular $\ker(\rho) \normsub G$. Y tenemos que:
    $$
        \quo{G}{\ker \rho} \isom \rho(G) \leq S_{\Omega}
    $$
    Recíprocamente, todo homomorfismo $\rho: G \to S_\Omega $ define una acción de $G$ en $S_\Omega$.
\end{pro}

Además, sea $(G, \cdot)$ un grupo, $\Omega \neq \varnothing$ un conjunto. Si $G$ actúa sobre $\Omega$ ($\Omega^{\actar G}$), entonces define una relación binaria de equivalencia (r.b.e) de la forma:
$$
    \omega \sim \omega' \iff \exists g \in G:\ \omega \cdot g = \omega'
$$
Las clases de equivalencia definidas por esta relación se denominan $G$-órbitas.

\begin{dfn}[$G$-órbita de un elemento de un conjunto]
    Sea $G$ un grupo, la \textbf{órbita} de un elemento $\omega$ de un conjunto $\Omega$ es la clase de equivalencia:
    $$
        \omega^G = \theta_\omega = \sdf{\omega \cdot g \mid \forall g \in G} \subseteq \Omega
    $$
\end{dfn}

\begin{dfn}[Estabilizador de un elemento de un conjunto]
    Sea $G$ un grupo, el \textbf{estabilizador} de un elemento $\omega$ de un conjunto $\Omega$ es:
    $$
        G_\omega = \mathrm{St}(\omega) = \sdf{g \in G \mid \omega \cdot g = \omega} \leq G
    $$
    que es un subgrupo de $G$.
\end{dfn}

\begin{thm}[Teorema de la Órbita-Estabilizador]\label{thm:orbita}
    Sea $G$ un grupo que actúa sobre $\Omega$ y $\omega \in \Omega$ un elemento, entonces la longitud de la $G$-órbita de $\omega$ ($\left|\theta_\omega\right| = \vabs{G_\omega}$) es:
    $$
        \vabs{G_\omega} = \vabs{\theta_\omega} = \vabs{G:G_\omega}
    $$
\end{thm}

\begin{dfn}[Puntos fijos por la acción de un grupo]
    Sea $G$ un grupo y $\Omega$ un conjunto.
    \begin{itemize}
        \item Los puntos fijos por la acción de $G$ son los pertenecientes al conjunto:
        $$
            \Omega^G = \sdf{ \omega \in \Omega \mid \omega \cdot g = \omega,\ \forall g \in G}
        $$
        \item Los puntos fijos por la acción de $g \in G$ son los pertenecientes al conjunto:
        $$
            \Omega^g = \sdf{ \omega \in \Omega \mid \omega \cdot g = \omega}
        $$
    \end{itemize}
\end{dfn}

\subsection{Grupo de Galois}

En esta subsección, vamos a considerar $E$ como una extensión. Además, sea $\aut(E)$ el conjunto de automorfismos de $E$, podemos comprobar que $(\aut(E), \circ)$ es un grupo (no abeliano en general).\\

El grupo $(\mathrm{Aut}(E), \circ)$, actúa sobre $E$. ($e \cdot \sigma = \sigma(e) = e$).

\begin{dfn}[Grupo de Galois]
Sea $\sigma \in \aut(E)$, $E^\sigma = \sdf{e \in E \mid e \sigma = e}$ el conjunto de puntos fijos de $E$ por la acción de $\sigma$ (es subcuerpo de $E$). Sea $E/K$ una extensión, llamamos \textbf{grupo de Galois} de $E/K$ a:
$$
    \gal(E/K) = \sdf{\sigma \in \aut(E) \mid \sigma(k) = k \forall k \in K} = \sdf{\sigma \in \aut(E) \mid K \subseteq E^\sigma}
$$
Donde además, el \textit{grupo de Galois} es un cuerpo intermedio de $E/K$.
\end{dfn}

\begin{pro}
Sea $L$ un cuerpo intermedio de $E/K$, entonces: $\gal(E/L) \leq \gal(E/K)$. En general $\left.\sigma\right|_L \notin \gal(L/K)$.
\end{pro}

\begin{obs}[Notación]
    Sea $\sigma \in \gal(E/K)$, decimos que es un $K$-automorfismo de $E$ si fija todo $K$ elemento a elemento.
\end{obs}

\begin{dfn}[$K$-isomorfismo]
    Sea $E/K$ una extensión, $L_1, L_2$ cuerpos intermedios, $\sigma: L_1 \to L_2$ es un $\mbf K$\textbf{-isomorfismo} si $\sigma: L_1 \to L_2$ es un isomorfismo y $\sigma(k) = k,\ \forall k \in K$.
\end{dfn}

Con esta notación vamos a reescribir el teorema \ref{thm:3.8}.
\begin{thm}\label{thm:3.8bis}
    Sea $E/K$ extensión normal,$L_1, L_2$ cuerpos intermedios. Si $\sigma: L_1 \to L_2$ es un $K$-isomorfismo, entonces: $\exists \tau \in \gal(E/K)$ tal que $\left. \tau \right|_{L_1} = \sigma$.
\end{thm}

\begin{cor}[al teorema \ref{thm:3.8bis}] \label{cor:3.10}
    Sea $E/K$ una extensión normal, $p \in K[x]$ un polinomio irreducible. Si $a$ y $b$ son raíces de $p$ en $E$, entonces $\exists \tau \in \gal(E/K):\ \tau(a) = b$.
\end{cor}
\begin{proof}
    Por el corolario \ref{cor:2.6}, existe un $K$-isomorfismo $\sigma: K(a) \to K(b)$. Por el teorema \ref{thm:3.8bis}, $\sigma$ se extiende a $\tau \in \gal(E/K)$.
\end{proof}

\begin{obs}
    $\gal(E/K)$ actúa sobre las raíces de $f \in K[x]$ en $E$.
\end{obs}

\begin{thm}[Grupo de Galois y raíces de un polinomio]\label{thm:3.11}
    Sea $E/K$ una extensión, $0 \neq f \in K[x]$, $\Omega = \sdf{a_1, \ldots, a_n}$ el conjunto de todas las raíces distintas de $f$ en $E$ (no quiere decir que $E$ contenga todas las raíces de $f$), con $n \leq 1$ y sea $L = K(\mbf a) \subseteq E$ un subcuerpo intermedio, entonces:
    \begin{itemize}
        \item[(a)] Si $a \in \Omega$ y $\sigma = \gal(E/K)$, entonces $\sigma(a) \in \Omega$. En particular, $\left.\sigma\right|_\Omega \in S_\Omega$ y $\sigma(L) = L$.
        \item[(b)] La aplicación $\psi: \gal(E/K) \to \gal(L/K);\ \sigma \mapsto \left.\sigma\right|_L$ es un homomorfismo de grupos con $\ker(\psi) = \gal(E/L) \normsub \gal(E/K)$.
        \item[(c)] Si $E/K$ es normal, entonces $\psi$ es sobreyectivo. En particular, $\quo{\gal(E/K)}{\gal(E/L)} \isom \gal(L/K)$.
        \item[(d)] La aplicación $\rho: \gal(E/K) \to S_\Omega;\ \sigma \mapsto \left.\sigma\right|_\Omega$ es un homomorfismo de grupos con $\ker(\rho) = \gal(E/L)$.
        \item[(e)] Si $E/K$ es normal y $\rho$ es irreducible, dados $a, b \in \Omega$ existe $\sigma \in \gal(E/K)$ tal que $\sigma(a) = b$. Por tanto:
        $$
            n = \left|\gal(E/K):\gal(E/K(a))\right|
        $$
    \end{itemize}
\end{thm}

\begin{obs}[al teorema \ref{thm:3.11}]
    $$
        \left(\gal(L/L) = \sdf{id}\right)
    $$
    En el teorema, (d) aplicado a $L/K$ implica que:
    $$
        \gal(L/K) \isom \rho(\gal(L/K)) \leq S_\Omega
    $$
\end{obs}

\begin{proof}$ $
    \begin{itemize}
        \item[(a)]
        $$
            0 = \sigma(0) = \sigma(f(a)) = \sigma(f)(\sigma(a)) = f(\sigma(a)) \implies \sigma(a) \in \Omega
        $$
        Como:
        $$
            \left. \sigma \right|_\Omega: \Omega \to \Omega \text{ es inyectiva,}
        $$
        entonces:
        $$
            \Omega \text{ es finita } \implies \left. \sigma \right|_\Omega \in S_\Omega \text{ es biyectiva,}
        $$
        por tanto:
        $$
            \sigma(L) = \sigma(K(a_1, \ldots, a_n)) = \sigma(K)(\sigma(a_1), \ldots, \sigma(a_n)) = K(a_1, \ldots, a_n) = L
        $$
        \item[(b)] Se deja como ejercicio.
        \item[(c)] Sea $\theta \in \gal(L/K)$, por el teorema \ref{thm:3.8bis}, $\exists \sigma \in \gal(E/K)$ tal que $\left. \sigma \right|_L = \theta$. En particular, $\psi$ es sobreyectiva. (La segunda conclusión se sigue del teorema de isomorfía).
        \item[(d)] Se deja como ejercicio partiendo de $\left. \sigma \right|_\Omega \in S_\Omega$.
        \item[(e)] Por el corolario \ref{cor:3.10} si $a, b \in \Omega$ cualesquiera, $\exists \sigma \in \gal(E/K)$ con $a\sigma = \sigma(a) = b$ y $\theta_a = \Sigma$ por el teorema \ref{thm:orbita}.
        %%TODO: COMPLETAR demostración del cardinal. Santorum.
    \end{itemize}
\end{proof}

%%%%%%%%%%%%%%%%%%%%%%%%%%%%%%%%%%%%%%%%%%%%%%%%%%%%%%%%%%%%%%%%%%%%%%%%%% 29/10

\begin{cor}[Caracterización de normalidad a través de isomorfismos]\label{cor:3.12}
    Sea $E/K$ una extensión, $L$ un cuerpo intermedio:
    \begin{itemize}
        \item[(a)] Si $L/K$ es norma entonces $\tau \in \gal(E/K)$:
        $$
            \tau(L) = L \left(\rest{\tau}_L\right) \in \gal(L/K)
        $$
        \item[(b)] Si $E/K$ es normal:
        $$
            L/K \text{es normal} \iff \forall \tau \in \gal(E/K),\ \tau(L) = L
        $$
        En particular $\gal(E/L) \normsub \gal(E/K)$ y $\quo{\gal(E/K)}{\gal(E/L)} \isom \gal(L/K)$.
    \end{itemize}
\end{cor}

\begin{proof}$ $
    \begin{itemize}
        \item[(a)] $L = K(f) = K(a_1, \ldots, a_n)$. $f \in K[x] = K(a_1, \ldots, a_n)$. Con $\Omega = \sdf{a_i}$ todas las raíces distintas entre sí de $f$ en $L$.
        Por el teorema \ref{thm:3.11}(a), si $\sigma \in \gal(E/K)$ $\sigma(L)=L$.
        \item[(b)] Supongamos $E/K$ normal.
        \begin{itemize}
            \item[$(\implies)$] Directo de firma análoga al apartado (a).
            \item[$(\ \Longleftarrow \ )$] Suponemos que $\tau \in \gal(E/K),\ \tau(L) = L$ y queremos probar que $L/K$ es normal. Para ello, usamos el teorema \ref{thm:3.9}.\\

            Sea $p \in K[x]$ irreducible con una raíz $a \in L$m queremos ver que $p$ se escinde en $L$. Por el teorema \ref{thm:3.9}, $p$ se escinde en $E$ (pues $E/K$) es normal. Sea $b \in E$ una raíz de $p$, por el corolario \ref{cor:3.10}, $\exists \tau \in \gal(E/K)$ tal que $\tau(a) = b$ y además $a \in L \implies b = \tau(a) \in L$.\\

            Como $L$ tiene todas las raíces de $p$, entonces $p$ se escinde en $L$. La última parte se demuestra de forma directa aplicando el teorema \ref{thm:3.11}(c).
        \end{itemize}
    \end{itemize}
\end{proof}

\begin{obs}
    Puede parecer que el corolario \ref{cor:3.12}(b) no nos da una caracterización por que requiere estar dentro de una extensión normal, pero si $L/K$ es finita, existe una extensión $E/K$, $L \subseteq E$, tal que $E/K$ es normal (y es la menor con estas propiedades). $E$ es la \textit{clausura normal} de $L/K$.\\\\

    Por ejemplo, sea $L = K(a_1, \ldots, a_n)$ (por ejemplo adjuntando una $K$-base). Sea $a_i \in L$ es algebraico sobre $K$. Para constuir $E$ la clausura normal:
    $$
        p_i = Irr(K, a_i),\ f = \prod_{i=1}^{n} p_i, \text{ entonces } E = K(f)
    $$
\end{obs}

\begin{pro}[Finitud de las extensiones de Galois]\label{pro:ext-gal-finit}
    Sea $L/K$ una extensión, si $L/K$ es finita entonces $\gal(L/K)$ es finita.
\end{pro}

\begin{proof}
    Si $L/K$ es normal, entonces sea $L = K(a_1, \ldots, a_n)$ y $\Omega = \sdf{a_i}$ todas las raíces distintas entre sí de $f \in K[x]$. Por \ref{thm:3.11}(d), $\gal(L/K) \leq S_\Omega \isom S_n$, donde $S_n$ es el enésimo grupo simétrico.\\
    Sabemos que $\vabs{S_n} = n! \implies \text(por Lagrange) \vabs{\gal(L/K)} \divides \vabs{S_n} = n!$.\\
\end{proof}

\begin{eg}[Ejemplos de ejercicios de extensiones de Galois]$ $
    \begin{enumerate}
        \item Sea $E = \quo{\F_2[x]}{\gen{x^2 + x + 1}} = \sdf{\bar 0, \bar 1, \bar x, \bar x + \bar 1}$, además $E/\bar \F_2$, de hecho, $E = \bar \F_2(\bar x)$. Comprueba que $E = \bar\F_2(y^2 + y + 1)$ y calcula $\gal(E/\bar \F_2)$.\\\\

        Por el lema $\ref{lm:3.3}$ $\bar x$ es una raíz, y $y^2 + y + 1 \in \bar\F_2[y]$. Sabemos además que $y^2 + y + 1$ tiene como máximo dos raíces distintas en su cuerpo de escisión.\\
        $\gal(E/\bar\F_2)$ manda $\bar x$ en otra raíz de $y^2 + y + 1$. Es decir, sea $\sigma \in \gal(E/K), f\in K[x], a\in E$ una raíz de $f$, entonces:
        $$
            0 = \sigma(0) = \sigma(f(a)) = \sigma(f)(\sigma(a)) = f(\sigma(a))
        $$
        Sabemos además que $Frob \in \gal(E/\bar \F_2)$ donde $Frob$ es el automorfismo de Frobenius, ya que fija el cuerpo primo de $E$ que es $\bar \F_2$. Por tanto:
        $$
            Frob(\bar x) = \bar x^2 = \bar x + \bar 1 \in E \text{ es otra raíz de } y^2 + y + 1
        $$
        Por tanto, $E = \bar \F_2(\bar x, \bar x + \bar 1)$, es el cuerpo de escisión de $y^2 + y + 1$.\\

        Para calcular $\gal(E/\bar \F_2)$, por \ref{thm:3.11}(d): $\sdf{id, Frob} = \gal(E/\bar \F_2) \leq S_2$ Donde $\vabs{S_2} = 2 \implies \gal(E/\bar \F_2) = \sdf{id, Frob} = \gen{Frob}$.

        \item Sea $f(x) = x^3 - 2 \in \Q[x]$.
        \begin{itemize}
            \item[(a)] $E = \Q(f)$.
            \item[(b)] $\vabs{E: \Q}$
            \item[(c)] Calcula $\gal(E/\Q)$.
        \end{itemize}

        Vamos a ver la solución:
        \begin{itemize}
            \item[(a)] Las raíces de $f$ en $\C$ son $\sdf{\sqrt[3]{2}, \sqrt[3]{2} \omega, \sqrt[3]{2} \omega^2}$ donde:
            $$
                \omega = exp\left( \frac{2\pi}{3} \right) = -\frac{1}{2} + \frac{\sqrt 3}{2} i
            $$
            y por tanto:
            $$
                E = \Q(\sqrt[3]{2}, \sqrt[3]{2} \omega, \sqrt[3]{2} \omega^2) = \Q(\sqrt[3]{2}, \omega) = \Q(\sqrt[3]{2}, \sqrt{3} i)
            $$
            \item[(b)] Sabemos por el teorema \ref{thm:2.3} que $\vabs{\Q(\sqrt[3]{2}):\Q} = 3$ y $\vabs{\Q(\omega):\Q} = 2$. Por tanto:
            $$
                \Q \subseteq \Q(\sqrt[3]{2}) \subseteq \Q(\sqrt[3]{2})(\omega) \implies \vabs{\Q(\sqrt[3]{2}, \omega):\Q} = 6
            $$
            \item[(c)] Sea $L = \Q(\omega)$, sabemos que $\Q \subseteq L \subseteq E$. Por el teorema \ref{thm:3.11}(d):
            $$
                \gal(E/\Q) \leq S_3 \implies \vabs{\gal(E/\Q)} \divides 6
            $$
            Por el teorema \ref{thm:2.1}, sabemos que una $\Q$-base de $E$ es $\sdf{1, \sqrt[3]{2}, \sqrt[3]{4} \sqrt[3]{2} \omega, \sqrt[3]{4} \omega,  \omega}$. Luego $\tau \in \gal(E/\Q)$ queda determinado por $\tau(\sqrt[3]{2})$ y $\tau(\omega)$.\\
            Las distintas posibilidades de $\tau$ son:
            $$
                \tau(\omega) \in \sdf{\omega, \bar \omega} \text{ y } \tau(\sqrt[3]{2}) \in \sdf{\sqrt[3]{2}, \sqrt[3]{2} \omega, \sqrt[3]{2} \bar \omega}
            $$
            Donde con un rápido cálculo combinatorio vemos que hay como mucho $6$ posibles $\tau$.
            %% TODO: Completar de Santorum.
        \end{itemize}
    \end{enumerate}
\end{eg}

%%%%%%%%%%%%%%%%%%%%%%%%%%%%%%%%%%%%%%%%%%%%%%%%%%%%%%%%%%%%%%%%%%%%%%%%%% 31/10
\begin{eg}[Hallar grupo de Galois dado un polinomio y una extensión]
    Sea $f \in \Q[x]$, $f(x) = x^4 - 2$. Y sea $E = \Q(\sqrt[4] 2)$, entonces el conjunto $\Omega$ del teorema \ref{thm:3.11} es $\Omega = \sdf{\sqrt[4] 2, -\sqrt[4] 2}$. Sea $L = \Q(\Omega) = E$, por el teorema \ref{thm:3.11}(d), $\gal(E/\Q) \leq S_\Omega \isom S_2 \implies |\gal(E/\Q)| = 1$ o $2$ por el teorema de Lagrange de grupos. Además:
    $$
        \gal(E/\Q) = \aut(E) = \aut(\Q(\sqrt[4]{2}))
    $$
    Por \ref{cor:2.6}, $\exists K$-isomorfismo $\sigma$ tal que:
    $$
        \sigma: \Q(\sqrt[4]{2}) \to \Q(-\sqrt[4] 2) = E;\ \ \sqrt[4]{2} \mapsto -\sqrt[4]{4} \implies \sigma \in \gal(E/\Q) = \sdf{1, \sigma}
    $$
    Considerando las extensiones:
    %   TODO
    % \Q(\sqrt[4]{2}) = E \to  E
    %                   |      |
    %                   |      |
    % \Q(\sqrt{2})    = L \to  L
    %                   |      |
    %                   \Q \to \Q
    %
    Donde la extensión $L/\Q$ es normal pero $E/\Q$ no.  Además, sea $\theta: L \to L;\ \sqrt 2 \mapsto \sqrt 2$ no se extiende a $E$.\\

    Este ejemplo muestra que la hipótesis $E/K$ normal es necesaria en el teorema \ref{thm:3.8bis}.
\end{eg}

\section{Extensiones separables}

El siguiente teorema y proposición se vieron en el capítulo 1 pero vamos a reescribirlos con el lenguaje actual, es por ello que no se aporta demostración de los mismos.

\begin{thm}[Derivada formal y raíces multiples]\label{thm:3.13}\label{thm:3.4.1}$ $
    \begin{itemize}
        \item[(a)] Si $f \in K[x]$ tal que $mcd(f, f') = 1$ y $f' \neq 0$ donde $f'$ es la derivada formal, entonces todas las raíces de $f$ en su cuerpo de escisión son distintas.
        \item[(b)] Si $f \in K[x]$ irreducible y $f'$ su derivada formal tal que $f' \neq 0$, entonces $f$ es separable.
    \end{itemize}
\end{thm}
\begin{obs}
    Si $\car K = 0$ y $f\in K[x]$ con $\delta f > 1$, entonces $f' \neq 0$.
\end{obs}

\begin{eg}[Aplicación del contrarrecíproco del teorema \ref{thm:3.13}]
    Sea $f(x) = x^p -1$, $\car K = p$, vemos que $f' = 0$. Entonces no todas las raíces de $f$ en su cuerpo de escisión son distintas, de hecho $f = (x-1)^p$.
\end{eg}

\begin{pro}[Multiplicidad de raíces y derivada formal]\label{pro:3.4.2}
    Sea $E/K$ una extensión, $f \in K[x]$  un polinomio y $a \in E$ tal que $f(a) = 0$, y sea $m$ la multiplicidad de la raíz $a$, entonces:
    $$
        m > 1 \iff f(a)=0=f'(a)
    $$
\end{pro}

\begin{dfn}[Separabilidad. Polinomios, elementos y extensiones]$ $
    \begin{itemize}
        \item[($i$)] Sea $f \in K[x]$ un polinomio, decimos que es un \textbf{polinomio separable} si todas sus raíces en un cuerpo de escisión son distintas, es decir, $f' \neq 0$ (por el teorema \ref{thm:3.13}).
        \item[($ii$)] Sea $E/K$ una extensión, $a \in E$ un elemento algebraico, decimos que $a$ es un \textbf{elemento separable} si $p = Irr(K, a)$ es separable.
        \item[($iii$)] Sea $E/K$ una extensión algebraica, decimos que es una \textbf{extensión separable} si todo $a \in E$ es separable.
    \end{itemize}
\end{dfn}

\begin{pro}[Divisores irreducibles de un polinomio separable]\label{pro:sepdiv}
    Sea $f \in K[x]$ un polinomio separable, entonces para todo $p$ irreducible tal que $p \divides f$, $p$ es separable.
\end{pro}
\begin{proof}
    Si $p$ divide a $f$ quiere decir que todas sus raíces están contenidas en las de $f$, por tanto, si $f$ es separable, $p$ también lo es.
\end{proof}

\begin{pro}[Separabilidad de extensiones intermedios]
    Sea $E/K$ una extensión separable, y $L$ un cuerpo intermedio, entonces, $E/L$ y $L/K$ son separables.
\end{pro}
\begin{proof}
    Es fácil ver que $L/K$ es separable. Vamos a ver que $E/L$ es separable.\\
    Sean $a \in E$, $p = Irr(L, a)$. Y sea $q = Irr(K, a)$ que es separable por hipótesis. Como $p \divides q$ (en $L[x]$), sea $M$ un cuerpo de escisión de $q$ sobre $L$ (que sabemos que existe por el teorema \ref{thm:3.4}), entonces $q$ se escinde en $M \implies p$ se escinde en $M$ por \ref{lm:3.1}($ii$).\\

    Si $\alpha, \beta$ son raíces de $p$ en $M$, enotnces $\alpha, \beta$ son raíces de $q$ en $M \implies \alpha \neq \beta$.
\end{proof}

\begin{lm}[Separabilidad en cuerpos de característica $0$]\label{lm:3.4.3}
    Todo polinomio $f \in K[x]$ con $\car K = 0$ es separable y toda extensión $E/K$ algebraica es separable.
\end{lm}
\begin{proof}
    Se deja como ejercicio.
\end{proof}

\begin{cor}[Separabilidad en cuerpos de característica $p$]\label{cor:3.4.4}
    Sea $f \in K[x]$ irreducible con $\car K = p$:
    $$
        f(x) \text{ es separable } \iff f \notin K[x^p]
    $$
\end{cor}

Sin embargo el corolario no nos resuelve una pregunta. \textit{¿Podemos encontrar polinomios irreducibles que no sean separables?}, es decir, \textit{ ¿podemos encontrar un polinomio irreducible en $K[x^p]$?}.\\

Recordemos que un cuerpo $K$ es perfecto si el monomorfismo de Frobenius es un isomorfismo, es decir, $\varphi = Frob \in \aut(K)$, donde $\varphi(a) = a^p$.\\
En particular $\forall b \in K, \exists a \in K$ tal que $a^p = b$

\begin{thm}[Irreducibilidad sobre cuerpos perfectos]\label{thm:3.4.5}
    Sea $K$ un cuerpo perfecto, todo polinomio irreducible sobre $K$ es separable. En particular, todo polinomio es separable.
\end{thm}
\begin{proof}
    Sea $K$ perfecto, entonces:
    $$
        \forall b \in K,\ \exists a \in K:\ a^p = b.
    $$
    Sea $f\in K[x]$ irreducible. Sabemos por el corolario \ref{cor:3.4.4} que si $f \notin K[x^p]$, entonces $f$ es separable. Supongamos que $f \in K[x^p]$ de la forma:
    \begin{align*}
        f &= b_0 + b_1 x^{pn_1} + \cdots + b_k x^{pn_k}\\
          &= a_0^p + \cdots + a_k^p (x^{n_k})^p\\
          &= a_0^p + \cdots + (a_k x^{n_k})^p\\
          &= (a_0 + \cdots a_k x^{n_k})^p \text{ que no es irreducible }
    \end{align*}
\end{proof}

\begin{ex}[H3.16]
    ¿Cuántas raíces distintas tiene $x^{12} + 2x^6 + 1 \in \F_3[x]$ en su cuerpo de escisión?\\\\

    \begin{align*}
        x^{12} + 2x^6 + 1 &= (x^4)^4 + (-1)^3 (x^2)^3 + 1^3\\
                          &= (x^4 -x^2 +1) ^3\\
                          &= ((x^2 + 1)^2)^3 = (x^2 + 1)^6
    \end{align*}
    Sabemos que $x^2 + 1$ es separable por ser $\F_3$ perfecto. Además como es separable es irreducible y por tanto, tiene dos raíces distintas en su cuerpo de escisión.
\end{ex}

%%%%%%%%%%%%%%%%%%%%%%%%%%%%%%%%%%%%%%%%%%%%%%%%%%%%%%%%%%%%%%%%%%%%%%%%%% 04/11
\begin{eg}[Polinomio irreducible que no es separable]
    Vamos a buscar un polinomio $f$ irreducible que no sea separable. Además por el lema \ref{lm:3.4.3} y el teorema \ref{thm:3.4.5}, tenemos que buscar $f \in K[x]$ tal que $\car K = p$ con $K$ no perfecto.\\
    Por ejemplo, si consideramos $K = \F_p$, podemos ver que $K$ no es perfecto ya que $\varphi = Frob \in \mathrm{End}(\F_p(t))$.\\
    Por otra parte, $\varphi(\F_p(t)) = \F_(t^p) \subsetneq \F_p(t)$. Vamos a considerar entonces la extensión $\F_p(t)/F_p(t^2)$, donde podemos ver que $\vabs{\F_p(t) : \F_p(t^2)} = p$ pues tenemos que $t$ es raíz de $x^p - t^p \in \F_p(t^p)[x]$.\\

    Tenemos que ver ahora que $x^p - t^p = (x - t)^p$ es irreducible. En otro caso, $x^p - t^p = g(x) \cdot h(x)$, con $\delta g,\ \delta h \leq p$. Por tanto:
    $$
        g(x) = (x - t)^n \text{ con } 0 < n < p
    $$
    Lo que es imposible por que el término independiente de $g$ es $t^n$ que no pertenece a $\F_p(t^p)$. Luego $Irr(\F_p(t^p), t) = x^p - t^p$.
\end{eg}

\begin{pro}[Raíces comunes y polinomios irreducibles mónicos]
    Sean $p \neq q \in K[x]$ irreducibles mónicos, entonces $p$ y $q$ no pueden tener una raíz en común.
\end{pro}

\begin{proof}
    Sea $E/K$ una extensión y $\alpha \in E$ la raíz común de $p$ y $q$, entonces $p = Irr(K, \alpha) = q$ por la unicidad del polinomio mónico.\\\\
    Otra forma de verlo es recordar que sean $f, g \in K[x]$ entonces $mcd_K(f, g) = mcd_E(f, g)$. Si $\alpha$ es una raíz común entonces $p \divides q$ y además $(x - \alpha) \divides mcd(p, q) > 1 \implies q \divides p$ y por tanto $p = q$.
\end{proof}

\begin{ex}[H3.??]
    Calcular las raíces distintas de $f \in K[x]$ separables en su cuerpo de escisión.\\\\
    Vamos a dar un esquema de la resolución del ejercicio:
    \begin{enumerate}
        \item Descomponer $f$ es factores irreducibles:
        $$
            f = a p_1^{e_1} p_2^{e_2} \ldots p_r^{e_r} \text{ donde los $p_i$ son irreducibles y distintos entre sí.}
        $$
        \item Por definición (por ser separable) cada $p_i$ es separable y además el número de raíces de $p_i$ coincide con $\delta p_i$.\\
        Con esto vemos que el número de raíces distintas de $f$ es igual a $\sum \delta p_i$.
    \end{enumerate}
\end{ex}


\begin{dfn}[Elemento primitivo]
    Sea $E/K$ una extensión simple, a un elemento de $\gamma \in E$ se le denomina \textbf{elemento primitivo} si $E = K(\gamma)$.
\end{dfn}

\begin{thm}[del Elemento Primitivo]\label{thm:elemprim}
    Toda extensión finita y separable es simple.
\end{thm}

\begin{proof}
    Se proporciona un esquema de la demostración. Sea $E/K$ finita y separable, entonces $E = K(\alpha_1, \ldots, \alpha_n)$ con $\alpha_i$ algebraicos y separables.
    \begin{itemize}
        \item Por inducción sobre $n$, basta probar el resultado cuando $n =2$, con $n = 1$ habríamos terminado pues $E = K(\alpha_1)$ y por tanto sería simple. Vamos a considerar el caso $n > 1$, es decir $L = K(\alpha_1, \ldots, \alpha_{n-1})$. Por inducción $L = K(\alpha)$ y entonces $E = K(\alpha_1, \ldots, \alpha_n) = K(\alpha_1, \alpha_{n-1})(\alpha) = K(\alpha)(\alpha_n) = K(\alpha, \beta)$
        \item Queremos probar que $K(\alpha, \beta)/K$ es simple. Esta parte es constructiva. Habría que conectar separabilidad y normalidad.
    \end{itemize}
\end{proof}


\begin{pro}[Irreducibilidad y acción del grupo de Galois] % Lo vimos como ejercicio
    Sea $f \in K[x]$ un polinomio, $E = K(f)$ y $G = \gal(E/K)$. Entonces:
    \begin{enumerate}
        \item Si $f$ es irreducible, $G$ actúa transitivamente (la acción define una sola G-órbita) sobre el conjunto $\Omega$ de raíces de $f$.
        \item Si $f$ no tiene raíces múltiples en $E$ y $G$ actúa transitivamente sobre $\Omega$, entonces $f$ es irreducible.
    \end{enumerate}
\end{pro}
\begin{proof}$ $
    \begin{enumerate}
        \item Se sigue inmediatamente del corolario \ref{cor:3.10}.
        \item Sea $f = a p_i \ldots p_r$ con $p_i$ primos irreducibles. Supongamos que $r > 1$, sea $\alpha \in E$ raíz de $p_1$, $\beta \in E$ raíz de $p_2$. Por hipótesis $\exists \sigma \in G$ tal que $\alpha \cdot \sigma = \sigma(\alpha) = \beta$. Entonces:
        $$
            0 = \sigma(p_1(\alpha)) = \sigma(p_1)(\sigma(\alpha)) = p_1 \beta \implies Irr(K, \alpha) = p_1 = Irr(K, \beta) = p_2
        $$
    \end{enumerate}
\end{proof}

\begin{thm}[Numero de extensiones de un isomorfismo]\label{thm:3.4.6}
    Sea $\sigma: K_1 \to K_2$ un isomorfismo, $f_1 \in K_1[x]$, $f_2 = \sigma(f_1) \in K_2[x]$ polinomios y $E_i = K(f_i)$ extensiones. Entonces:
    \begin{itemize}
        \item[(a)] $\sigma$ se extiende a un isomorfismo $\tau: E_1 \to E_2$. (Teorema \ref{thm:3.4.5}).
        \item[(b)] Si $f_1$ es separable entonces hay exactamente $\vabs{E_1:K_1}$ extensiones como en (a).
    \end{itemize}
\end{thm}

%%%%%%%%%%%%%%%%%%%%%%%%%%%%%%%%%%%%%%%%%%%%%%%%%%%%%%%%%%%%%%%%%%%%%%%%%% 05/11

\begin{proof}$ $
    \begin{itemize}
        \item[(a)] Teorema \ref{thm:3.4.5}.
        \item[(b)] Por inducción sobre $\vabs{E_1:K_1}$.
        \begin{itemize}
            \item $\vabs{E_1:K_1} = 1 \iff g_1$ se escinde en $E_1 \iff f_2$ se escinde en $E_2 \iff \vabs{E_2:K_2} = 1 \implies \tau = \sigma$
            \item Podemos suponer que $\vabs{E_1:K_1} > 1$, entonces:
            $$
                f_1 = p_1g_1 \in K_1[x], \text{ donde } p_1 \text{ es irreducible y } \delta p_1 = d \geq 2.
            $$
            por tanto:
            $$
                f_2 = p_2g_2, \text{ donde } p_2 = \sigma^\star(p_1) \text{ es irreducible y } \delta p_2 = \delta p_1 = d \geq 2.
            $$
            Como $f_1$ es separable, entonces $p_1$ es separable (por la proposición \ref{pro:sepdiv}). Como $E_1 = K_1(f_1)$, entonces $p_1$ se escinde en $E_1$ (por el lema \ref{lm:3.1}(ii)). Ahora, desarrollando:
            \begin{align*}
                p_1(x) = (x-\alpha_1) \cdots (x - \alpha_d) \in E_1[x]\\
                \text{Por (a)} \exists \tau:E_1 \to E_2 :\ \rest{\tau}_{K_1} = \sigma\\
                p_2 = \tau^\star(p_1) = (x - \tau(\alpha_1)) \cdots (x - \tau(\alpha_d))\\
                \beta_j = \tau(\alpha_j) \text{ donde las $\beta_j$ son las raíces distintas de } p_2 \text{ en } E_2.
            \end{align*}
            y por tanto $p_2$ es spearable.\\\\
            Sea $\alpha = \alpha_1$, por el teorema \ref{thm:2.5}, para cada $j = 1, \ldots, d$, existe un isomorfismo:
            $$
                \hat\sigma : K_1(\alpha) \to K_2(\beta_j)
            $$
            que extiende a $\sigma$. Como:
            $$
                \vabs{E_1:K_1(\alpha_1)} = \frac{\vabs{E_1:K_1}}{\vabs{K_1(\alpha):K_1}} = \frac{\vabs{E_1:K_1}}{d} < \vabs{E_1:K_1}.
            $$
            Por inducción, para cada $j = 1, \ldots, d$, existen $\vabs{E_1:K_1}$ extensiones exactamente de $\hat\sigma_j$.\\
            Es fácil ver que tenemos $\frac{\vabs{E_1:K_1}}{d} \cdot d$ extensiones de $\sigma$ a $E_1 \to E_2$. ¿Por qué son todas?.\\

            Si $\tau: E_1 \to E:2$ extiende a $\sigma$, en particular $\tau$ manda $\sigma$ en una raíz de $\tau^\star(p_1) = \sigma^\star(p_1) = p_2$. Es decir, $\tau(\alpha) = \beta_j$ luego, $\rest{\tau}_{K_1(\alpha)} = \hat\sigma_j$.
            Por tanto, $\sigma$ tiene exactamente $\vabs{E_1:K_1}$ extensiones a $E_1 \to E_2$.
        \end{itemize}
    \end{itemize}
\end{proof}

\begin{dfn}[Extensión de Galois]
    Sean $E, K$ cuerpos, diremos que la extensión $E/K$ es una \textbf{extensión de Galois} si $E/K$ es normal y separable.
\end{dfn}
\begin{obs}
    Si la extensión $E/K$ es de Galois, y sea $L$ un cuerpo intermedio, entonces $E/L$ es una extensión de Galois.
\end{obs}

\begin{cor}\label{cor:3.4.7}
    Sea $E/K$ una extensión de Galois, y $K \subseteq L \subseteq E$, $L/K$ es una extensión de Galois. Además, cada $\sigma \in \gal(L/K)$ tiene exactamente $\vabs{E:L}$ extensiones a $\gal(E/K)$. En particular:
    $$
        \vabs{\gal(E/K)} = \vabs{E:K}
    $$
\end{cor}
\begin{proof}
    Consecuencia del teorema \ref{thm:3.4.6}. Los detalles se dejan como ejercicio.
\end{proof}

\begin{ex}[Ejercicio] %TODO: Añadir numeración
    Sea $\xi$ la raíz primitiva quinta de $\1$. Demuestra o calcula:
    \begin{itemize}
        \item[(a)] $\Q(\xi)/\Q$ es normal.
        \item[(b)] $\vabs{\Q(\xi):\Q}$.
        \item[(c)] $\gal{\Q(\xi)/\Q}$.
    \end{itemize}

    \textbf{Solución:}\\\\
    \begin{itemize}
        \item[(a)] $\xi$ es raíz de $x^5 - 1$ y las raíces de $x^5 - 1$ son $\sdf{1, \xi, \xi^2, \xi^3, \xi^4}$, que son las raíces del quinto polinomio ciclotómico. Entonces $\Q(\xi) = \Q(x^5 - 1)$. Por tanto, el polinomio irreducible $Irr(\Q, \xi) = \phi_5(x) = x^4 + x^3 + x^2 + x + 1$.
        \item[(b)] $\vabs{\Q(\xi):\Q} = \delta \phi_5(x) = 4$
        \item[(c)] Sabemos que $\Q(\xi)/\Q$ es una extensión de Galois, (es normal por (a)  y separable por que $\car \Q = 0$). Además:
        $$
            \vabs{\gal(\Q(\xi)/\Q)} = 4, \text{ (usando el teorema \ref{thm:3.11}(d) sabemos que } \gal(\Q(\xi)/\Q) \leq S_4 \text{)}
        $$
        Por el corolario \ref{cor:3.10}, para $j = 1, 2, 3, 4$ sabemos que $\exists \sigma_j \in \gal(\Q(\xi)/\Q)$ tal que $\sigma_j(\xi) = \xi^j$, donde $\sigma_1 = id$. Y además, un $\sigma \in \gal(\Q(\xi)/\Q) = \sdf{id, \sigma_2, \sigma_3, \sigma_4}$ está determinado por la imagen de $\xi$. De hecho, $\gal(\Q(\xi)/\Q) = \gen{\sigma_2} \isom C_4$.
    \end{itemize}
\end{ex}

\section{Cuerpos finitos}

Vamos a empezar con un recordatorio de \textit{Teoría de Grupos}.

\begin{dfn}[Orden del elemento de un grupo]
    Sea $G$ un grupo finito, $g \in G$ un elemento, decimos que el \textbf{orden} de $g$ es:
    $$
        o(g) = n = \min\sdf{m \mid g^m = 1 } < \infty
    $$
\end{dfn}

\begin{thm}[Resultados de orden en un grupo]\label{thm:3.5.1}
    Sea $G$ un grupo finito, $g \in G$ un elemento del grupo:
    \begin{itemize}
        \item[(a)] $g^m = 1 \iff o(g) \divides m$.
        \item[(b)] $o(g) \divides \vabs{G}$. En particular $g^{\vabs{G}} = 1$.
        \item[(c)] Si $o(g) = n$, entonces $\gen{g} = \sdf{1, g, \ldots, g^{n-1}}$ y $\gen{g} = \gen{g^k} \iff mcd(g, k)=1$.
        \item[(d)] $G$ es cíclico $\iff \exists g \in G:\ o(g) = \vabs{G}$.
    \end{itemize}
\end{thm}

\begin{obs}
    Sea $K$ un  cuerpo finito y $F$ su cuerpo primo. Entonces:
    $$
        F \isom \F_p \text{ y además } \vabs{K:F} = n
    $$
    $$
        K \isom F^n \isom \F_p^n \implies \vabs{K} = p^n
    $$
\end{obs}

\begin{thm}[Clasificación de cuerpos finitos]\label{thm:3.5.2}$ $
    \item[(a)] Para cada primo $p$ y natural $n$, existe un cuerpo $\vabs{E} = p^n$.
    \item[(b)] Dos cuerpos $E$ y $L$ con el mismo número de elementos son isomorfos.
\end{thm}

\begin{obs}
    $$
        \F_{p^n} = GF(p^n)
    $$
\end{obs}

\begin{proof}$ $
    \begin{itemize}
        \item[(a)] Sea $F = \F_p$ y $f(x) = x^{p^n} - x \in F[x]$, entonces $E = F(f)$ (por el teorema \ref{thm:3.4}).\\
        Sea $U$ el conjunto de raíces de $f$ en $E$, y como $f'(x) = -1$, entonces $mcd(f, f') = 1$. Por el teorema \ref{thm:3.4.1}(a), obtenemos que $\vabs{U} = p^n$.\\

        \begin{enumerate}
            \item $F \subseteq U$ ya que $\forall a \in F$ tenemos que $a^p = a$ (por el pequeño teorema de Fermat).
            \item $U$ es un cuerpo:
            \begin{align*}
                \alpha, \beta \in U \text{ consideramos } \alpha^{p^n} = \alpha, \beta^{p^n} = \beta \\
                (\alpha-\beta)^{p^n} = \alpha^{p^n} - \beta^{p^n} = \alpha - \beta \\
                \text{Si } \beta \neq 0 \implies \frac{\alpha^{p^n}}{\beta^{p^n}} = \frac{\alpha}{\beta}
            \end{align*}
            Y por tanto, $F \subseteq U \subseteq E = F(U) \implies E = F(U) = U$.
        \end{enumerate}
        \item[(b)] Supongamos que $L$ es otro cuerpo con $\vabs{L} = p^n$. Sea $D \subseteq L$ su cuerpo primo, $D \isom \F_p \isom F$. Ahora $L^\times = \uds L$ es un grupo de $p^n - 1$ elementos. En particular si $l \in L^\times$, entonces $l^{p^n-1} = 1$. (por \ref{thm:3.5.1}(b)) $\implies l^{p^n} = l$. Es decir, $L$ es un cuerpo de escisión de $f(x) = x^{p^n} - x$. Entonces tenemos el diagrama (que demuestra el isomorfismo):
        %% TODO
        %% E = F(f) -> L = D
        %%      |        |
        %%      |  isom  |
        %%      F -----> D
        %%
        con $f(n) = x^{p^n} -  x$.
    \end{itemize}
\end{proof}

\begin{obs}
    Sea $K$ u cuerpo finito, y $F$ su cuerpo primo, como consecuencia del teorema \ref{thm:3.5.2}, $K/F$ es una extensión de Galois y si $\vabs{K:F} = n$ entonces $\vabs{\gal(K/F)} = n$.
\end{obs}
