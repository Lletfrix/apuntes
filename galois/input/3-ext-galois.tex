% !TeX root = ../galois.tex

\chapter{Extensiones de Galois}

\section{Cuerpos de escisión}

\begin{dfn}[Escisión de un polinomio]
    Sea $f \in K[x]$, $K$ cuerpo, decimos que $f$ se \textbf{escinde} si $\exists a, a_i\in K$ si:
    $$
        f(x) = a(x - a_1) \ldots (x - a_n)
    $$
\end{dfn}

\begin{obs}
    Si $f(x) \in K[x]$, $\delta f = 1$, entonces $f$ se escinde en $K$.
    $$
        f(x) = ax + b = a \left(x - \left(- \frac{b}{a}\right)\right),\ \ \ a, -\frac{b}{a} \in K
    $$
\end{obs}

\begin{obs}
    Si $f \in K[x]$, se escinde en $E$, sea $E/K$ una extensión:
    $$
        f(x) = a (x - a_i) \ldots (x - a_n) \in E[x]
    $$
    Entonces las raíces de $f$ en cualquier extensión de $E$ son $a_1, \ldots, a_n$
\end{obs}

\begin{dfn}[Cuerpo de escisión]
Sea $f \in K[x]$, entonces $E$ es un cuerpo de escisión de $f$ sobre $K$ si:
    \begin{itemize}
        \item $E/K$ es una extensión.
        \item $f$ se escinde en $E$.
        \item Si $f$ se escinde en $L$, con $L$ un cuerpo intermedio $K \subseteq L \subseteq E$, entonces $L=E$, es decir, $E = K(f)$. (Es el menor subcuerpo con la propiedad de que $f$ se escinda en él).
    \end{itemize}
\end{dfn}

\begin{eg}
    $x^2 + 1$ se escinde en $\C$ pero su cuerpo de escisión es $\Q(i)$.
\end{eg}

\begin{obs}
    Si $f \in K[x]$ se escinde en $E$, entonces $E$ tiene todas las raíces de $f$. Para construir su cuerpo de escisión basta adjuntar todas las raíces al cuerpo sobre el que está definido:
    $$
        f(x) = a (x - a_1) \ldots (x - a_n) \in E[x], \text{ entonces su cuerpo de escisión es } K(f) = K(a_1, \ldots, a_n)
    $$
\end{obs}

\begin{lm}[Escisión de polinomios no constantes]\label{lm:3.1}
    Sea $K$ un cuerpo:\\
    \begin{enumerate}
        \item Si $f \in K[x]$ no constante se escinde en $K$ si y sólo si todos los polinomios en la descomposición en términos de factores irreducibles tienen grado $1$.
        \item Si $f \in K[x]$ no constante que se escinde en $K$ y sea $p \in K[x]$ tal que $p \divides f$, entonces $p$ se escinde en $K$.
    \end{enumerate}
\end{lm}

\begin{proof}
    La demostración se deja como ejercicio.
\end{proof}

\begin{lm}[Cuerpos de escisión y cuerpos intermedios]\label{lm:3.2}
    Sean $E/L$ y $L/K$ extensiones, y $f \in K[x]$ un polinomio no constante. Si $E$ es el cuerpo de escisión de $f$ sobre $K$, entonces $E$ es el cuerpo de escisión de $f$ sobre $L$.
\end{lm}

\begin{proof}
    Como $E = K(f)$, tenemos que $f(x) = a (x - a_1) \ldots (x - a_n) \in E[x]$ y $E = K(a_1, \ldots, a_n)$. Además, $f \in K[x] \subseteq L[x]$. Si el cuerpo de escisión de $f$ sobre $L$ es $E'$, fácilmente llegamos a que $E' = L(f) \subseteq E$ y por tanto: $E = E'$ ya que $E$ era el menor cuerpo de escisión de $f$ sobre $K$.
\end{proof}

Si $f$ es un polinomio sobre $\Q[x]$ de grado $n$, \textit{sabemos} que $\exists \alpha_i \in \C$ con $i \in \setdef{1, \ldots n}$ tal que $f(x) = a (x - \alpha_1) \ldots (x - \alpha_n)$, y hemos visto para calcular su cuerpo de escisión basta con adjuntar las raíces de $f$, es decir, $\Q(f) = \Q(\alpha_1, \ldots, \alpha_n)$.\\

Vamos a ver resultados para generalizar este proceso.

\begin{lm}[Lema de Kronecker]\label{lm:3.3}
    Sea $p \in K[x]$ un polinomio irreducible, entonces existe $E/K$ tal que $p$ tiene una raíz en $E$.
\end{lm}

\begin{proof}
    Sea $L = \quo{K[x]}{\gen{p}}$, ya vimos que $L$ es un cuerpo por ser $p$ irreducible. Además se puede comprobar que:
    $$
        \bar x = x + \gen{p} \in L \text{ es raíz de } \bar p \in \bar K[y]
    $$
    Donde $\bar p = \overline{a_0} + \ldots + \overline{a_n}x^n$ y $K \isom \bar K = \setdef{\bar K \mid k \in K}$.

    Hay que comprobar que $\bar K \subseteq L$. Tras ello, habría que ver que $L = \bar K(\bar x)$.
\end{proof}

\begin{thm}[Existencia de cuerpos de escisión]\label{thm:3.4}
    Sea $K$ un cuerpo, $f \in K[x]$ un polinomio no constante. Entonces existe un cuerpo de escisión $E$ de $f$ sobre $K$.
\end{thm}

\begin{proof}
    Basta construir una extensión en la que $f$ se escinda. La demostración sigue por inducción sobre $\delta(f)$.
    \begin{itemize}
        \item[$\delta(f) = 1$] Entonces por el lema \ref{lm:3.1}, $K$ es un cuerpo de escisión de $f$ sobre $K$.
        \item[$\delta(f)>1$] Sea $p \divides f$ un factor irreducible de $f$ en $K[x]$, por el lema \ref{lm:3.3}, sabemos que existe $E/K$ en la que $p$ tiene una raíz $\alpha \in E$, en particular $f(\alpha) = 0$ y por Ruffini:
        $$
            f(x) = (x - \alpha) g(x),\text{ con } g(x) \in K(\alpha)[x]
        $$
        Como $1 \leq \delta(g) \leq \delta(f) - 1 \leq \delta(f)$, por inducción sabemos que existe un cuerpo de escisión $L$ de $g$ sobre $K(\alpha)$. Si $\alpha_1, \ldots \alpha_n \in L$ son las raíces de $g$, entonces:
        $$
            L = K(\alpha)(\alpha_1, \ldots \alpha_n) = K(\alpha, \alpha_1, \ldots, \alpha_n)
        $$
        y por tanto $f(x) = (x - \alpha) g(x) = (x - \alpha) p (x - \alpha_1) \ldots (x - \alpha_n) \in L[x]$, así que $f$ se escinde sobre $L$, y de hecho $L$ es un cuerpo de escisión de $f$ sobre $K$.
    \end{itemize}
\end{proof}

\begin{ex}[Cálculo del cuerpo de escisión]
    Describir el cuerpo de escisión de $f(x) = x^4 - 4x^2 + 2$ sobre $\Q$.\\\\
    Comenzamos hallando las raíces, que resulta fácil pues $f(x) = 0$ es una ecuación bicuadrática. Tenemos entonces las raíces:
    $$
        \alpha = \sqrt{2 + \sqrt 2},\ \beta = \sqrt{2 - \sqrt{2}},\ -\alpha,\ -\beta
    $$
    De donde vemos que $f$ se escinde en $R$ ya que todas las raíces son reales. Para ver su cuerpo de escisión vamos a adjuntar las raíces a $\Q$.
    $$
        E = \Q(\alpha,  \beta, -\alpha, -\beta) = \Q(\alpha, \beta)
    $$
    Sin embargo, podemos reducirlo aún más si vemos que $\Q(\alpha, \beta) = \Q(\beta)$.
    $$
        \alpha\beta = \sqrt{(2 + \sqrt 2)(2 - \sqrt 2)} = \sqrt 2,\ \alpha^{-1} = \left(\frac{\beta}{\sqrt 2}\right) \in \Q(\alpha) \implies \beta = \frac{\beta}{\sqrt 2} \cdot \sqrt 2 \in \Q(\alpha)
    $$
    Con lo que el cuerpo de escisión es: $\Q(\sqrt{2 + \sqrt 2})$.
\end{ex}

\begin{obs}
    Si $E = K(f)$, entonces $E/K$ es finita por el teorema \ref{thm:extalg}. Anteriormente, vimos que si $f \in K[x]$, entonces existe un cuerpo de escisión de $f$ sobre $K$.\\
    Sea $p$ un polinomio irreducible tal que $p \divides f$ y $\delta p > 1$, entonces consideramos $\quo{K[x]}{\gen{p}}$ y en ese cuerpo hay una raíz de $p$ (que también es raíz de $f$), y después aplicamos inducción.
\end{obs}

\begin{thm}\label{thm:3.5}
    Sea $K_1, K_2$ cuerpos, y $\sigma: K_1 \to K_2$ un isomorfismo de cuerpos. Sea $f_1 \in K_1[x]$ no constante y $\delta(f_1) = f_2$.\\ Supongamos que $E_i$ son cuerpos de escisión de $f_1$ sobre $K_i$ con $i \in \setdef{1, 2}$. Entonces existe un isomorfismo $\tau: E_1 \to E_2$ que extiende a $\sigma$, es decir, $\left. \tau \right|_{K_1} = \sigma$.
\end{thm}

\begin{proof}
    Pedir a Santorum.
\end{proof}

\begin{cor}[Unicidad de los cuerpos de escisión]
    Sea $K$ un cuerpo y $f \in K[x]$ no constante. Si $E_1$ y $E_2$ son cuerpos de escisión de $f$ sobre $K$ entonces $\exists \tau$ un isomorfismo $\tau: E_1 \to E_2$ tal que $\tau(a) = a$ $\forall a \in K$.
\end{cor}
\begin{proof}
    Basta tomar $\sigma = id_K$ en el teorema \ref{thm:3.5}.
\end{proof}

%% TODO: COMPLETAR EL DIA 22/10

%%%%%%%%%%%%%%%%%%%%%%%%%%%%%%%%%%%%%%%%%%%%%%%%%%%%%%%%%%%%%%%%%%%%%%%%%% 24/10

\begin{lm}[Normalidad de extensiones en cuerpos intermedios]\label{lm:3.7}
    $K \subseteq L \subseteq E$ extensiones, entonces:
    $$
        E/K \text{ normal } \implies E/L \text{ normal.}
    $$
\end{lm}

\begin{proof}
    Es consecuencia directa del lema \ref{lm:3.2} y de la definición.
\end{proof}


\begin{cor}[al teorema \ref{thm:3.5}]
    Sea $E/K$ una extensión normal, $K \subseteq M_1, M_2 \subseteq E$ con $M_1, M_2$ cuerpos intermedios.\\\\
    Si existe un isomorfismo $\sigma: M_1 \to M_2$ que fija a $K$, es decir, $\sigma(a) = a,\ \forall a \in K$, entonces $\sigma$ se extiende a un isomorfismo $\tau: E \to E$ ($\left. \tau \right|_{M_1} = \sigma$).
\end{cor}
\begin{proof}
    Sea $E = K(f)$ con $f \in K[x]$ no constante. En particular por el lema \ref{lm:3.2}, $E = M_1(f) = M_2(f)$. Como $\sigma(f) = f$, el resultado se sigue del teorema \ref{thm:3.5}.
\end{proof}
\begin{obs}
    La hipótesis $\sigma(a) = a,\ \forall a \in K$ es necesaria. La idea es que sin esta hipótesis:
    $$
        E = M_1(f) = M_2(f)
    $$
    pero no sabríamos si $E = M_2(\sigma(f))$.
\end{obs}

\begin{thm}[Condición necesaria y suficiente para la normalidad de una extensión]\label{thm:3.9}
    Sea $E/K$ una extensión finita, $E/K$ es normal si y sólo si todo polinomio $p \in K[x]$ irreducible con una raíz en $E$ se escinde en $E$.
\end{thm}

\begin{proof}$ $
    \begin{itemize}
        \item[$\ \Longleftarrow\ $] Sabemos que $E/K$ es finita, digamos $|E:K| = n$, podemos tomar una $K$-base $\sdf{a_1, \ldots, a_n}$. En particular $E=K(a_1, \ldots, a_n)$.\\
        Como $E/K$ es finita, por el teorema \ref{thm:2.2}, los $a_i$ en son algebraicos sobre $K$. Sea $p_i = Irr(k, a_i) \in K[x]$, por hipótesis cada $p_i$ se escinde en $E$.\\

        Tomamos $f = \prod_{i=1}^{n} p_i \in K[x]$ se escinde en $E$. De hecho, $E = K(f)$. De esta forma hemos probado que:
        $$
            K(f) \subseteq E = K(a_1, \ldots, a_n) \subseteq K(f)
        $$
        \item[$\implies$] Ahora partimos de que $E/K$ es finita y normal. Entonces $\exists f \in K[x]$ tal que $E = K(f)$.\\
        De hecho, si $f(x) = c(x - b_1) \ldots (x-b_n) \in E[x]$, $E = K(b_1, \ldots, b_n)$.\\
        Sea $p\in K[x]$ irreducible con una raíz $a\in E$, por Ruffini (corolario \ref{cor:ruffini}) $p(x) = (x-a)g(x)\in E[x]$. Además, por el teorema \ref{thm:3.4}, existe $M/E$ un cuerpo de escisión de $p$ sobre $E$, por tanto:
        $$
            p(x) = d(x-d_1) \cdot \ldots \cdot (x-d_r), \text{ podemos suponer que $a = d_1$}
        $$
        Dado $d_i$ con $i > 1$, queremos ver que $d_i \in E$. Llamemos $b = d_i$ para algún $i$. Como $a$ y $b$ son raíces de $p \in K[x]$ irreducible, por el teorema \ref{thm:2.5} $\exists \sigma: K(a) \to K(b)$ tal que $\sigma(k) = k,\ \forall k \in K$.

        %
        %\sigma: K(a) --------> K(b)
        %         ^              ^
        %         |              |
        %         v              v
        %id:      K  --------->  K

        Como $K(a) \subseteq E$, en particular $E = K(a)(f)$. Por otro lado $K(b)(f) = E(b)$. Las raíces de $f$ en $E(b)$ son $b_1, \ldots, b_m$, así que:
        $$
            K(b)(f) = K(b)(b_1, \ldots, b_m) = K(b_1, \ldots, b_m) = E(b)
        $$
        Por el teorema \ref{thm:3.5} existe un isomorfismo $\tau: E \to E(b)$ que extiende a $\sigma$.
        %
        %\tau:  E=K(a)(f) ----> E(b) = K(b)(f)
        %         |              |
        %         |              |
        %\sigma: K(a) --------> K(b)
        %         ^              ^
        %         |              |
        %         v              v
        %id:      K  --------->  K

        Por el ejercicio $H2.15$, $|E:K| = |E(b):K|$ y $|K(a):K| = |K(b):K|$. Además $|E:K(a)| = |E(b):K(b)|$.
        Por tanto:
        $$
            |E:K| = |E:K(a)||K(a):K| = |E(b):K(b)||K(b):K| = |E(b):K| = |E(b):E||E:K| \text{ usando el teorema \ref{thm:transgr}.}
        $$
        Con lo que concluimos:
        $$
            |E(b):E| = 1 \implies E(b) = E \implies b\in E
        $$
    \end{itemize}
\end{proof}


\begin{ex}[H3.5]
    Decide si son normales las siguientes extensiones:
    \begin{enumerate}
        \item $\Q(\sqrt{5}i)/\Q$.\\\\

        Es normal ya que $\Q(\sqrt{5}i) = \Q(x^2+5)$ es el cuerpo de escisión de $x^2+5$.
        \item $\Q(\sqrt{5})/\Q$.\\\\

        Es normal ya que $\Q(\sqrt{5})/\Q = \Q(x^2-5)$ es el cuerpo de escisión de $x^2-5$.

        \item $\Q(\sqrt[4]{5}/\Q)$.\\\\

        No es normal. Sabemos que $\Q(\sqrt[4]{5})/\Q$ es finita, y como $p(x) = x^4 - 5$ es irreducible en $\Q[x]$, tiene una raíz en $\Q(\sqrt[4]{5})$ pero no escinde, por la caracterización de normalidad (teorema \ref{thm:3.9}), concluimos con que $\Q(\sqrt[4]{5})/\Q$ no es normal.
    \end{enumerate}
\end{ex}

\begin{ex}[H3.6]
    Demuestra que $\Q(\sqrt[3]{2})/\Q$ no es normal. Encuentra una extensión normal que contenga $\Q(\sqrt[3]{2})$ como subcuerpo.\\\\

    Sabemos que no es normal porque sea $p(x) = x^3 - 2 \in \Q[x]$, se puede comprobar que $p(x)$ es irreducible y tiene una raíz en $\Q(\sqrt[3]{2})$ pero no se escinde. Por tanto, por el teorema \ref{thm:3.9}, el cuerpo no es normal.\\
    Una extensión normal que contiene a $\Q(\sqrt[3]{2})$ puede ser $\Q(\sqrt[3]{2}, \omega)$ con $\omega \in \C,\ o(\omega) = 3$.
\end{ex}

\begin{obs}
    Si $E/K$ es normal y $f\in K[x]$ no necesariamente irreducible que tiene una raíz en $E$, $f$ no tiene por que escindirse.
\end{obs}
