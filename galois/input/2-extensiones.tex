% !TeX root = ../galois.tex

\chapter{Extensiones de cuerpos}

\section{Grados de cuerpos}

\begin{dfn}[Extensión]
    Sean $K, E$ cuerpos, decimos que $E$ es una \textbf{extensión} de $K$ (denotado por $E/K$) si $K$ es un subcuerpo de $E$.
\end{dfn}

\begin{eg}[Extensiones]$ $
    \begin{itemize}
        \item $\C/\Q$
        \item $\R/\Q$
        \item $\C/\R$
        \item $\Q[\sqrt{n}]/\Q$ con $n \neq$ de un cuadrado perfecto.
    \end{itemize}
\end{eg}

\begin{pro}[Extensión como espacio vectorial]
    Si $E$ es una extensión de $K$, entonces $E$ es un espacio vectorial sobre $K$.
\end{pro}
\begin{proof}
    Basta interpretar el producto por escalares $\cdot: K \times E \to E$ como la restricción del producto sobre $E \times E$ a $K$. La suma está bien definida por ser $E$ un grupo abeliano con la suma.
\end{proof}

\begin{dfn}[Grado de una extensión]
    Sea $E/K$ una extensión, el grado de la extensión es $|E:K| = \dim_K E$, que coincide con la dimensión del espacio vectorial que define $E$ sobre $K$.
\end{dfn}

\begin{dfn}[Extensión finita]
    Sea $E/K$ una extensión, es \textbf{finita} si y sólo si $\exists \setdef{a_1, \ldots, a_n} \subseteq E$ tales que forman una $K$-base. Es equivalente a decir que $\dim_K E = n < \infty$.
\end{dfn}

\begin{lm}[Extensión de grado 1]
    Sea $E/K$ una extensión:
    $$
        |E:K| = 1 \iff E = K
    $$
\end{lm}

\begin{proof}$ $
    \begin{itemize}
        \item[$\implies$] Si $|E:K| = \1$, entonces $\exists e \in E$ tal que $\setdef{e}$ es una $K$-base. Por tanto: $\1 = k \cdot e$ con $k \in K \implies e = k^{-1} \implies e \in K \implies E = K$.
        \item[$\ \Longleftarrow \ $] $\setdef{\1}$ es $K$-base de $K$. $\dim_K K = 1 = |K:K|$.
    \end{itemize}
\end{proof}

\begin{thm}[Transitividad de grados]\label{thm:transgr}\label{thm:2.1}% 2.1
    Sea una extensión $E/K$ y un subcuerpo $L$ intermedio $K \subseteq L \subseteq E$, entonces la extensión $E/K$ es finita si y sólo si $|E:L| < \infty,\ \land |L:K| < \infty$, y en tal caso: $|E:K| = |E:L|\cdot |L:K|$.
\end{thm}

\begin{proof}
    Supongamos $\dim_K E = r < \infty$, y $\setdef{e_1, \ldots, e_r}$ una $K$-base de $E$, entonces $\setdef{e_1, \ldots, e_r}$ es un $L$-sistema generador de $E \implies E/L$ es finita.\\
    Como $K \subseteq L \subseteq E$, $L$ es un $K$-subespacio vectorial de $E$, en particular $\dim_K L \leq dim_K E < \infty$. Si $E/L$ y $L/K$ son finitas, cogemos $\setdef{b_1, \ldots, b_m}$ una $L$-base de $E$ y $\setdef{a_1, \ldots, a_n}$ una $K$-base de $E$.\\
    Queremos ver que $\setdef{a_ib_j \mid 1 \leq i \leq n,\ 1 \leq j \leq m}$ es una $K$-base de $E$ (en particular con esto habremos probado que $|E:K| = |E:L|\cdot |L:K|$). Sabemos que:
    $$
        x \in E, x = \sum_{j=1}^m l_j b_j,\ l_j \in L
    $$
    pero además
    $$
        l_j = \sum_{i = 1}^{n} k_{ij} a_i,\ k_{ij} \in K \implies x = \sum_{1 \leq i \leq n,\ 1 \leq j \leq m} k_{ij}a_ib_j,\ k_{ij} \in K
    $$
    Faltaría ver que $\setdef{a_ib_j}$ es $K$-libre.
    $$
        \sum_{1 \leq n,\ 1 \leq i\leq j \leq m} k_{ij}a_ib_j = 0 \implies \sum_j l_j b_j = 0 \implies l_j = 0 \implies \sum_i k_{ij}a_i = 0 \implies k_{ij} = 0 \forall i \forall j
    $$
\end{proof}

\begin{dfn}[Menor subanillo y subcuerpo]Sea $E/K$ una extensión y sea $a \in E$.
    \begin{itemize}
        \item Denotaremos por $K[a]$ al \textbf{menor subanillo} de $E$ que contiene a $K$ y a $a \in E$. Se puede probar que $K[a] = \setdef{f(a) \forall f \in K[x]}$.
        \item Denotaremos por $K(a)$ al \textbf{menor subcuerpo} de $E$ que contiene a $K$ y a $a \in E$. Se puede probar que $K(a) = \setdef{\frac{f(a)}{g(a)} \forall f,g \in K[x] \mid g(a) \neq 0}$.
    \end{itemize}
\end{dfn}

\begin{eg}$ $
    \begin{itemize}
        \item $\Q[\sqrt{n}] = \setdef{a + \sqrt{n} \mid a, b\in \Q}$. En este caso $\Q[\sqrt{n}] = \Q(n)$.
        \item $X \subseteq E$, $K(X)$ es el menor subcuerpo de $E$ que contiene a $K$ y a $X$. $K(X)$ se obtiene al \textit{adjuntar} $X$ a $K$.
    \end{itemize}
\end{eg}

\begin{obs}
    En general $K[a] \subseteq K(a)$, pero hay en casos en los que la igualdad no se cumple.
\end{obs}

\begin{dfn}[Extensión simple]
    Sea $E/K$ una extensión, es \textbf{simple} si $\exists a \in E$ tal que $E = K(a)$.
\end{dfn}

\begin{eg}[Extensión simple]$ $
    \begin{itemize}
        \item $\C/\R$, ya que $\C = \R(i)$.
        \item Con $p\neq q$ primos, $\Q(\sqrt{p}, \sqrt{q})$ es simple, ya que se puede demostrar que $\Q(\sqrt{p}, \sqrt{q}) = \Q(\sqrt{p} + \sqrt{q})$.
    \end{itemize}
\end{eg}

\begin{pro}[Dimensión de un cuerpo finito]
    Sea $K$ un cuerpo finito, entonces $|K| = p^n$ con $p$ primo.
\end{pro}

\begin{proof}
    Sea $K$ un cuerpo y $F$ su cuerpo primo, sabemos que $F$ es isomorfo a algún $\F_p$ con $p$ un primo. Además, $K/F$ es una extensión. Y como $K$ es subespacio vectorial $|K:F| = \dim_F K = n$. Entonces $K \isom F^n \implies |K| = |F|^n = p^n$.
\end{proof}

\section{Extensiones algebraicas y trascendentes}

\begin{dfn}[Extensión algebraica. Extensión trascendente]
    Sea $E/K$ una extensión.
    \begin{itemize}
        \item Sea $a \in E$, $a$ es \textit{algebraico} si $\exists f(x) \neq 0 \in K[x]:\ f(a) = 0$. $E/K$ es una \textbf{extensión algebraica} si todo $a \in E$ es algebraico sobre $E$.
        \item Sea $a \in E$, $a$ es \textit{trascendente} si no es algebraico. $E/K$ es una \textbf{extensión trascendente} si existe $a \in E$ trascendente sobre $E$.
    \end{itemize}
\end{dfn}

\begin{eg}[Extensiones algebraicas y trascendentes]$ $
    \begin{itemize}
        \item $K/K$ es algebraica. Todo elemento de $K$ es raíz de $x - k \in K[x]$.
        \item $\Q(\sqrt{n})/\Q$ es algebraica.
        \item $\R/\Q$ es trascendente. $e$ y $\pi$ son trascendentes.
        \item Sea $K[t]$ un dominio de integridad- Podemos construir su cuerpo de fracciones:
        $$
            K(t) = \setdef{\frac{f(t)}{g(t)} \mid f,g \in K[t],\ g(t) \neq 0}
        $$
        Entonces $K(t)/K$ es trascendente. ($t$ siempre es trascendente).
    \end{itemize}
\end{eg}

\begin{pro}[Extensiones y cuerpos intermedios]
    Sea $E/K$ una extensión y $K \subseteq L\subseteq E$ un cuerpo intermedio:
    \begin{enumerate}
        \item $E/K$ es algebraica $\iff L/K$ y $E/L$ son algebraicas.
        \item Si $E/L$ es trascendente, entonces $E/K$ es trascendente.
    \end{enumerate}
\end{pro}

\begin{proof}
    Se deja como ejercicio.
\end{proof}

\begin{thm}[Extensiones finitas y algebraicas]\label{thm:finitaalgebraica}\label{thm:2.2} %2.2
    Toda extensión finita es algebraica.
\end{thm}
\begin{proof}
    Sea $E/K$ una extensión, $a \in E$, queremos ver que es raíz de $0 \neq f(x) \in K[x]$. Suponemos $|E : K| = n$ con un $K$-sistema: $\setdef{1, a_1, \ldots, a^{n-1}} \subseteq E$ con $n$ elementos. Entonces, el sistema puede ser:
    \begin{itemize}
        \item [$K$-ligado] Existen $k_i \in K$ no todos nulos tales que $k_0 + k_1a + \ldots + k_{n-1}a^{n-1} = 0$, entonces $a$ es raíz de $f(x) = k_0 + \ldots + k_{n-1}x^{n-1}$.
        \item [$K$-libre] Como $\dim_K E = n$, $\setdef{1, a_1, \ldots, a^{n-1}, a^n}$ es $K$-ligado, y de nuevo, $a$ es raíz de $f(x) = k_0 + \ldots + k_{n-1}x^{n-1} + k_n x^n$.
    \end{itemize}
\end{proof}

\section{Teorema del elemento algebraico}

\begin{thm}[Teorema del elemento algebraico]\label{thm:elemalg}\label{thm:2.3} %2.3
    Sea $E/K$ una extensión, $a \in E$ un elemento algebraico sobre $K$.
    \begin{enumerate}
        \item Existe un único polinomio irreducible mónico $p \in K[x]$ tal que $p(a) = 0$.
        \item Si $q \in K[x]$ y $q(a) = 0$, entonces $p \divides q$.
        \item $K(a) = \setdef{f(a) \mid f \in K[x]} = K[a]$.
        \item Si $\delta(p) = n$, entonces $\setdef{1, a, \ldots, a^{n-1}}$ es una $K$-base de $K(a)$. En particular, $|K(a) : K| = \delta(p)$ y $K(a) = \setdef{k_0 + k_1 a +\ldots + k_{n-1} a^{n-1} \mid k_i \in K}$.
    \end{enumerate}
\end{thm}

\begin{proof}
    ($\cdots$) (Es muy tarde ahora mismo para pasar esto a limpio, que alguien me mate por favor.) % TODO
\end{proof}

\begin{dfn}[Polinomio mínimo]
    Sea $E/K$ una extensión y $a\in E$ algebraico sobre $K$, al único polinomio mónico e irreducible $p \in K[x]$ dado por el teorema \ref{thm:elemalg} se le llama \textbf{polinomio mínimo} o \textbf{polinomio irreducible} de $a$ sobre $K$ y escribimos $p = Irr(K, a)$.
\end{dfn}

\begin{obs}
    Sea $b = k_0 + \ldots + k_{n-1} a^{n-1} \in K(a)$, ¿cómo se expresa $b^{-1}$ en la misma base?\\
    Consideramos $f(x) = k_0 + \ldots + k_{n-1}x^{n-1} \in K[x]$, con $f(a) = b \neq 0$ y $\delta(f) \leq \delta(p)$, siendo $p$ el polinomio irreducible. Entonces $mcd(f, g) = 1$. Por la identidad de Bezout:
    $$
        \exists h, g \in K[x]:\ 1 = fh + gp \implies \text{ (evaluando en $a$)} 1 = f(a)h(a) + g(a)p(a) \implies 1 = f(a)h(a) = b h(a) \implies b^{-1} = h(a)
    $$
\end{obs}

\begin{eg}[Ejercicio tipo]
    ($\cdots$) (Es aún más tarde. \url{https://www.youtube.com/watch?v=I_6Gej1m4SU}) % TODO
\end{eg}

\begin{thm}[Extensión por varios elementos algebraicos]\label{thm:extalg}\label{thm:2.4} %2.4
    Sea $E/K$ una extensión, y sea $\mbf{a} = (a_1, \ldots, a_n)$ con $a_i \in E$ algebraicos sobre $K$, entonces $K(\mbf{a})/K$ es finita. En particular, $K(\mbf{a})/K$ es algebraica.
\end{thm}

\begin{proof}
    Por inducción sobre $n$. Si $n=1$ por el teorema del elemento algebraico (teorema \ref{thm:elemalg}) sabemos que $|K(a_1)/K| = \delta(Irr(K, a_1)) < \infty$.\\
    Veamos el caso $n>1$. Sea $L = K(a_1, \ldots, a_{n-1})$, entonces $K(\mbf{a}) = L(a_n)$. Por la hipótesis de inducción $L/K$ es finita, por el teorema \ref{thm:elemalg} $L(a_n)/L$ es finita y por tanto, por el teorema de transitividad de grados (teorema \ref{thm:transgr}) $L(a_n)/K$ es finita, donde $L(a_n)$ era $K(\mbf{a})$. La segunda parte se sigue directamente aplicando el teorema \ref{thm:finitaalgebraica}.
\end{proof}

\begin{ex}[H2.7]
    Dada $E/K$ una extensión, prueba que $L = \setdef{e \in E \mid e \text{ es algebraico sobre } K} \supseteq K $ es un cuerpo. Sea $\A \subset \C$ los elementos algebraicos sobre $\Q$, prueba que $\A/\Q$ es una extensión de grado infinito\\
    Sean $a, b \in L$ cualesquiera, entonces $a, b \in K(a, b)^\times$ y por el teorema \ref{thm:extalg} $K(a,b)/K$ es algebraica. Como $a \pm b$ y $ab^{\pm 1} \in K(a, b)$, entonces son algebraicos sobre $K \implies (a \pm b), (ab^{\pm 1}) \in L$, con lo que es cerrado por ambas operaciones y $L$ es un cuerpo.\\
    Por la primera parte del ejercicio, sabemos que $\Q \subseteq \A \subseteq \C$ es un subcuerpo intermedio, de $\C/\Q$, y por definición $\A/\Q$ es algebraica.\\
    Por el criterio de Einsestein (teorema \ref{thm:Einsestein}), para cada $n \in \N^\times$, $x^n - 2$ es irreducible en $\Q[x]$. $\sqrt[\leftroot{-3}\uproot{3}n]{2}$ es solución por tanto es algebraico y por el teorema \ref{thm:elemalg} $|\Q(\sqrt[\leftroot{-3}\uproot{3}n]{2})/\Q|=n$.\\
    Como podemos hacerlo $\forall n \in \N^\times$, hemos comprobado que $\A/\Q$ es no finita.
\end{ex}

\begin{ex}[H2.6]
    ($\cdots$) (Yep. \url{https://www.youtube.com/watch?v=pwSsT8IU0WE})
\end{ex}

\section{Isomorfismos de cuerpos}

Ya definimos qué es un homomorfismo de cuerpos en la sección \ref{section:homomorfismos}. En esta sección vamos a ampliar los conocimientos cuando aplicamos los homomorfismos específicamente a cuerpos.

\begin{obs}
    Si $\car{E} \neq \car{K}$ entonces: $Hom(E, K) = \varnothing = Hom(K, E)$, con $Hom(X, Y) = \setdef{\varphi: X \to Y \mid \varphi \text{ es un homomorfismo de cuerpos}}$. Además, si $K$ es finito, $End(K) = Aut(K)$ (conjunto de endomorfismos y automorfismos respectivamente), pero en general: $End(K) \subsetneq Aut(K)$. %TODO: ¿ esta bien la dirección ?
\end{obs}

\begin{obs}
    Sea $\varphi \in End(K)$ y $F$ es el cuerpo primo de $K$, entonces $\varphi(a) = a,\ \forall a \in F$. De forma más general, si $\varphi \in Hom(E, K)$ y $E, K$ tienen el mismo cuerpo primo, $\varphi(a) = a$, por ejemplo:
    $Aut(\F_p) = \setdef{id}$, $Aut(\Q) = \setdef{id}$.
\end{obs}

En ocasiones querremos saber como extender un isomorfismo de cuerpos a una extensión de dichos cuerpos. Vamos a ver un lema y un teorema.

\begin{lm}[Restricción de un isomorfismo de cuerpos]
    Sea $E_1/K_1$ una extensión, $\theta: E_1 \to E_2$ un isomorfismo de cuerpos, y sea $K_2 = \theta(K_1)$, entonces:
    \begin{enumerate}
        \item $E_2/K_2$ es una extensión y $|E_1:K_1| = |E_2:K_2|$.
        \item Sean $a_1, \ldots, a_n \in E_1$. $\theta(K_1(a_1, \ldots, a_n)) = K_2(\theta(a_1), \ldots, \theta(a_n))$.
        \item $\theta$ se extiende a un isomorfismo de anillos $\theta: K_1[x] \to K_2[x]$, aplicando $\theta$ individualmente a cada coeficiente del polinomio.
    \end{enumerate}
\end{lm}

\begin{proof}
    Como ejercicio.
\end{proof}

\begin{thm}[Extensión de un isomorfismo de cuerpos]\label{thm:2.5} % 2.5
    Sean $E_1/K_1$, $E_2/K_2$ extensiones, $\sigma: K_1 \to K_2$ un isomorfismo de cuerpos, $p_1$ irreducible en $K_1$, $p_2$ irreducible en $K_2$, y $a_1$ y $a_2$ raíces de los polinomios respectivamente, entonces:
    $$
        \sigma \text{ se extiende a } \theta \iff \left.\theta\right|_{K_1} = \sigma
    $$
    Donde $\theta: K_1(a_1) \to K_2(a_2)$ tal que $\theta(a_1) = a_2$.
\end{thm}

\begin{proof}
    A completar.
\end{proof}

\begin{cor}\label{cor:2.6}\label{cor:2.7} % 2.6, 2.7
    Sea $E/K$ una extensión, $p \in K[x]$ irreducible, entonces:\\

    $$a, b \in E \text{ son raíces de } p \iff \exists \text{ un isomorfismo } \theta: K(a) \to K(b) \text{ tal que } \theta(a) = b \text{ y } \theta(k) = k\ \forall k \in K$$
\end{cor}

\begin{ex}[H2.4 (parte)]
    Halla el grado y base de las siguientes extensiones de cuerpos:
    \begin{enumerate}
        \item $\Q(\sqrt{2}, \sqrt{3}, i)/\Q$.\\
        Consideramos $L = \Q(\sqrt{2}, \sqrt{3}) = \Q(\sqrt{2} + \sqrt{3})$ cuerpo intermedio, y además, $\Q(\sqrt{2}, \sqrt{3}, i) = L(i)$. $i$ es raíz del polinomio $x^2 + 1$ qque no tiene raíces en $L$ y por tanto, $x^2 + 1 = Irr(L, i)$. Por el teorema \ref{thm:elemalg}, $|\Q(\sqrt{2}, \sqrt{3}, i):\Q| = 2 = \delta(Irr(L, i))$.\\

        Además, $|L:\Q| = 4$ por el ejercicio 1 de la hoja 2. Por el teorema \ref{thm:transgr}, $|\Q(\sqrt{2}, \sqrt{3}, i) : \Q| = |L(i):L|\cdot|L:\Q| = 8$. Una vez sabemos el grado, podemos encontrar una $\Q$-base:
        $$
            \setdef{1, \sqrt{2}, \sqrt{3}, \sqrt{6}, \sqrt{2} i, \sqrt{3} i, \sqrt{6} i}
        $$
        $$
            \setdef{1, \alpha, \alpha^2, \alpha^3, i, \alpha i, \alpha^2 i, \alpha^3 i}
        $$

        \item $\Q(\sqrt[4]{2})/\Q(\sqrt{2})$.\\
        $\alpha = \sqrt[4]{2}$ es raíz de $x^2 - \sqrt{2} \in \Q(\sqrt{2})[x]$, y es irreducible porque no tiene raíces en $Q(\sqrt{2})$, (se prueba por reducción al absurdo).\\
        Por tanto: $|\Q(\sqrt[4]{2}):\Q(\sqrt{2})| = 2$ y una base:
        $$
            \setdef{1, \sqrt[4]{2}}
        $$

        \item $\Q(\sqrt{1+\sqrt 3})/\Q$.\\
        Consideramos el cuerpo intermedio $L = \Q(\sqrt{3})$. Es fácil ver que $|L/\Q| = 2$. Falta encontrar el grado de $|\Q(\sqrt{1+\sqrt 3})/L|$. Sea $\alpha = \sqrt{1 + \sqrt 3}$, $\alpha$ es raíz de $x^2 - (1+\sqrt 3)$, que se puede demostrar que es irreducible en $L$. Por tanto, por el teorema \ref{thm:transgr}, $|\Q(\sqrt{1+\sqrt 3}):\Q| = 4$, y la base:
        $$
            \setdef{1, \sqrt 3, \sqrt{1 + \sqrt 3}, \sqrt 3 \sqrt{1 + \sqrt 3}}
        $$
    \end{enumerate}
\end{ex}

\begin{ex}[H2.5]
    Halla grado y base de $\F_7(t) / \F_7(t^2)$. Halla la expresión de $t^{-1}$ y $(t+1)^{-1}$ en la base que has hallado.\\\\

    Consideramos en polinomio $x^2 - t^2 \in F(t^2)[x]$, donde $t$ es una raíz y $\pm t \notin \F_7(t^2)$. Se puede demostrar por reducción al absurdo que el polinomio $x^2 - t^2$ es irreducible. Por tanto, por el teorema \ref{thm:elemalg}, $|\F_7(t) / \F_7(t^2)| = 2$.\\
    Una $\F_7(t^2)$-base de $\F_7(t)$ es: $\setdef{1, t}$.\\

    Vamos a expresar ahora los elementos que se nos piden. Consideramos $t = f(t)$, $f(x) = x$.
    $$
        x^2 - t^2 = 0,\ x \cdot x = t^2 \implies x \cdot \frac{1}{t^2} \cdot x = 1\text{, que evaluando en $t$: } t \cdot \left(\frac{1}{t^2} \cdot t\right) = 1
    $$
    Con ello, hemos hallado el inverso de $t$. Para hallar el inverso de $t+1$ procedemos de forma parecida. Consideramos $f(x) = x + 1$.
    Vemos que $mcd(f(x), x^2 - t^2) = 1$. Procediendo con el algoritmo de división de polinomios, podemos expresar:
    $$
        x^2 - t^2 = f(x) (x - 1) + (1 - t^2) \implies (x^2 - t^2) + f(x)(1-x) = 1-t^2 \in \F_7(t^2)
    $$
    Entonces:
    $$
        \frac{1}{1 - t^2} (x^2 - t^2) + f(x) \frac{1-x}{1-t^2} = 1
    $$
    Evaluando en $t$:
    $$
        f(t)\frac{1-t}{1-t^2} = 1 \implies (t+1)^{-1} = \frac{1}{1-t^2} \cdot 1 + \frac{1}{t^2 - 1} t
    $$
\end{ex}

\begin{ex}[H2.10]
    Sea $E/K$ una extensión, $\alpha \in E$ algebraico sobre $K$ y $L$ un subcuerpo intermedio. Prueba que $q(x) = Irr(L, \alpha) \divides Irr(K, \alpha) = p(x)$.\\\\

    $p(x) \in K[x] \subseteq L[x]$, entonces $p(x)$ tambien es un polinomio de $L[x]$. Como $p(\alpha) = 0$, por el teorema del elemento algebraico, (teorema \ref{thm:elemalg}) $q(x) \divides p(x)$. Y entonces:
    $$
        |L(\alpha):L| = \delta(q(x)) \leq \delta(p(x)) = |K(\alpha):K|
    $$
\end{ex}

\begin{ex}[H2.11]
    Sea $E/K$ una extensión. Demuestra:
    \begin{enumerate}
        \item Si $|E/K| = p$ con $p$ primo, demuestra que no hay subcuerpos intermedios.\\\\
        Por el teorema de transitividad de grados (teorema \ref{thm:transgr}), sabemos que $|E:K| = |E:L| \cdot |L:K| = p \implies |E:L| = 1 \lor |L:K| = 1$, y por tanto $E = L \lor L = K$.

        \item Sea $|E:K| = p$ con $p$ primo, entonces $E/K$ es simple.\\\\
        Consideramos $L = K(\alpha)$ con $\alpha \in E, \alpha \not in K$, además $K(\alpha) \neq K$ y por tanto (por el primer apartado) $L = E = K(\alpha)$ y es simple ya que $K(\alpha)$ lo es.

        \item Supongamos $\alpha \in E$ tal que $Irr(K, \alpha) = x^3 + x - 1$. Queremos calcular $Irr(K, \alpha^2)$. Como sabemos que $K(\alpha)/K$ tiene grado $3$, y por el apartado siguiente, $K(\alpha) = K(\alpha^2)$, entonces por el teorema \ref{thm:elemalg}, $\delta(Irr(K, \alpha^2))=2$.\\\\
        Entonces:
        $$
            \alpha^3 + \alpha^2 = 1 \iff \alpha^6 + 2\alpha^4 + \alpha^2 - 1 = 0
        $$
        Sea $\beta = \alpha^2$, entonces se satisface que:
        $$
            \beta^3 + 2\beta^2 + \beta - 1 = 0
        $$
        y con ello hemos hallado el polinomio irreducible que buscábamos.

        \item Si $\alpha \in E$, con $|K(\alpha):K| = n$ impar, calcula $|K(\alpha^2):K|$.\\\\
        Como $n$ es impar sabemos que $\alpha^2 \in K(\alpha)$ y por tanto $K(\alpha^2) = K(\alpha)$.\\

        \item Sea $K \subseteq L_1, L_2 \subseteq E$ dos cuerpos intermedios de grado coprimos sobre $K$ demuestra que $L_1 \cap L_2 = K$.\\\\
        Se considera el cuerpo $L_1 \cap L_2$. Sea $d = |L_1 \cap L_2 / K$, $n = |L_1/K|$ y $m=|L_2/K$. Entonces por el teorema \ref{thm:transgr}, $d \divides n$ y $d \divides m$ con $n, m$ coprimos, por tanto $d = 1$ y $\L_1 \cap L_2 = K$.
    \end{enumerate}
\end{ex}

\begin{ex}[H2.12]
    Sea $E/K$ una extensión, y sean $a, b \in E$ algebraicos con $|K(a):K| = n$, $|K(b):K| = m$. Prueba que:

    \begin{enumerate}
        \item $|K(a, b) : K(b) \leq n$.\\\\
        Mirar el ejercicio H2.10

        \item Sean $n$ y $m$ son coprimos, entonces $K(a)\cup K(b) = K$ y $|K(a, b):K|=nm$. Deduce que $Irr(K, a) = Irr(K(b), a)$.\\\\
        Mirar ejercicio H2.11\\
        \item Calcula $Irr(\Q, \alpha)$ con $\alpha = \sqrt{3} + \sqrt[3]{2}$
    \end{enumerate}
\end{ex}
