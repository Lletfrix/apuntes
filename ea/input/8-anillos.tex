% !TeX root = ../apuntes-ea.tex

\chapter{Anillos}


\begin{dfn}[Anillo]
	Un anillo es una terna $(A, +, \cdot)$ donde $+$ es una operación a la que llamamos suma, $\cdot$ es otra operación a la que llamamos producto y se verifican las siguientes propiedades
	\begin{enumerate}
		\item El par $(A, +)$ es un grupo abeliano
		\item El producto $\cdot$ es asociativo
		\item Se cumplen las propiedades distributivas:
		\begin{align}
			\forall a, b , c \in A,\ a\cdot (b + c) = a\cdot b + a \cdot c \\
			\forall a, b , c \in A,\ (a + b) \cdot c = a\cdot c + b \cdot c
		\end{align}
	\end{enumerate}
\end{dfn}

Con la operación $+$ tenemos las siguientes propiedades
\begin{enumerate}
	\item Asociatividad: $(a+b)+c = a+(b+c)$
	\item Elemento neutro aditivo: $\exists 0 \in A \mid 0+a = a$
	\item Elemento inverso aditivo: $\forall a \in A, \exists -a \in A \mid a + (-a) = 0$
	\item Conmutatividad aditiva: $\forall a, b \in A,\ a+b = b+a$
\end{enumerate}

Con la operación $\cdot$ tenemos las siguientes propiedades
\begin{enumerate}
	\item Asociatividad: $a\cdot (b \cdot c) = (a \cdot b) \cdot c$
	\item Elemento neutro multiplicativo: $\exists 1 \in A \mid a\cdot 1 = 1 \cdot a = a$
	\item No siempre existe inverso multiplicativo: $\inv{a} \mid a\cdot \inv{a} = 1$
	\item No siembre se da la conmutatividad multiplicativa: $a \cdot b = b\cdot a$
\end{enumerate}

\begin{dfn}[Unidades en anillos]
	Dado $(A, +, \cdot)$ anillo. El grupo de unidades es
	\begin{align}
		\uds{A} = (\{a \in A \mid \exists \inv{a} \in A,\ a\cdot \inv{a} = 1\}, \cdot)
	\end{align}
	Los elementos del grupo de unidades se llaman elementos invertibles.
\end{dfn}

\begin{ej}
	Las matrices cuadradas $2\times 2$ con coeficientes reales: $(M_{2\times 2}(\R), +, \cdot)$ es un anillo. Tiene unidades $\uds{A} = (GL_2(\R), \cdot)$
\end{ej}

\begin{ej}
	Los numeros enteros $(\Z, +, \cdot)$ es un anillo y tienen unidades $\uds{\Z} = (\{-1, 1\}, \cdot)$
\end{ej}

\begin{pro}
	Sea $-1$ el inverso aditivo del neutro multiplicativo $1$. Entonces $\forall a \in A$ el inverso aditivo es $-a = -1 \cdot a$ y se tiene $-1\cdot a + a = 0$.
\end{pro}

\begin{pro}
	Sea $A$ un anillo. El neutro aditivo $0$ verifica $0 \not\in \uds{A}$
\end{pro}

\begin{dfn}[Anillo conmutativo]
	Sea $A$ un anillo. $A$ es un anillo conmutativo $\iff \forall a, b \in A,\ a\cdot b = b \cdot a$.
\end{dfn}

\begin{pro}[Propiedad cancelativa]
	Sea $a \in \uds{A}$. Entonces $\forall b,c$ se tiene $b, c \in A \implies a\cdot b = a\cdot c \implies b = c$
\end{pro}

\begin{dfn}[Divisor de 0]
	Sea $(A, +, \cdot)$ un anillo. Diremos que $a \in A$ es divisor de $0 \iff a \neq 0 \land \exists 0 \neq b \in A \mid a\cdot b = 0$
\end{dfn}

\begin{ej}
	En $\Z/8\Z$ el elemento $\overline{2}$ tiene dimensión 0.
\end{ej}

\begin{pro}
	Sea $A$ un anillo. $\forall a \in A$ no divisor de 0 $\implies$ se cumple la propiedad cancelativa.
\end{pro}

\begin{proof}
	$ab = ac \implies b = c \iff ab + (-ac) = a(b -c) = 0$
\end{proof}

\begin{dfn}[Dominio de integridad]
	Un anillo que no tiene elementos divisores de 0 se llama dominio de integridad (DI).
\end{dfn}

\begin{ej}
	\begin{itemize}
		\item $\Z$ es un dominio de integridad ya que todo $a \in \Z, a \neq 0$ tiene un inverso multiplicativo $\inv{a}$.
		\item $\Z/p\Z$ con $p$ primo es un dominio de integridad.
		\item $\Z/n\Z$ con $n$ no primo no es un dominio de integridad ya que si $\overline n = ab$ con $a \neq n \land b \neq n$ se tiene $\overline{a} \cdot \overline{b} = \overline{n} = \overline{0}$ con $\overline{a} \neq 0 \land \overline{b} \neq 0$.
	\end{itemize}
\end{ej}

% -------------

\begin{thm}
	Dado el anillo $A$ y un ideal propio $I$
	\begin{align*}
		\pi: A \to A/I,\qquad I \subset \inv{\pi}(\overline{J}) \subset A,\qquad \overline{0} \in \overline{J} \subset A / I
	\end{align*}
	
	existe una identificación entre el retículo de ideales $A / I$ con el subretículo de ideales de $A$ que contienen a $I$. 
	
	Es decir, si $J$ es un ideale en $A/I$ entonces $\inv{\pi}(\overline{J})$ es un ideal en $A$ que contiene al ideal $I$.
\end{thm}


El ideal cero de $A/I$ tiene contraimagen $\inv{\pi}(\{0\}) = I$. Si $\overline{J}$ es un ideal en $A/I$
\begin{align*}
	\pi : A \to A/I \to (A/I) / \overline{J}
\end{align*}

es un homomorfismo de anillos (la composición de homomorfismos de anillos es un homomorfismo de anillos). $\inv{\pi}(\overline{J}) = \ker$ de la composición.