% !TeX root = ../apuntes-ea.tex

\chapter{Anillos}


\begin{dfn}[Anillo]
	Un anillo es una terna $(A, +, \cdot)$ donde $+$ es una operación a la que llamamos suma, $\cdot$ es otra operación a la que llamamos producto y se verifican las siguientes propiedades
	\begin{enumerate}
		\item El par $(A, +)$ es un grupo abeliano
		\item El producto $\cdot$ es asociativo
		\item Se cumplen las propiedades distributivas:
		\begin{align}
			\forall a, b , c \in A,\ a\cdot (b + c) = a\cdot b + a \cdot c \\
			\forall a, b , c \in A,\ (a + b) \cdot c = a\cdot c + b \cdot c
		\end{align}
	\end{enumerate}
\end{dfn}

Con la operación $+$ tenemos las siguientes propiedades
\begin{enumerate}
	\item Asociatividad: $(a+b)+c = a+(b+c)$
	\item Elemento neutro aditivo: $\exists 0 \in A \mid 0+a = a$
	\item Elemento inverso aditivo: $\forall a \in A, \exists -a \in A \mid a + (-a) = 0$
	\item Conmutatividad aditiva: $\forall a, b \in A,\ a+b = b+a$
\end{enumerate}

Con la operación $\cdot$ tenemos las siguientes propiedades
\begin{enumerate}
	\item Asociatividad: $a\cdot (b \cdot c) = (a \cdot b) \cdot c$
	\item Elemento neutro multiplicativo: $\exists 1 \in A \mid a\cdot 1 = 1 \cdot a = a$
	\item No siempre existe inverso multiplicativo: $\inv{a} \mid a\cdot \inv{a} = 1$
	\item No siembre se da la conmutatividad multiplicativa: $a \cdot b = b\cdot a$
\end{enumerate}

\begin{dfn}[Unidades en anillos]
	Dado $(A, +, \cdot)$ anillo. El grupo de unidades es
	\begin{align}
		\uds{A} = (\{a \in A \mid \exists \inv{a} \in A,\ a\cdot \inv{a} = 1\}, \cdot)
	\end{align}
	Los elementos del grupo de unidades se llaman elementos invertibles.
\end{dfn}

\begin{ej}
	Las matrices cuadradas $2\times 2$ con coeficientes reales: $(M_{2\times 2}(\R), +, \cdot)$ es un anillo. Tiene unidades $\uds{A} = (GL_2(\R), \cdot)$
\end{ej}

\begin{ej}
	Los numeros enteros $(\Z, +, \cdot)$ es un anillo y tienen unidades $\uds{\Z} = (\{-1, 1\}, \cdot)$
\end{ej}

\begin{pro}
	Sea $-1$ el inverso aditivo del neutro multiplicativo $1$. Entonces $\forall a \in A$ el inverso aditivo es $-a = -1 \cdot a$ y se tiene $-1\cdot a + a = 0$.
\end{pro}

\begin{pro}
	Sea $A$ un anillo. El neutro aditivo $0$ verifica $0 \not\in \uds{A}$
\end{pro}

\begin{dfn}[Anillo conmutativo]
	Sea $A$ un anillo. $A$ es un anillo conmutativo $\iff \forall a, b \in A,\ a\cdot b = b \cdot a$.
\end{dfn}

\begin{pro}[Propiedad cancelativa]
	Sea $a \in \uds{A}$. Entonces $\forall b,c$ se tiene $b, c \in A \implies a\cdot b = a\cdot c \implies b = c$
\end{pro}

\begin{dfn}[Divisor de 0]
	Sea $(A, +, \cdot)$ un anillo. Diremos que $a \in A$ es divisor de $0 \iff a \neq 0 \land \exists 0 \neq b \in A \mid a\cdot b = 0$
\end{dfn}

\begin{ej}
	En $\Z/8\Z$ el elemento $\overline{2}$ tiene dimensión 0.
\end{ej}

\begin{pro}
	Sea $A$ un anillo. $\forall a \in A$ no divisor de 0 $\implies$ se cumple la propiedad cancelativa.
\end{pro}

\begin{proof}
	$ab = ac \implies b = c \iff ab + (-ac) = a(b -c) = 0$
\end{proof}

\begin{dfn}[Dominio de integridad]
	Un anillo que no tiene elementos divisores de 0 se llama dominio de integridad (DI).
\end{dfn}

\begin{ej}
	\begin{itemize}
		\item $\Z$ es un dominio de integridad ya que todo $a \in \Z, a \neq 0$ tiene un inverso multiplicativo $\inv{a}$.
		\item $\Z/p\Z$ con $p$ primo es un dominio de integridad.
		\item $\Z/n\Z$ con $n$ no primo no es un dominio de integridad ya que si $\overline n = ab$ con $a \neq n \land b \neq n$ se tiene $\overline{a} \cdot \overline{b} = \overline{n} = \overline{0}$ con $\overline{a} \neq 0 \land \overline{b} \neq 0$.
	\end{itemize}
\end{ej}

% -------------

\begin{thm}
	Dado el anillo $A$ y un ideal propio $I$
	\begin{align*}
		\pi: A \to A/I,\qquad I \subset \inv{\pi}(\overline{J}) \subset A,\qquad \overline{0} \in \overline{J} \subset A / I
	\end{align*}
	
	existe una identificación entre el retículo de ideales $A / I$ con el subretículo de ideales de $A$ que contienen a $I$. 
	
	Es decir, si $J$ es un ideale en $A/I$ entonces $\inv{\pi}(\overline{J})$ es un ideal en $A$ que contiene al ideal $I$.
\end{thm}


El ideal cero de $A/I$ tiene contraimagen $\inv{\pi}(\{0\}) = I$. Si $\overline{J}$ es un ideal en $A/I$
\begin{align*}
	\pi : A \to A/I \to (A/I) / \overline{J}
\end{align*}

es un homomorfismo de anillos (la composición de homomorfismos de anillos es un homomorfismo de anillos). $\inv{\pi}(\overline{J}) = \ker$ de la composición.

% --------------------

\begin{thm}
	Sea $\alpha: A \to B$ un homomorfismo de anillos.
	\begin{itemize}
		\item $\ker \alpha$ es un ideal
		\item $\ima \alpha$ es un subanillo
		\item $\alpha$ es sobreyectivo $\iff \ima \alpha = B$
		\item $\alpha$ es inyectivo $\iff \ker \alpha = \{0\}$
	\end{itemize}
\end{thm}

\begin{dfn}
	Un homomorfismo de anillos $\alpha: A \to B$ es un isomorfismo cuando es una biyección. En este caso decimos que $A$ y $B$ son isomorfos y lo notamos con $A \isom B$.
\end{dfn}

\begin{pro}
	Si $\alpha: A \to B$ es un homomorfismo de anillos y una biyección de conjuntos entonces $\inv{\alpha}:B \to A$ es nuevamente un homomorfismo de anillos.
\end{pro}

\subsubsection{Homomorfismos de anillos e ideales}

\begin{thm}
	Sea $\alpha: A \to B$ un homorfismo de anillos. Entonces
	\begin{enumerate}
		\item Si $J \subset B$ es un ideal en $B$ entonces $\inv{\alpha}(J)$ es un ideal en $A$.
		\item Si $\alpha$ es sobreyectiva entonces la imagen $\alpha(I)$ de un ideal $I \subset A$ es un ideal en $B$
	\end{enumerate}
\end{thm}

\begin{proof}
	\begin{figure}[h]
		\centering
		\begin{tikzpicture}
		\node (A) at (0,0) {$A$};
		\node (B) at (4,0) {$B$};
		\node (BJ) at (4, -3) {$B / J$};
		
		\draw[-{Latex[length=2mm]}] (A) -- (B) node[pos=.5, above] {$\alpha$};
		\draw[-{Latex[length=2mm]}] (B) -- (BJ) node[pos=.5, left]{$\pi$};
		\draw[-{Latex[length=2mm]}] (A) -- (BJ) node[pos=.5, below] {$\pi \circ \alpha$};
		\end{tikzpicture}
	\end{figure}
	\begin{enumerate}
		\item $\inv{\alpha}(J) = \ker(\pi \circ \alpha)$ y por tanto es un ideal.
		\item Probamos las propiedades de los ideales:
		\begin{enumerate}
			\item $\alpha(0) = 0 \in \alpha(I)$
			\item Sean $b_1, b_2 \in \alpha(I)$ tenemos que ver que $b_1 + b_2 \in \alpha(I)$. Sean $a_1, a_2 \in I$ tales que $b_1 = \alpha(a_1) \land b_2 = \alpha(a_2)$. Por ser $\alpha$ h. de anillos tenemos que $b_1 + b_2 = \alpha(a_1 + a_2) = \alpha(a_1) + \alpha(a_2)$.
			\item Sean $b \in B,\ b' \in \alpha(I)$. Tenemos que probar que $bb' \in \alpha(I)$. Sabemos que $b' \in \alpha(I) \iff b' = \alpha(a),\ a \in I$. Como $b \in B$ y $\alpha$ es sobre tiene que existir $d \in I \mid \alpha(d) = b$. Por tanto $\alpha(d\cdot a) = b \cdot b' \implies bb' \in \alpha(I)$.
		\end{enumerate}
	\end{enumerate}
\end{proof}

Fijado $I \subset A$ consideramos $\pi:A \to A/I$ que es un homomorfismo de anillos sobreyectivo.

\begin{enumerate}
	\item Si $\overline{J} \subset A /I$ es un ideal en $A / I$ entonces $\inv{\pi}(\overline{J})$ es un ideal en $A$ que contiene a $I$.
	\item Si $J$ es un ideal en $A$ entonces $\pi(J)$ es un ideal en $A/J$ y $J \subseteq \inv{\pi}(\pi(J))$ (es claro porque si $j \in J$ entonces $\pi(j) \in \pi(J)$).
	\begin{enumerate}
		\item Además, si $I \subseteq J$ entonces $J = \inv{\pi}(\pi(J))$.
		\begin{proof}
			Si $\delta \in \inv{\pi}(\pi(J)) \implies \delta \in J$. Además, $\delta \in \inv{\pi}(\pi(J)) \iff \pi(\delta) \in \pi(J) \iff \pi(\delta) = \pi(d_1),\ d_1 \in J \iff \delta - d_1 \in \ker \pi = I$. Tomamos
			\begin{align*}
				\delta = \underbrace{(\delta - j_i)}_{\in I} + \underbrace{j_i}_{\in J} \in J
			\end{align*}
			porque $I \subset J$.
		\end{proof}
	\end{enumerate}
\end{enumerate}


La siguiente proposición nos llevará al primer teorema de la isomorfía.
\begin{pro}
	Sea $\varphi: A \to B$ un homomorfismo de anillos con $\ker \varphi$ ideal en $A$. Sea $I$ un ideal en $A$ con $I \subset \ker \varphi$.
	\begin{itemize}
		\item Existe un único homomorfismo de anillos $\overline{\varphi}: A / I \to B$ tal que $\varphi = \overline{\varphi} \circ \pi$.
		\begin{figure}[h]
			\centering
			\begin{tikzpicture}
			\node (A) at (0,0) {$A$};
			\node (B) at (4,0) {$B$};
			\node (AI) at (0, -3) {$A / I$};
			
			\draw[-{Latex[length=2mm]}] (A) -- (B) node[pos=.5, above] {$\varphi$};
			\draw[-{Latex[length=2mm]}] (A) -- (AI) node[pos=.5, left]{$\pi$};
			\draw[-{Latex[length=2mm]}] (AI) -- (B) node[pos=.5, below] {$\overline{\varphi}$};
			\end{tikzpicture}
		\end{figure}
		\begin{proof}
			Definimos $\overline{\varphi}(\overline{a}) = \varphi(a)$. Aunque choque (porque el $\overline{a}$ puede venir de muchos $a$) aseguramos que $\overline{\varphi}$ está bien definida. Veamos por qué. Sabemos que $a'$ y $a$ definen el mismo elemento en $A / I \iff a' - a \in I$. Sopongamos que $I \subset \ker \varphi$. Entonces $\varphi(a - a') = 0 \iff \varphi(a) - \varphi(a') = 0 \implies \overline{\varphi}$ está bien definida como función.
			
			Veamos ahora que en efecto se cumple que $\overline{\varphi}$ es un homomorfismo de anillos, es decir que $\overline{\varphi}(\overline{a} + \overline{b}) = \overline{\varphi}(\overline{a}) + \overline{\varphi}(\overline{b})$. Recordando la definición que hemos dado de $\varphi$ y la propiedad $\overline{a} + \overline{b} = \overline{a + b}$ es claro que $\overline{\varphi}(\overline{a} + \overline{b}) = \overline{\varphi}(\overline{a +b}) = \varphi(a + b) = \varphi(a) + \varphi(b) = \overline{\varphi}(\overline{a}) + \overline{\varphi}(\overline{b})$. Es análogo para el producto ya que $\overline{a} \cdot \overline{b} = \overline{a \cdot b}$.
		\end{proof}
		\item $\ker \overline{\varphi} = \ker \varphi / I$
		\begin{proof}
			Sea $\overline{a} \in A / I$. Entonces $\overline{a} \in \ker \overline{\varphi} \iff \overline{\varphi}(\overline{a}) = 0 \iff \varphi(a) = 0 \iff a \in \ker \varphi$.
		\end{proof}
	\end{itemize}
\end{pro}

\begin{thm}[Primer teorema de la isomorfía (anillos)]
	Si $\alpha: A \to B$ es un homomorfismo de anillos sobreyectivo entonces $B \isom A / \ker \alpha$.
\end{thm}

	\begin{figure}[h]
	\centering
	\begin{tikzpicture}
	\node (A) at (0,0) {$A$};
	\node (B) at (4,0) {$B$};
	\node (AI) at (0, -3) {$A / \ker \alpha$};
	
	\draw[-{Latex[length=2mm]}] (A) -- (B) node[pos=.5, above] {$\alpha$};
	\draw[-{Latex[length=2mm]}] (A) -- (AI) node[pos=.5, left]{$\pi$};
	\draw[-{Latex[length=2mm]}] (AI) -- (B) node[pos=.5, below] {$\overline{\alpha}$};
	\end{tikzpicture}
\end{figure}

\begin{proof}
	Nos apoyamos en la proposición anterior tomando $I = \ker \alpha$. Como $\alpha$ y $\pi$ son sobreyectivas tenemos que $\overline{\alpha}$ es sobreyectiva. Aplicando el segundo resultado de la proposición anterior tenemos que $\ker \overline{\alpha} = \ker \alpha / \ker \alpha = \{ 0\} \implies \overline{\alpha}$ es inyectiva. Concluimos que $\overline{\alpha}$ es un isomorfismo de anillos y por tanto $B \isom A / \ker \alpha$.
\end{proof}




