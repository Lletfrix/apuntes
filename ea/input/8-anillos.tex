% !TeX root = ../apuntes-ea.tex

\chapter{Anillos}

\newcommand{\0}{\mathbf{0}}
\newcommand{\1}{\mathbf{1}}

\section{Definición y propiedades básicas}

\begin{dfn}[Anillo]
	Un anillo es una terna $(A, +, \cdot)$ donde $+$ es una operación a la que llamamos suma, $\cdot$ es otra operación a la que llamamos producto y se verifican las siguientes propiedades
	\begin{enumerate}
		\item El par $(A, +)$ es un grupo abeliano
		\item El producto $\cdot$ es asociativo
		\item Se cumplen las propiedades distributivas:
		\begin{align}
			\forall a, b , c \in A,\ a\cdot (b + c) = a\cdot b + a \cdot c \\
			\forall a, b , c \in A,\ (a + b) \cdot c = a\cdot c + b \cdot c
		\end{align}
	\end{enumerate}
\end{dfn}

Con la operación $+$ tenemos las siguientes propiedades
\begin{enumerate}
	\item Asociatividad: $(a+b)+c = a+(b+c)$
	\item Elemento neutro aditivo: $\exists! \0 \in A \mid \0+a = a$
	\item Elemento inverso aditivo: $\forall a \in A, \exists -a \in A \mid a + (-a) = \0$
	\item Conmutatividad aditiva: $\forall a, b \in A,\ a+b = b+a$
\end{enumerate}

Con la operación $\cdot$ tenemos las siguientes propiedades
\begin{enumerate}
	\item Asociatividad: $a\cdot (b \cdot c) = (a \cdot b) \cdot c$
	\item Elemento neutro multiplicativo: $\exists \1 \in A \mid a\cdot 1 = 1 \cdot a = a$
	\item No siempre existe inverso multiplicativo: $\inv{a} \mid a\cdot \inv{a} = \1$
	\item No siembre se da la conmutatividad multiplicativa: $a \cdot b = b\cdot a$
\end{enumerate}

Además, de la unicidad del $\0$ tenemos

\begin{pro}
	$\forall a \in A,\ a\cdot \0 = \0$
\end{pro}

\begin{proof}
	$a \cdot \0 = a \cdot(\0 + \0) = a\cdot \0 + a\cdot \0 \implies \0 = a\cdot \0$
\end{proof}

Y por último

\begin{pro}
	Sea $(A, +, \cdot)$ un anillo con más de un elemento ($|A| > 1$). Entonces $\0 \neq \1$ (el neutro aditivo es distinto del neutro multiplicativo).
\end{pro}

\begin{proof}
	Por contradicción. Sea $a \neq \0 \in A$ y $\0 = \1$. Entonces $a = a\cdot \1 = a\cdot 0 = 0 \implies a = \0$ contradicción.
\end{proof}

Para acabar veamos algunos ejemplos.

\begin{ej}
	Las matrices cuadradas $2\times 2$ con coeficientes reales: $(M_{2\times 2}(\R), +, \cdot)$ es un anillo. Tiene unidades $\uds{A} = (GL_2(\R), \cdot)$
\end{ej}

\begin{ej}
	Es cierto que $\Z, \mathbb{Q}, \R, \mathbb{C}$ son anillos con las operaciones suma y producto habituales. Además hay otros como por ejemplo $\Z[i] = \{a+bi \in \mathbb{C} \mid a,b \in \Z\}$.
\end{ej}


\subsection{Unidades en anillos}

Además, para pulir lo de que no siempre existe inverso multiplicativo daremos la definición de Grupo de unidades en el contexto de anillos que da sentido al \autoref{ej:grupounidades} que hemos estado utilizando.

\begin{dfn}[Unidades en anillos]
	Dado $(A, +, \cdot)$ anillo. El grupo de unidades es
	\begin{align}
		\uds{A} = (\{a \in A \mid \exists \inv{a} \in A,\ a\cdot \inv{a} = \1\}, \cdot)
	\end{align}
	Los elementos del grupo de unidades se llaman elementos invertibles.
\end{dfn}

Para que este grupo quede bien definido estaría bien que $\inv{a}$ fuera único.

\begin{pro}
	El inverso multiplicativo es único.
\end{pro}

\begin{proof}
	Sea $a \in A$. Supongamos $a', a'' \in A$ son ambos inversos multiplicativos de $a$:
	\begin{align*}
		aa' = \1 \land aa'' = \1\implies \begin{cases}
		a''(aa') = a'' \\
		(a''a)a' = a''
		\end{cases}\implies a' = a''
	\end{align*}
\end{proof}

\begin{pro}
	Sea $-\1$ el inverso aditivo del neutro multiplicativo $\1$. Entonces $\forall a \in A$ el inverso aditivo es $-a = -\1 \cdot a$ y se tiene $-\1\cdot a + a = 0$.
\end{pro}

\begin{pro}
	Sea $A$ un anillo. El neutro aditivo $\0$ verifica $\0 \not\in \uds{A}$
\end{pro}

\begin{pro}[Propiedad cancelativa]
	Sea $a \in \uds{A}$. Entonces $\forall b,c$ se tiene $b, c \in A \implies a\cdot b = a\cdot c \implies b = c$
\end{pro}

\begin{ej}
	Los numeros enteros $(\Z, +, \cdot)$ es un anillo y tienen unidades $\uds{\Z} = (\{-1, 1\}, \cdot)$
\end{ej}


\section{Dominios de integridad. Cuerpos.}

\begin{dfn}[Divisor de 0]
	Sea $(A, +, \cdot)$ un anillo. Diremos que $a \in A$ es divisor de $\0 \iff a \neq \0 \land \exists b \neq \0 \in A \mid a\cdot b = \0$
\end{dfn}

\begin{ej}
	En $\Z/8\Z$ el elemento $\overline{2}$ es un divisor de $\0$:
	\begin{align*}
		\overline{2}\cdot\overline{4} = \overline{8} = \0
	\end{align*}
	Sin embargo, $\overline{3}$ no lo es ya que
	\begin{align*}
		\overline{3}\overline{a} = \0 \implies \inv{\overline{3}}\overline{3}\overline{a} = \inv{\overline{3}}\0 \implies \1 \overline{a} = \0 \implies \overline{a} = 0
	\end{align*}
\end{ej}

Otra manera de ver esto es plantear la ecuación $a \cdot b = \0$. El $\0$ siempre es una solución, pero podemos decir que $a$ es un divisor de $\0$ si la ecuación tiene soluciones no triviales.

\begin{ej}$ $\newline
	\begin{itemize}
		\item En $\Z$ la ec. $2x = 0$ solo tiene la solución trivial $\implies 2$ no es un divisor de $\0$.
		\item En $\Z/6\Z$ la ec. $\overline{2}\overline{x} = 0$ tiene la solución $\overline{x} = \overline{3}$ luego $\overline{2}$ sí que es un divisor de $0$.
	\end{itemize}
\end{ej}

\begin{pro}
	Si $a \in \uds{A}$ entonces $a$ no es un divisor de $0$.
\end{pro}

\begin{proof}
	Si $a \in \uds{A}$ sabemos que existe $\inv{a} \mid \inv{a}a = \1 \implies ax = \0 \iff \inv{a}ax = \inv{a} \0 \iff \1 x = \0 \iff x = \0$ luego la ecuación solo tiene la solución trivial.
\end{proof}

\begin{pro}
	Sea $A$ un anillo. $\forall a \in A$ no divisor de 0 $\implies$ se cumple la propiedad cancelativa.
\end{pro}

\begin{proof}
	$ab = ac \implies b = c \iff ab + (-ac) = a(b -c) = \0$
\end{proof}

\begin{dfn}[Dominio de integridad]
	Un anillo que no tiene elementos divisores de 0 se llama dominio de integridad (DI).
\end{dfn}

% TODO profundizar con la p. 159 de santorum
\begin{ej}$ $\newline
	\begin{itemize}
		\item $\Z$ es un dominio de integridad ya que todo $a \in \Z, a \neq \0$ tiene un inverso multiplicativo $\inv{a}$.
		\item $\Z/p\Z$ con $p$ primo es un dominio de integridad.
		\item $\Z/n\Z$ con $n$ no primo no es un dominio de integridad ya que si $\overline n = ab$ con $a \neq n \land b \neq n$ se tiene $\overline{a} \cdot \overline{b} = \overline{n} = \0$ con $\overline{a} \neq \0 \land \overline{b} \neq \0$.
	\end{itemize}
\end{ej}

\begin{pro}
	$A$ es un DI si el producto de elementos no nulos es no nulo.
\end{pro}

\begin{dfn}[Anillo conmutativo]
	Sea $A$ un anillo. $A$ es un anillo conmutativo $\iff \forall a, b \in A,\ a\cdot b = b \cdot a$.
\end{dfn}

\begin{dfn}[Cuerpo]
	Diremos que un anillo conmutativo es un cuerpo si $\uds{A} = A \setminus \{\0\}$
\end{dfn}

\begin{ej}
	\item $\Z/p\Z$ con $p$ primo es un cuerpo
	\item $\Z$ o $\Z/6\Z$ no son cuerpos
\end{ej}

\begin{pro}
	Si $A$ es un cuerpo entonces $A$ es un DI.
\end{pro}


Hay que buscar una buena definición para esto.
\begin{dfn}[Estructura de anillos]
	% TODO: dar una definición de verdad
	Sea $(A, +, \cdot)$ un anillo. Una estructura de anillos $A[x]$ es una cosa con elementos de la forma $ax$ para $a \in A$.
\end{dfn}

\begin{pro}
	$A$ DI $\implies A[x]$ DI
\end{pro}

\begin{obs}
	NO se da el recíproco: por ejemplo, $\Q$ es un cuerpo $\implies \Q$ DI $\implies \Q[x]$ dominio, PERO $\Q[x]$ no es un cuerpo.
\end{obs}

\section{Ideales}

\begin{dfn}[Ideal]
	Un subconjunto $I \subset A$ es un ideal $\iff$
	\begin{enumerate}
		\item $\0 \in I$
		\item $s,t \in I \implies s+t \in I$
		\item $s \in I \land a \in A \implies a \cdot s \in I$
	\end{enumerate}
\end{dfn}

\begin{pro}
	Si $I$ es un ideal en $A$ entonces $(I, +) < A$ (es decir, $I$ es un subgrupo aditivo de $A$).
\end{pro}

\begin{ej}
	$0\Z,\ 1\Z,\ 2\Z, 3\Z,\ 4\Z$ son todos ideales de $\Z$ (y subgrupos de $(\Z, +)$).
\end{ej}

\begin{thm}
	Sea $\{I_j\}_{j \in J}$ una familia de ideales en $A$ ($I_j \subset A$ es un ideal para cada $j \in J$). Entonces $\cap_{j \in J} I_j$ es un ideal en $A$.
\end{thm}

\begin{ej}
	En el anillo $\Z$ están los ideales $6\Z$ y $8\Z$. Por la proposición anterior $6\Z \cap 8\Z$ es un ideal (es $mcd(6,8)\Z = 2\Z$).
\end{ej}

\begin{pro}
	Sean $I, J \subset A$ ideales. Entonces $I + J = \{i + j \mid i \in I \land j \in J\}$ es un ideal\footnote{Ojo: aquí $I+J$ no tiene nada que ver con la suma directa (el caso particular del producto directo) que había en grupos.} de $A$ que contiene a $I$, a $J$, y además es el mayor ideal con esta propiedad.
\end{pro}

\begin{pro}
	Sea $S = \{S_1, \dots, S_r\} \subset A$ y sea $\gen{S} = \{\sum_{i=1}^{r}a_is_i \mid a_i \in A\}$. Entonces
	\begin{enumerate}
		\item $\gen{S}$ es un ideal en $A$
		\item $S \subseteq \gen{S}$
		\item Si $T$ es un ideal, entonces $S \subset T \iff \gen{S} \subset T$
	\end{enumerate}
\end{pro}

\begin{dfn}[Ideal principal]
	Diremos que un ideal es principal si está generado por un único elemento.
\end{dfn}

\begin{ej}
	En $\Z$ consideramos el subconjunto $S = \{6\} \subset \Z$. El conjunto $\gen{S} = \{a6 \mid a \in \Z\} = 6\Z$ es un ideal y como está generado por un único elemento, es un ideal principal.
\end{ej}

\begin{obs}
	En $\Z$ todo ideal es principal.
\end{obs}

\section{Subanillos}

% TODO: me quedo aquí

\section{Por revisar}

% -------------

\begin{thm}
	Dado el anillo $A$ y un ideal propio $I$
	\begin{align*}
		\pi: A \to A/I,\qquad I \subset \inv{\pi}(\overline{J}) \subset A,\qquad \overline{0} \in \overline{J} \subset A / I
	\end{align*}
	
	existe una identificación entre el retículo de ideales $A / I$ con el subretículo de ideales de $A$ que contienen a $I$. 
	
	Es decir, si $J$ es un ideale en $A/I$ entonces $\inv{\pi}(\overline{J})$ es un ideal en $A$ que contiene al ideal $I$.
\end{thm}


El ideal cero de $A/I$ tiene contraimagen $\inv{\pi}(\{0\}) = I$. Si $\overline{J}$ es un ideal en $A/I$
\begin{align*}
	\pi : A \to A/I \to (A/I) / \overline{J}
\end{align*}

es un homomorfismo de anillos (la composición de homomorfismos de anillos es un homomorfismo de anillos). $\inv{\pi}(\overline{J}) = \ker$ de la composición.

% --------------------

\begin{thm}
	Sea $\alpha: A \to B$ un homomorfismo de anillos.
	\begin{itemize}
		\item $\ker \alpha$ es un ideal
		\item $\ima \alpha$ es un subanillo
		\item $\alpha$ es sobreyectivo $\iff \ima \alpha = B$
		\item $\alpha$ es inyectivo $\iff \ker \alpha = \{\0\}$
	\end{itemize}
\end{thm}

\begin{dfn}[Isomorfismo de anillos]
	Un homomorfismo de anillos $\alpha: A \to B$ es un isomorfismo cuando es una biyección. En este caso decimos que $A$ y $B$ son isomorfos y lo notamos con $A \isom B$.
\end{dfn}

\begin{pro}
	Si $\alpha: A \to B$ es un homomorfismo de anillos y una biyección de conjuntos entonces $\inv{\alpha}:B \to A$ es nuevamente un homomorfismo de anillos.
\end{pro}

\subsubsection{Homomorfismos de anillos e ideales}

\begin{thm}
	Sea $\alpha: A \to B$ un homorfismo de anillos. Entonces
	\begin{enumerate}
		\item Si $J \subset B$ es un ideal en $B$ entonces $\inv{\alpha}(J)$ es un ideal en $A$.
		\item Si $\alpha$ es sobreyectiva entonces la imagen $\alpha(I)$ de un ideal $I \subset A$ es un ideal en $B$
	\end{enumerate}
\end{thm}

\begin{proof}
	\begin{figure}[h]
		\centering
		\begin{tikzpicture}
		\node (A) at (0,0) {$A$};
		\node (B) at (4,0) {$B$};
		\node (BJ) at (4, -3) {$B / J$};
		
		\draw[-{Latex[length=2mm]}] (A) -- (B) node[pos=.5, above] {$\alpha$};
		\draw[-{Latex[length=2mm]}] (B) -- (BJ) node[pos=.5, left]{$\pi$};
		\draw[-{Latex[length=2mm]}] (A) -- (BJ) node[pos=.5, below] {$\pi \circ \alpha$};
		\end{tikzpicture}
	\end{figure}
	\begin{enumerate}
		\item $\inv{\alpha}(J) = \ker(\pi \circ \alpha)$ y por tanto es un ideal.
		\item Probamos las propiedades de los ideales:
		\begin{enumerate}
			\item $\alpha(0) = 0 \in \alpha(I)$
			\item Sean $b_1, b_2 \in \alpha(I)$ tenemos que ver que $b_1 + b_2 \in \alpha(I)$. Sean $a_1, a_2 \in I$ tales que $b_1 = \alpha(a_1) \land b_2 = \alpha(a_2)$. Por ser $\alpha$ h. de anillos tenemos que $b_1 + b_2 = \alpha(a_1 + a_2) = \alpha(a_1) + \alpha(a_2)$.
			\item Sean $b \in B,\ b' \in \alpha(I)$. Tenemos que probar que $bb' \in \alpha(I)$. Sabemos que $b' \in \alpha(I) \iff b' = \alpha(a),\ a \in I$. Como $b \in B$ y $\alpha$ es sobre tiene que existir $d \in I \mid \alpha(d) = b$. Por tanto $\alpha(d\cdot a) = b \cdot b' \implies bb' \in \alpha(I)$.
		\end{enumerate}
	\end{enumerate}
\end{proof}

Fijado $I \subset A$ consideramos $\pi:A \to A/I$ que es un homomorfismo de anillos sobreyectivo.

\begin{enumerate}
	\item Si $\overline{J} \subset A /I$ es un ideal en $A / I$ entonces $\inv{\pi}(\overline{J})$ es un ideal en $A$ que contiene a $I$.
	\item Si $J$ es un ideal en $A$ entonces $\pi(J)$ es un ideal en $A/J$ y $J \subseteq \inv{\pi}(\pi(J))$ (es claro porque si $j \in J$ entonces $\pi(j) \in \pi(J)$).
	\begin{enumerate}
		\item Además, si $I \subseteq J$ entonces $J = \inv{\pi}(\pi(J))$.
		\begin{proof}
			Si $\delta \in \inv{\pi}(\pi(J)) \implies \delta \in J$. Además, $\delta \in \inv{\pi}(\pi(J)) \iff \pi(\delta) \in \pi(J) \iff \pi(\delta) = \pi(d_1),\ d_1 \in J \iff \delta - d_1 \in \ker \pi = I$. Tomamos
			\begin{align*}
				\delta = \underbrace{(\delta - j_i)}_{\in I} + \underbrace{j_i}_{\in J} \in J
			\end{align*}
			porque $I \subset J$.
		\end{proof}
	\end{enumerate}
\end{enumerate}


La siguiente proposición nos llevará al primer teorema de la isomorfía.
\begin{pro}
	Sea $\varphi: A \to B$ un homomorfismo de anillos con $\ker \varphi$ ideal en $A$. Sea $I$ un ideal en $A$ con $I \subset \ker \varphi$.
	\begin{itemize}
		\item Existe un único homomorfismo de anillos $\overline{\varphi}: A / I \to B$ tal que $\varphi = \overline{\varphi} \circ \pi$.
		\begin{figure}[h]
			\centering
			\begin{tikzpicture}
			\node (A) at (0,0) {$A$};
			\node (B) at (4,0) {$B$};
			\node (AI) at (0, -3) {$A / I$};
			
			\draw[-{Latex[length=2mm]}] (A) -- (B) node[pos=.5, above] {$\varphi$};
			\draw[-{Latex[length=2mm]}] (A) -- (AI) node[pos=.5, left]{$\pi$};
			\draw[-{Latex[length=2mm]}] (AI) -- (B) node[pos=.5, below] {$\overline{\varphi}$};
			\end{tikzpicture}
		\end{figure}
		\begin{proof}
			Definimos $\overline{\varphi}(\overline{a}) = \varphi(a)$. Aunque choque (porque el $\overline{a}$ puede venir de muchos $a$) aseguramos que $\overline{\varphi}$ está bien definida. Veamos por qué. Sabemos que $a'$ y $a$ definen el mismo elemento en $A / I \iff a' - a \in I$. Sopongamos que $I \subset \ker \varphi$. Entonces $\varphi(a - a') = 0 \iff \varphi(a) - \varphi(a') = 0 \implies \overline{\varphi}$ está bien definida como función.
			
			Veamos ahora que en efecto se cumple que $\overline{\varphi}$ es un homomorfismo de anillos, es decir que $\overline{\varphi}(\overline{a} + \overline{b}) = \overline{\varphi}(\overline{a}) + \overline{\varphi}(\overline{b})$. Recordando la definición que hemos dado de $\varphi$ y la propiedad $\overline{a} + \overline{b} = \overline{a + b}$ es claro que $\overline{\varphi}(\overline{a} + \overline{b}) = \overline{\varphi}(\overline{a +b}) = \varphi(a + b) = \varphi(a) + \varphi(b) = \overline{\varphi}(\overline{a}) + \overline{\varphi}(\overline{b})$. Es análogo para el producto ya que $\overline{a} \cdot \overline{b} = \overline{a \cdot b}$.
		\end{proof}
		\item $\ker \overline{\varphi} = \ker \varphi / I$
		\begin{proof}
			Sea $\overline{a} \in A / I$. Entonces $\overline{a} \in \ker \overline{\varphi} \iff \overline{\varphi}(\overline{a}) = 0 \iff \varphi(a) = 0 \iff a \in \ker \varphi$.
		\end{proof}
	\end{itemize}
\end{pro}

\begin{thm}[Primer teorema de la isomorfía (anillos)]
	Si $\alpha: A \to B$ es un homomorfismo de anillos sobreyectivo entonces $B \isom A / \ker \alpha$.
\end{thm}

	\begin{figure}[h]
	\centering
	\begin{tikzpicture}
	\node (A) at (0,0) {$A$};
	\node (B) at (4,0) {$B$};
	\node (AI) at (0, -3) {$A / \ker \alpha$};
	
	\draw[-{Latex[length=2mm]}] (A) -- (B) node[pos=.5, above] {$\alpha$};
	\draw[-{Latex[length=2mm]}] (A) -- (AI) node[pos=.5, left]{$\pi$};
	\draw[-{Latex[length=2mm]}] (AI) -- (B) node[pos=.5, below] {$\overline{\alpha}$};
	\end{tikzpicture}
\end{figure}

\begin{proof}
	Nos apoyamos en la proposición anterior tomando $I = \ker \alpha$. Como $\alpha$ y $\pi$ son sobreyectivas tenemos que $\overline{\alpha}$ es sobreyectiva. Aplicando el segundo resultado de la proposición anterior tenemos que $\ker \overline{\alpha} = \ker \alpha / \ker \alpha = \{ 0\} \implies \overline{\alpha}$ es inyectiva. Concluimos que $\overline{\alpha}$ es un isomorfismo de anillos y por tanto $B \isom A / \ker \alpha$.
\end{proof}

% ------- 20181217

\begin{thm}
	\begin{align*}
	D \text{ es un dominio de ideales principales (DIP) } \implies D \text{ es un dominio de factorización única (DFU)}
	\end{align*}
\end{thm}

El recíproco de este teorema no es cierto en general. Véase por ejemplo el caso de $\Z$ que es un dominio de ideales principales pero no se cumple que $\Z[X]$ es un dominio de factorización única. Si se cumpliera el recíproco entonces el siguiente teorema sería un simple corolario.

\begin{thm}
	\begin{align*}
	D \text{ es un dominio de factorización única (DFU) } \implies D[X] \text{ es un dominio de factorización única (DFU)}
	\end{align*}
\end{thm}

Este segundo teorema no lo vamos a probar. Probamos el primero.

\begin{dfn}[Asociados]
	Sea $D$ un domino, $a,a' \in D$. DIremos que $a$ y $a'$ son asociados $\iff \exists u \in \uds{D} \mid a = u a'$.
\end{dfn}

\begin{proof}
	Sea $D$ un dominio, $a \in D \mid a \neq 0 \land a \not\in \uds{D}$. Sabemos que $a, a' \in D$ son asociados si $\exists u \in \uds{D} \mid a = ua'$. Por ejemplo, los polinomios $3x-2$ y $x - 2/3$ en $\Q[X]$ son asociados.
	
	Observemos que si $a$ y $a'$ son asociados entonces $\gen{a} = \gen{a'}$. Si $u \in \uds{a}$ entonces $ua' = a \in \gen{a'}$. Análogamente $\inv{u}a = a' \in \uds{a}$. Luego tenemos $\gen{a} \subset \gen{a'} \land \gen{a'} \subset \gen{a} \implies \gen{a} = \gen{a'}$. Recíprocamente si $0 \neq \gen{a} = \gen{a'} \implies \exists u \in \uds{D} \mid a = ua'$. $a \in \gen{a'} \land a' \in \gen{a} \implies a = a't \land a' = as \implies a' = a'ts \implies 1 = ts \implies t,s \in \uds{D}$.
	
	Recordemos las hipótesis iniciales: $a \in D \mid a \neq 0 \land a \not\in \uds{D}$. Esto nos da que $0 \neq \gen{a} \land \gen{a} \subsetneq D$. Pensemos en qué significa que un elemento no nulo $a$ no sea una unidad. Supongamos $a = st$. Si $a$ no es una unidad podría ocurrir que $s$ es una unidad (por ejemplo $6 = (-1)(-6),\ -1 \in \uds{\Z}$). Lo que sí que está claro es que no puede ocurrir que a la vez $s$ y $t$ sean unidades. Es decir, tiene que ocurrir que al menos uno de los dos no es una unidad. Por tanto podemos suponer sin pérdida de generalidad que si expresamos $a = a' \cdot s$  entonces $a' \not \in \uds{D}$. Tenemos dos situaciones posibles
	\begin{enumerate}
		\item $s \in \uds{D} \implies \gen{a} = \gen{a'}$
		\item $s \not \in \uds{D} \implies \gen{a} \subsetneq \gen{a'}$ ya que $\gen{a} = \gen{a'} \iff a = a'u$ con $u \in \uds{D}$ pero hemos tomado $s \not\in \uds{D}$
	\end{enumerate}
\end{proof}

Aquí para de demostrar y empieza a dar definiciones.

\begin{dfn}[Irreducible]
	Sea $D$ un dominio y $0 \neq a \not\in \uds{D}$. Diremos que $a$ es irreducible en $D \iff \forall a',s \in D,\ a' \not \in \uds{D},\ a = a's \implies s \in \uds{D}$
\end{dfn}

\begin{obs}
	Un elemento es irreducible $\iff$ cualquier asociado lo es.
\end{obs}

\begin{dfn}[Dominio de factorización única (DFU)]
	Sea $D$ un dominio. Diremos que $D$ es un dominio de factorización única (DFU) si se cumplen las siguientes condiciones $\forall a \in D$:
	\begin{itemize}
		\item $a \neq 0 \land a \not\in\uds{D} \implies a = p_1p_2\dots p_r$ donde $p_i$ es irreducible en $D$
		\item $a = p_1p_2\dots p_r,\ p_i$ irreducible y $a = q_1q_2 \dots q_s,\ q_i$ irreducible $\implies r = s$ y además $r_i$ y $q_i$ son asociados para $i = 1, \dots, r$ (la igualdad es un caso particular de el ser asociados).
	\end{itemize}
\end{dfn}

\begin{obs}
	Sea $I_1 \subseteq I_2 \subseteq I_3 \subseteq ...$ una cadena creaciente de ideales de un anillo $A$. Entonces $\bigcup I_i$ es un ideal.\footnote{Literalmente ha dicho que esto no viene a cuento. Que esto es una digresión de las suyas.}
\end{obs}

\begin{proof}
	Probamos las propiedades de los ideales.
	\begin{enumerate}
		\item $0 \in \bigcup I_i$
		\item $s,t \in \bigcup I_i \implies s+t \in \bigcap I_i$
		\item $s \in \bigcup I_i,\ a \in A \implies as \in \bigcup I_i$.
	\end{enumerate}
\end{proof}

\begin{dfn}
	[Propiedad de cadena creciente]
	
	Diremos que un anillo $A$ tiene la propiedad de cadena creciente $\iff$ toda cadena creciente $I_1 \subseteq I_2 \subseteq I_3 \subseteq \dots \subseteq I_n \subseteq \dots$ es finita. Es decir, que $\exists n \mid I_n = I_{n+1} = I_{n+2} = \dots$.
\end{dfn}

\begin{thm}
	Si $D$ es un DIP entonces $D$ tiene la propiedad de cadena creciente.
\end{thm}

La demostración es tan ingenua como uno quiera.

\begin{proof}
	Sea $I_1 \subseteq I_2 \subseteq I_3 \subseteq \dots \subseteq I_n \subseteq \dots$ una cadena de ideales. Sabemos que en cualquier anillo $\bigcup I_i$ es un ideal. Sea $J = \gen{d}$ para algún $d \in D$. Como $D$ es un DIP ocurre que $d \in \bigcup I_i \implies d \in I_{n_0} \implies \gen{d} \subset I_{n_0} \implies I_{n_0} = I_{n_0 + 1} = \dots$
\end{proof}

