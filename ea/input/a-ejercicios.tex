% !TeX root = ../apuntes-ea.tex

\chapter{Ejercicios}

\section{Hoja 1}

\begin{ex}[H1.2]
	Sean $a,b,c \in G = (-1,1)$. Probamos las propiedades de los grupos.
	\begin{itemize}
		\item \textbf{Asociatividad:}
		\begin{align*}
			(a \ast b) \ast c = \left(\frac{a+b}{1+ab}\right) \ast c = \frac{\left(\frac{a+b}{1+ab}\right) + c}{1 + \left(\frac{a+b}{1+ab}\right)c} = \frac{\frac{a+b+c+abc}{1+ab}}{\frac{1+ab+ac+bc}{1+ab}} = \frac{a+b+c+abc}{1+ab+ac+bc} \\
			a \ast (b \ast c) = a \ast \left(\frac{b+c}{1+bc}\right) = \frac{a+\left(\frac{b+c}{1+bc}\right)}{1+a\left(\frac{b+c}{1+bc}\right)} = \frac{\frac{a + b + c + abc}{1+bc}}{\frac{1+ab+ac+bc}{1+bc}} = \frac{a+b+c+abc}{1+ab+ac+bc}
		\end{align*}
		\item \textbf{Elemento neutro:} es el $0$ ya que $x\ast 0 = \frac{x+0}{1+x\cdot 0} = \frac{x}{1} = x$ y además $0 \ast x = \frac{0+x}{1+0\cdot x} = \frac{x}{1} = x$
		\item \textbf{Elemento inverso:} la ecuación
		\begin{align*}
			x \ast \inv{x} = 0 \iff \frac{x + \inv{x}}{1+x\inv{x}} = 0 \iff \inv{x} = -x
		\end{align*}
		siempre tiene solución y ocurre lo mismo para la ecuación $\inv{x} \ast x = 0 \iff \inv{x} = -x$
		\item \textbf{Clausura:} tenemos que probar que si $x,y \in (-1,1)$ entonces $x \ast y \in (-1,1)$. Consideramos $f(x,y) = x \ast y = \frac{x+y}{1+xy}$. Derivando tenemos que $\nabla f(x,y) = (\frac{1}{(1+xy)^2},\ \frac{1}{(1+xy)^2}) \neq 0, \forall x,y \in [-1,1]\times[-1,1]$. Si el máximo no se alcanza en ningún sitio de dentro del cuadrado $(-1,1)\times(-1,1)$ se tendrá que alcanzar en el borde.
		\begin{itemize}
			\item Fijado $x = 1$ tenemos que $f(1,y) = \frac{1+y}{1+y} = 1 \implies f(1, -1 < y < 1) < 1$ porque si $f(1, -1 < y < 1)$ tomara un valor mayor que $1$ habría un máximo en $(-1,1)\times(-1,1)$ y esto no puede ser pues $\nabla f$ no se anula en el cuadrado.
			\item Fijado $x = -1$ tenemos que $f(-1, y) = \frac{y-1}{1-y} = -1 \implies f(-1, -1 < y < 1) > -1$ por la misma razón que antes.
			\item Hacemos lo mismo fijando la $y$ y variando la $x$.
		\end{itemize}
		En el borde (que no está incluido) se alcanzan máximo y mínimo que acotan a $f$ en el cuadrado:
		\begin{align*}
			-1 < f(x,y) = x \ast y < 1,\quad \forall x, y \in G
		\end{align*}
	\end{itemize}
\end{ex}

\begin{ex}[H1.3]Hallar los inversos de los siguientes elementos, cada uno en su grupo correspondiente:
	\begin{enumerate}
		\item $o(\overline{11})$ en $\uds{\Z^\ast/23\Z}$
		\item $o(\overline{5})$ en $\uds{\Z^\ast/31\Z}$
	\end{enumerate}
\end{ex}

\begin{proof}
	Por el pequeño teorema de Fermat (\cite{flt}) tenemos que $a^{p-1} \equiv 1 \mod p$ para $p$ primo. Entonces
	\begin{align*}
		11^22 \equiv 1 \mod 23 &\implies o(\overline{11}) = 22 \implies \inv{\overline{11}} = \overline{11}^{21} \\
		5^20 \equiv 1 \mod 31 &\implies o(\overline{5}) = 30 \implies \inv{\overline{5}} = \overline{5}^{29}
	\end{align*}
\end{proof}

\begin{ex}[H1.33]
	\label{ex:h1.33}
	Sea $G$ un grupo. Suponed que existe un único $a \in G$ de orden 2. Demostrad que $a \in Z(G)$.
\end{ex}

\begin{proof}
	Recordamos que $a \in Z(G) \iff ga = ag,\ \forall g \in G$. Definimos el isomorfismo de conjugación $\phi_g (x) = gx\inv{g}$ para algún $g$. Como $\phi_g$ es isomorfismo lleva elementos de orden $n$ en elementos de orden $n$. Entonces $\phi_g(a) = a$ ya que $a$ es el único elemento de orden 2. Por tanto $ga\inv{g} = a \implies ga = ag \implies a \in Z(G)$.
\end{proof}

\begin{ex}[H1.31]
	Sea $G$ un grupo de orden 8. Probad que o bien $G$ es cíclico o $a^4 = 1$, para cada $a \in G$.
\end{ex}

\begin{proof}
	Tenemos que probar que $G$ cíclico XOR $\forall a \in G,\ a^4 = 1$. Probamos dos implicaciones:
	\begin{itemize}
		\item $G$ cíclcio $\implies \exists a \in G \mid a^4 \neq 1$. Si $G$ es cíclcio entonces $\exists g \in G \mid o(g) = 8 \implies o(g^4) = \frac{o(g)}{mcd(o(g), 4)} = \frac{8}{mcd(8,4)} = 2$.
		\item Si para todo $a \in G a^4 = 1$ entonces no hay ningún $g \in G \mid o(g) = 8$ luego $G$ no puede estar generado por ningún elemento y por tanto no es cíclico.
	\end{itemize}
\end{proof}

\begin{ex}[H1.33]
	Sea $H$ un grupo. Suponed que existe un único $a\in G$ de orden 2. Demostrad que $a \in Z(G)$.
\end{ex}

\begin{proof}
	Sabemos que $Z(G) = \{a \in G\mid ag = ga, \forall g \in G\}$ y además $ag = ga \iff a = ga\inv{g}$. Definimos $\varphi_g(a) = ga\inv{g}$ (el isomorfismo conjugación) que es un isomorfismo $\forall g \in G \implies o(\varphi_g(a)) = o(a),\ \forall g \in G$. Si $a$ es el único elemento de orden 2 entonces necesariamente $\varphi_g(a) = a,\ \forall g\in G \implies a = ga\inv{g} \implies ag = ga \implies a \in Z(G)$.
\end{proof}

\section{Hoja 2}

\begin{ex}[H2.1]
	\label{ex:h2.1}
	Se considera el tercer grupo diédrico $D_3$. Se pide hallar lo siguiente:
	\begin{enumerate}
		\item Las clases de conjugación de cada uno de sus elementos.
		\begin{proof}
			Las clases dan una partición del grupo. Si un elemento pertenece a una clase, entonces la clase de ese elemento también es la clase a la que pertenece.
			\begin{itemize}
				\item $cl(e) = \{e\}$
				\item $cl(B)$? Sabemos que $|cl(B)| = [G:C(B)]$. Sabemos que $\gen{B} =\{1, B, B^2\} \subset C(B)$ luego $|C(B)| \geq 3$. Si hubiera más elementos en $C(B)$ tendríamos que $|C(B)| = 6$ pues $C(B) < D_3$. Esto no ocurre porque sabemos que $B$ no conmuta con todos los demás elementos. Por ejemplo $BA \neq AB$. Por tanto $|C(B)| = 3 \implies |cl(B)| = [D_3:C(B)]= 6/3 = 2$. Es claro que $B \in cl(B)$. Además, como $cl(B)$ contiene elementos transformados por el isomorfismo conjugación sabemos que el otro elemento que hay tiene orden 3. El único elemento que queda de orden 3 es $B^2 \implies cl(B) = \{B, B^2\}$.
				
				\item $cl(A)?$ Sabemos que $A$ no conmuta con todos ($A \not\in Z(D_3)$) luego $|C(A)| < 6$. Sabemos que $\gen{A} = \{1, A\} < C(A)$. Además, como $C(A)$ es un (sub)grupo sabemos que no puede haber más elementos porque si los hubiera, $|\gen{A}| \divides |C(A)| \implies C(A) \geq 6$ pero ya hemos visto que no puede ser. Es decir que $|cl(A)| = [D_3:D(A)] = 6 / 2 = 3$. Por tanto $cl(A)$ incluye los 3 elementos que nos quedan: $cl(A) = \{A, AB, AB^2\}$.
			\end{itemize}
		\end{proof}
	
		\item Los elementos de $\text{Int}(D_3)$.
		\item Los centralizadores $C_{D_3}(x)$ para cada $x \in D_3$
		\item Los normalizadores $N(H)$ para cada $H < D_3$.
	\end{enumerate}
\end{ex}

\begin{ex}[H2.2]
	
	\begin{proof}
		Obtenidas las clases en el ejercicio \nameref{ex:h2.1} se verifica que $|D_3| = |cl(e)| + |cl(B)| + |cl(A)| = 1 + 2 + 3 = 6$
	\end{proof}
	
\end{ex}

\begin{ex}[H2.6]
	Sea $G$ un grupo. ¿Verdadero o falso?
	\begin{enumerate}
		\item $H < G$ y $H$ conmutativo implica $H \normsub G$.
		\item $H < G$ y $|H| = 2$ implica $H \normsub G$.
		\item Si $\varphi: G \to G_1$ es un homomorfismo de grupos, entonces $\ima \varphi \normsub G$
		\item Si $H \normsub K$ y $K \normsub G$ entonces $H \normsub G$
		\item Si $H \normsub G$ y $|H| = m$ entonces $H$ es el único subgrupo de $G$ de orden $m$.
		\item Si $H \normsub G$ entonces $H < Z(G)$.
		\begin{proof}[FALSO]
			Contraejemplo: En $G = D_4$ tomamos $H = \gen{B^2} = \{1, B, B^2, B^3\} \not\subset Z(D_4) = \{1, B^2\}$.
		\end{proof}
	\end{enumerate}
\end{ex}

\begin{ex}[H2.10]
	\begin{proof}
		Fijado $n$ y definida $\alpha_n : G \to G,\ x \mapsto x^n$ tenemos que $\alpha_n$ es un homomorfismo de grupos. Además podemos expresar $H_2 = \ker \alpha_n \implies H_2 \normsub G$. Además también tenemos que $H_1 = \ima \alpha_n < G$. Veamos que $H_1 \normsub G$. Es decir, que $gH_1\inv{g} = H_1,\ \forall g \in G$. Para ello tomamos $x_1^n \in H_1$ y lo conjugamos $gx_1^n\inv{g} = (gx_1\inv{g})^n$ por ser $\alpha$ homomorfismo de grupos. En particular $(gx_1\inv{g})^n \in \ima \alpha \implies (gx_1\inv{g})^n \in H_1 \implies (gx_1\inv{g})^n = x_2^n$ para algún $x_2 \in H_1 \implies H_1 \normsub G$.
	\end{proof}
\end{ex}

\begin{ex}[H2.13] Si $A$ es un grupo abeliano con $n$ elementos y $k$ es un entero primo con $n$, demostrad que la aplicación $\varphi : A \to A$ definida por $\varphi(a) = a^k$ es un isomorfismo.
	\begin{itemize}
		\item $\varphi$ homomorfismo de grupos.
		\begin{proof}
			\begin{align*}
				\varphi(a)\varphi(b) = a^kb^k = (ab)^k = \varphi(ab)
			\end{align*}
		\end{proof}
		\item $\varphi$ biyectiva $\iff \varphi$ inyectiva ya que dominio y codominio coinciden
		\begin{proof}
			$\ker \varphi = \{a \in A \mid \varphi(a) = a^k = 1\}$. Probaremos que $a^k = 1 \iff a = 1$ y por tanto que $\ker \varphi = \{1\} \implies \varphi$ inyectiva. Sabemos que $a^k = 1 \iff o(a^k) = 1$. Sea $t = o(a) \divides n$. Distinguimos dos casos
			\begin{itemize}
				\item Si $t = 1$ entonces $a = 1$ y ya está
				\item Si $t > 1$ entonces $o(a^k) = \frac{t}{mcd(k, t)} = \frac{t}{1} > 1$ contradicción. Luego necesariamente $t = o(a) = 1$.
			\end{itemize}
		\end{proof}
	\end{itemize}
\end{ex}

\begin{ex}[H2.22]
	\label{ex:h2.22}
	Demostrad que si $G$ es un grupo no conmutativo y tiene orden $p^3$ ($p$ un número primo) entonces $Z(G)$ tiene orden $p$.
	
	\begin{proof}
		Sabemos que $Z(G) < G \implies |Z(G)| \divides |G| \implies |Z(G)| \in \{1, p, p^2, p^3\}$
		\begin{itemize}
			\item $|Z(G)| \neq p^3$ porque en tal caso $G$ sería conmutativo
			\item $|Z(G)| \neq 1$ porque $G$ es un p-grupo y por tanto su centro no es el trivial.
			\item Si $|Z(G)| = p^2$ entonces $|G/Z(G)| = p \implies G/Z(G)$ es cíclico lo que no es posible si $G$ no es abeliano.
		\end{itemize}
	
		Por descarte concluimos que $|Z(G)| = p$.
	\end{proof}
\end{ex}


\begin{ex}[H2.25]
	Sabemos que $\autom{\Z/12\Z} \isom \uds{\Z^\star/12\Z}$ donde $\Z^\star/12\Z$ es el grupo multiplicativo $(\{\overline{1}, \overline{2}, \overline{3}, \dots,  \overline{11}\}, \cdot)$. Queda $\uds{\Z^\star/12\Z} = (\{\overline{1}, \overline{5}, \overline{7}, \overline{11}\}, \cdot )$ y además da la casualidad que $\forall x \in \uds{\Z^\star/12\Z},\ o(x) = 2$ (todos los elementos son su propio inverso) por lo que no tenemos restricciones al definir $f: \Z/2\Z \to \autom{\Z/12\Z}$:
	\begin{align*}
		f:\Z/2\Z & \to \autom{\Z/12\Z} \isom \uds{\Z^\star/12\Z} \\
		e = \overline{0} &\mapsto 1 \\
		\overline{1} &\mapsto \{\overline{1}, \overline{5}, \overline{7}, \overline{11}\}
	\end{align*}
\end{ex}

\begin{ex}[H2.26]
	Sea $|G_1| = m, |G_2| = n,\ mcd(m, n) = 1$. Si $f:G_1 \to G_2$ es h. de g. sabemos que $o(f(a)) \divides o(a),\ \forall a \in G_1$. Además $o(a) \divides m \land o(f(a)) \divides n$ por el teorema de Lagrange (\ref{thm:lagrange}).
	\begin{align*}
		\begin{cases}
		o(a) \divides m \land o(f(a)) \divides n \\
		mcd(m, n) = 1 \\
		o(f(a)) \divides o(a)
		\end{cases} \implies o(a) = o(f(a)) = 1, \forall a \in G_1
	\end{align*}
	Por lo que solo puede haber un homomorfismo entre ellos y además es el trivial $f(a) = e_{G_2}$.
\end{ex}

\begin{ex}[H2.19]
	Definimos una función $f:[0, 2\pi] \subset \R \to \mathbb{S}^1, \alpha\mapsto \cos \alpha + i \sin \alpha$. Esta función tiene la propiedad de que $f(\alpha) \cdot f(\alpha') = \cos(\alpha + \alpha') + i\sin(\alpha + \alpha') = f(\alpha + \alpha')$ y por tanto es un h. de g.\footnote{A la izquierda (en $\R$) sumamos pero a la derecha (en $\mathbb{S}^1$) multiplicamos.} entre $R$ y $\mathbb{S}^1$.
	
	Un elemento de $\cos \alpha + i \sin \alpha \in \mathbb{S}^1$ es de torsión $\iff \exists n \mid (\cos \alpha + i \sin \alpha )^n = 1$. Ahora bien $(\cos \alpha + i \sin \alpha )^n = \cos n\alpha + i \sin n\alpha = 1 \iff n\alpha = k 2\pi$. 
\end{ex}

\section{Hoja 4}

\begin{ex}[H4.11] Hallar los subgrupos de Sylow de $S_5$.
	Sabemos que $|S_5| = 5! = 2^3\cdot3\cdot 5$ y por \nameref{thm:sylow1} tenemos lo siguiente:
	\begin{itemize}
		\item $\exists P_2,\ |P_2| = 2^3 = 8$.
		\item $\exists P_3,\ |P_3| = 3$. Además en $S_5$ hay $\binom{5}{3}2! =20 $ 3-ciclos y en cada $gP_3\inv{g} = \{1, a, a^2 \mid o(a) = 3\}$ hay 2 elementos de orden 3 distintos. Además, $g_1P_3\inv{g_1} \cap g_2P_3\inv{g_2} = \{e\}$ porque si su intersección fuera más grande entonces serían el mismo subgrupo (porque son cíclicos). Es por esto que tenemos que repartir 40 elementos dando 2 a cada 3-grupo con lo que obtenemos $n_3 = 20 / 2 = 10$ 3-subgrupos de Sylow en $S_5$.
		\item $\exists P_5,\ |P_5| = 5$. Además en $S_5$ hay $4! = 24$ 5-ciclos (elementos de orden 5) y en cada $gP_5\inv{g} = \{1, a, a^2, a^3, a^4 \mid o(a) = 5\}$ tenemos $4$ elementos de orden $5$ distintos. Además, $g_1P_5\inv{g_1} \cap g_2P_5\inv{g_2} = \{e\}$ porque si su intersección fuera más grande entonces serían el mismo. Así, tenemos $24$ 5-ciclos a repartir entre los diferentes $gP_5\inv{g}$ dando 4 5-ciclos a cada 1. Por tanto tenemos $n_5 = 24/4 = 6$ 5-subgrupos de Sylow en $S_5$.
	\end{itemize}
\end{ex}

\begin{ex}[H4.18]
	Demostrar que todo grupo de orden $|G| = 5^3 \cdot 7^3$ tiene un subgrupo normal de orden 125.
	
	\begin{proof}
		\nameref{thm:sylow1} $\implies \exists P_5 < G,\ |P_5| = 125$. \nameref{thm:sylow3} $\implies n_5 \divides 7^3 \land n_5 \equiv 1 \mod 5$ es decir $n_5 \in \{1, 7, 49, 343\} \land n_5 \in \{1, 6, 11, \dots\}$. Como ni $49$ ni $343$ son conguentes con $1$ módulo 5 tenemos que $n_5 = 1 \implies P_5 \normsub G$.
	\end{proof}
\end{ex}

\begin{ex}[H4.20] Hallad todos los grupos abelianos de órdenes 36, 64, 96 y 100.
	\begin{enumerate}
		\item $|G| = 36 = 2^2\cdot3^2$
		\begin{proof}
			\nameref{thm:sylow1} $\implies \exists P_2,\ |P_2| = 4 \land \exists P_3,\ |P_3| = 9$. Además $G$ abeliano $\implies P_2, P_3 \normsub G \implies G \isom P_2 \times P_3$. Estudiamos los grupos de orden 4 y de orden 9
			\begin{itemize}
				\item $|P_2| = 4$ entonces $P_2 \isom \Z/2\Z \times \Z/2\Z \lor P_2 \isom \Z/4\Z$
				\item $|P_3| = 9$ entonces $P_3 \isom \Z/3\Z \times \Z/3\Z \lor P_3 \isom \Z/9\Z$
			\end{itemize}
			Como $G \isom P_2 \times P_3$ tenemos 4 posibles grupos abelianos de orden 36.
		\end{proof}
	\end{enumerate}
\end{ex}

\begin{ex}[H4.22]
	Hallar todos los grupos abelianos de orden 175.
	
	\begin{proof}
		$|G| = 5^2\cdot 7$. Por el \nameref{thm:sylow1} tenemos que $\exists P_5, P_7 < G$ con $|P_5| = 25,\ |P_7| = 7$ y además por ser $G$ abeliano tenemos que $P_5, P_7 \normsub G \implies G \isom P_5 \times P_7$. Estudiamos los grupos de orden 25 y de orden 7:
		\begin{itemize}
			\item $|P_5| = 25 \land P_5$ abeliano $\implies P_5 \isom \Z/25\Z \lor P_5 \isom \Z/5\Z \times \Z/5\Z$. En ambos casos $P_5$ es producto directo de cíclicos pues $\Z/n\Z$ es cíclico.
			\item $|P_7| = 7 \land P_7$ abeliano $\implies P_7 \isom \Z/7\Z$. Ocurre lo mismo que con $P_5$.
		\end{itemize}
		Concluimos que $G \isom \Z/25\Z \times \Z/7\Z \lor G \isom \Z/5\Z \times \Z/5\Z \times \Z/7\Z$. Los dos casos son abelianos por ser producto directo de grupos cíclicos.
	\end{proof}
\end{ex}

\begin{ex}[H4.23]
	¿Cuántos elementos de orden 3 puede tener un grupo abeliano de orden 36?
	
	\begin{proof}
		$|G| = 36 = 2^2 3^2$. \nameref{thm:sylow1} $\implies \exists P_2, P_3 < G,\ |P_2| = 4,\ |P_3| = 9$. Estudiamos los grupos de órdenes 4 y 9:
		\begin{itemize}
			\item $|P_3| = 9 \implies P_3 \isom \Z/9\Z \lor P_3 \isom \Z/3\Z \times \Z/3\Z$
			% TODO: terminar
		\end{itemize}
	\end{proof}
\end{ex}

\section{Hoja 5}

\begin{ex}[H5.1]
	Demuestra que el conjunto $(\mathcal{C}([0, 1]), +, ·)$, donde $\mathcal{C}([0, 1]) := \{f : [0, 1] \to R \text{ continua}\}$ con las operaciones
	\begin{align*}
		(f+g)(x) = f(x)+g(x)\qquad\qquad (f\cdot g)(x) = f(x)\cdot g(x)
	\end{align*}
	es un anillo y di qué tipo de anillo es.
\end{ex}

\begin{proof}[Solución]
	Es claro que la suma es asociativa y conmutativa porque se reduce al caso numérico. Además el elemento neutro aditivo es $\0 := f(x) = 0$. El producto también es asociativo y conmutativo porque se reduce al caso numérico (para poder hacer estas reducciones es necesario que $f$ sea continua).
\end{proof}

\begin{ex}[H5.2]
	Halla las unidades en los siguientes anillos $\Z[\sqrt{-2}],\ \Z[\sqrt{-5}],\ M_2(\Q),\ M_2(\Z)$.
\end{ex}

\begin{proof}
	\begin{itemize}
		\item $\uds{M_2(\Q)} = \{A \in M_2(\Q) \mid \det A \neq 0\}$.
		\item $\uds{M_2(\Z)} = \{A \in M_2(\Z) \mid \det A = \pm 1\}$. Esto es porque $A\inv{A} = \1 \implies \det(A\inv{A}) = \1 \implies \det A = \frac{1}{\det \inv{A}}$ pero necesariamente $\det A \in \Z \implies \det A = \pm 1$.
	\end{itemize}
\end{proof}

\textbf{En lo que sigue consideramos anillos conmutativos con unidad, y supondremos que} $\1 \neq \0$.

\begin{ex}[H5.3]
	Demuestra que un cuerpo no tiene divisores de cero.
\end{ex}

\begin{proof}
	Supongamos $C$ cuerpo y $a \in C$ divisor de $\0$. Entonces $\exists b \in C \mid ab = \0$. Ahora bien, $ab \in C \implies \exists \inv{(ab)} \in C \mid (ab)\inv{(ab)} = \1$ pero entonces $\0\cdot \inv{(ab)} = \1 \implies \0 = \1$ contradicción.
\end{proof}

\begin{ex}[H5.4]
	\label{ex:h5.4}
	Demuestra que en un anillo finito, todo elemento no nulo es una unidad o bien un divisor de cero. Observa, en particular, que un anillo finito es un dominio si y solo si es un cuerpo.
\end{ex}

\begin{ex}[H5.6]
	Demuestra que si $A$ es un dominio conmutativo, entonces $A[X]$ es un dominio conmutativo.
\end{ex}

\begin{proof}
	Supongamos $A[X]$ no es un dominio conmutativo, es decir, que hay divisores de $\0$. Entonces $\exists p, q \in A[X] \mid pq = \0 \implies$ al multiplicar los coeficientes dan $\0$ con lo que nos reducimos al caso de $A$ que ya es un dominio.
\end{proof}

\begin{ex}[H5.7]
	Demuestra que el producto $R_1 \times R_2$ de anillos es un anillo (con las operaciones componente a componente). Indica cuáles son las unidades. Decide si el producto de dos cuerpos es un cuerpo.
\end{ex}

Ver \autoref{pro:prodirectodeanillos}.

\begin{ex}[H5.9]
	Se considera el anillo de los \textit{enteros algebraicos}
	\begin{align*}
		\Z^{int} = \{\alpha \in \C : p(\alpha) = 0 \text{ para algún } p \in \Z[X] \text{ mónico}\}
	\end{align*}
	Demuestra lo siguiente
	\begin{enumerate}
		\item $Z^{int} \cap \Q = \Z$
		\begin{proof}
			Veremos que las raíces racionales de $p$ son en realiad enteras. Sean $p, q \in \Z$ coprimos con $q \neq 0$. Si $p(\frac{p}{q}) = 0$ se tiene
			\begin{align*}
				p(x) = x^n + \dots a_1x + a_0 = 0 \\
				p(\frac{p}{q}) = \left(\frac{p}{q}\right)^n + \dots + a_1 \left(\frac{p}{q}\right) + a_0 = 0 \\
				p^n + \dots + a_n p q^{n-1} + a_0 q^n = 0 \\
				p^n = q(a_{n-1} + q a_{n-2} + \dots + a_0 q^{n-1}) \\
				\implies q = 1 \implies \frac{p}{q} \in \Z \implies Z^{int} \cap \Q = \Z
			\end{align*}
		\end{proof}
		\item el elemento 2 de $\Z^{int}$ no es una unidad ni es irreducible
		\begin{itemize}
			\item Para que $2$ sea unidad tiene que tener inverso multiplicativo en $\Z^{int}$. No es una unidad ya que su inverso es $\frac{1}{2} \in \Q \implies \frac{1}{2}\notin \Z^{int}$ por el apartado anterior.
			\item 2 no es irreducible si encontramos dos $a,b \in \Z^{int}$ tales que $ab = 2$ y $a,b \notin \uds{\Z^{int}}$. %TODO
		\end{itemize}
	\end{enumerate}
\end{ex}

\begin{ex}[H5.10]
	Demuestra que $\Q[i] = \{a+bi:a,b\in \Q\}$ es un subcuerpo de $\C$.
\end{ex}


\begin{ex}[H5.13]
	Demuestra que todo subanillo de un cuerpo es un dominio.
\end{ex}

\begin{proof}
	En un cuerpo no hay divisores de $\0$ por tanto cualquier subconjunto no los tendrá. Por la \autoref{dfn:dominiointegridad} el subanillo es un DI.
	%TODO: confirmar con turroduda
\end{proof}

\begin{ex}[H5.14]
	Sea $R$ un anillo e $I$ un ideal de $R$. Demuestra que los siguientes subconjuntos de $R$ son ideales de $R$.
	\begin{enumerate}
		\item $Rad(I):=\{a\in R:a^n \in I \text{ para algún } n \in \N\}$ (el radical de $I$);
		\begin{proof}
			Ver radical de un ideal en Wikipedia.
		\end{proof}
		\item $Ann(I):= \{a \in R \mid ax = \0, \text{ para todo } x \in I\}$
		\begin{proof}$ $ \newline
			\begin{enumerate}
				\item $\0 \in Ann(I)$. Sí porque $\0x = \0 \forall x \in I$
				\item $Ann(I)$ es cerrado por la suma. Sean $a,b \in Ann(I)$ entonces $ax = \0, bx = \0 \forall x \in I \implies ax + bx = \0 \implies (a+b)x = \0 \implies a+b \in  Ann(I)$.
			\end{enumerate}
		\end{proof}
	\end{enumerate}
\end{ex}

\begin{ex}[H5.15]
	Sea $R$ un anillo conmutativo con unidad. Demuestra lo siguiente:
	\begin{enumerate}
		\item Si $I$ es un ideal de $R$ entonces $I = R \iff$ existe una unidad de $R$ en $I$.
		\begin{proof}
			$\exists u \in \uds{R},\ u \in I \implies I = R$. Por ser $I$ ideal tenemos que $\forall a \in \R, \forall i \in I, ai \in I$. Tomando $a = u$ tenemos que $au \in I$. Como esto vale $\forall a \in R$ tenemos que también valdrá para $a = \inv{u} \in R$ por ser $u$ unidad. Por tanto $u\inv{u} = \1 \in I$. Ahora podemos repetir el proceso para $i = \1$. Obtenemos que $\forall a \in R,\ a\1 \in I \implies I = R$.
		\end{proof}
		\item $R$ es un cuerpo $\iff \{\0\}$ es el único ideal propio de $R$.
		\begin{proof}
			Utilizamos el apartado anterior. Si $R$ es un cuerpo entonces todo elemento excepto el $\0$ tiene inverso, es decir, $\forall a \in R, \exists \inv{a} \in R$. Si $I$ ideal en $R$ contiene a cualquier elemento distinto del $\0$ por el apartado anterior tenemos que $I=R$.
		\end{proof}

	\end{enumerate}
\end{ex}

\begin{ex}[H5.17]
	Encuentra todos los ideales maximales en $\Z/8\Z, \Z/10\Z$ y en $\Z/n\Z$.
\end{ex}

\begin{ex}[H5.21]
	Halla el cuerpo de fracciones de los siguientes dominios:
\end{ex}

Ver \cite[p.~208]{dor96}.

\begin{ex}[H5.22]
	
\end{ex}
