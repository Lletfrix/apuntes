% !TeX root = ../apuntes-ea.tex

\chapter{El teorema de Cauchy}

\section{Consideraciones previas}

\subsection{Centro de un grupo}

\begin{dfn}[Centro de un grupo]
	\label{dfn:centro}
	Sea $G$ un grupo finito. Definimos el centro de $G$, $Z(G) = \{a \in G \mid \forall g \in G,\ ag = ga\}$.
\end{dfn}

El centro es útil en grupos finitos no abelianos.

\begin{pro}
	Sean $a, b \in Z(G)$. Entonces $ab \in Z(G)$.
\end{pro}

\begin{proof}
	Tenemos que $ag = ga$ y que $bg = gb$. Ahora tenemos que probar que $g(ab) = (ab)g$. Es trivial manipulando $(ab)g = agb = gab$.
\end{proof}

\begin{pro}
	\label{pro:centronormal}
	Sea $G$ un grupo. $Z(G)$ es un subgrupo y además es un subgrupo normal.
\end{pro}

\begin{proof}
	$\forall g \in G,\ Z(G)g = \{ag \mid a \in G \land \forall b \in G,\ ab = ba\} = \{ga \mid a \in G \land \forall b \in G,\ ab = ba\} = gZ(G)$.
\end{proof}

\begin{pro}
	\label{pro:subcentronormal}
	Si $H < Z(G)$ entonces $H$ es abeliano y normal.
\end{pro}

\begin{pro}
	Sea $g \in G,\ \phi_g: G \to G$ el isomorfismo definido por $\phi_g(x) = gx\inv{g}$. Entonces
	\begin{align*}
	x \in Z(G) &\iff \forall g \in G, gx = xg \iff gx\inv{g} = x \\
	x \in Z(G) &\iff \forall g \in G,\ \phi_g(x) = x 
	\end{align*}
\end{pro}

\begin{pro}
	$G$ es abeliano $\iff G = Z(G)$
\end{pro}

Sea $a \in G \land o(a) = n$. Si $a$ es el único elemento de orden $n$ entonces $n = 2 \land a \in Z(G)$. Probamos primero que $n=2$. Si $a$ es el único elemento de orden $n$ entonces tiene que ocurrir que $a$ y $a^{n-1}$ tienen el mismo orden por lo que $1 = n-1 \implies n = 2$.

\begin{pro}
	\label{pro:triplecentro}
	Si $G/Z(G)$ es cíclico de orden $n$ entonces $n = 1$. Otra manera de formularlo: Si $G/Z(G)$ es cíclico, entonces $G = Z(G)$. Otra manera más de formularlo: si $G/Z(G)$ es cíclico entonces $G$ es abeliano.
\end{pro}

\begin{proof}
	Supongamos que $G/Z(G) \isom \ZnZ$. Vamos a probar que $n$ tiene que ser 1. Supongmos que $G/Z(G) = \{\overline{\alpha_i}, i = 1, \dots, n\}$ donde $\overline{\alpha_i} = \alpha^i Z(G)$. Fijamos $g \in G$ con $g = \alpha^j h,\ h \in Z(G),\ 0 \leq j < n$ y fijamos $f' \in G$ con $g' = {\alpha^j}' h',\ h' \in Z(G),\ 0 \leq j' < n$. Entonces $gg' = \alpha^j h{\alpha^j}' h' = \alpha^{j+j'}hh' = {\alpha^j}' h' \alpha^j h = gg'$ (podemos conmutar las $h$ con cualquier elemento porque $h \in Z(G)$, por el contrario, los $\alpha$ no necesitamos conmutarlos, solo agruparlos cuando están juntos). Es decir, que $\forall g, g' \in G$ tenemos que $gg' = g'g$ por lo que $G$ es abeliano.
\end{proof}


\begin{ex}[H1.33]
	Sea $G$ un grupo. Suponed que existe un único $a \in G$ de orden 2. Demostrad que $a \in Z(G)$.
\end{ex}

\begin{proof}
	Recordamos que $a \in Z(G) \iff ga = ag,\ \forall g \in G$. Definimos el isomorfismo de conjugación $\phi_g (x) = gx\inv{g}$ para algún $g$. Como $\phi_g$ es isomorfismo lleva elementos de orden $n$ en elementos de orden $n$. Entonces $\phi_g(a) = a$ ya que $a$ es el único elemento de orden 2. Por tanto $ga\inv{g} = a \implies ga = ag \implies a \in Z(G)$.
\end{proof}

\subsection{Centralizador de un elemento}

\begin{dfn}
	[Grupo de automorfismos]
	Sea $G$ un grupo. Llamamos grupo de automorfismos al grupo
	\begin{align}
	\autom{G} = \{f \mid f: G \to G \text{ isomorfismo}\}
	\end{align}
\end{dfn}

\begin{pro}
	La función $\gamma: G \to \autom{G}$ definida con $\gamma(g) \mapsto \gamma_g$, donde $\gamma_g : G \to G, \gamma_g(x) = gx\inv{g}$, es un homomorfismo.
\end{pro}

\begin{proof}
	Verifica la definición: para $g,g' \in G$
\end{proof}

\begin{dfn}[Elementos conjugados]
	\label{dfn:elementosconjugados}
	Sean $a,b \in G$. Decimos que $a$ y $b$ son conjugados $\iff \exists g \in G \mid \gamma_g(a) = b$.
\end{dfn}

\textbf{Nota:} La relación de conjugación solo merece la pena en grupos no abelianos, porque en un grupo abeliano, cualquier par de elementos es conjugado.

\begin{ej}
	En $S_3$ afirmamos lo siguiente:
	\begin{itemize}
		\item que $1$ solo tiene como conjugado a sí mismo,
		\item que $\{(12),(13),(23)\}$ son conjugados entre sí,
		\item y que $\{(123),(132)\}$ también son conjugados entre sí.
	\end{itemize}
	Es decir, que la conjugación nos genera una partición con 3 cajas disjuntas.
\end{ej}

\begin{pro}
	La relación de conjugación es una relación de equivalencia $aRb \iff a$ y $b$ son conjugados.
\end{pro}

\begin{proof}
	Comprobamos que $R$ es una relación de equivalencia:
	\begin{enumerate}
		\item Reflexiva: $\forall a \in R, aRa$: tomamos $g = e$ y automáticamente tenemos que $ea\inv{e} = a$.
		\item Simétrica: $\forall a,b \in R,\ aRb \implies bRa$: $\exists g, g a \inv{g} = b$. Tomamos $\gamma_{\inv{g}}$ y tenemos que $\gamma_{\inv{g}}(b) = a \implies bRa$.
		\item Transitiva: $\forall a,b,c \in G,\ aRb \land bRc \implies aRc$. Por hipótesis tenemos que $\exists g \in G \mid \gamma_g(a) = b \land \exists g' \in G \mid \gamma_{g'}(b) = c$. Por tanto $\gamma_{gg'}(a) = (\gamma_{g'} \gamma_g)(a) = \gamma_{g'}(b) = c$.
	\end{enumerate}
\end{proof}



En esta relación de equivalencia, las clases de equivalencia son de la forma $\overline{a} = \{ga\inv{g} \mid g \in G\}$ (conjuntos de los elementos que son conjugados de $a$). Queremos saber cuántos elementos hay en cada clase de equivalencia.

Fijamos $a \in G$ y definimos

\begin{dfn}[Centralizador de un elemento]
	\label{dfn:centralizador}
	Sea $a \in G$. Llamamos centralizador de $a$ al conjunto
	\begin{align}
	C(a) = \{g \in G \mid \gamma_g(a) = g a \inv{g} = a\}
	\end{align}
	Se tiene que $\forall a \in G,\ e \in C(a)$, es decir que $C(a)$ no es vacío.
\end{dfn}

\begin{pro}
	$a \in Z(G) \iff C(a) = G \iff [G:C(a)] = 1$
\end{pro}

\begin{proof}
	Es cristalina de las definiciones.
\end{proof}

\begin{pro}
	$C(a)$ es un subgrupo de $G$
\end{pro}

\begin{proof}
	Por el teorema \ref{thm:subconjuntocerrado} solo necesitamos probar la clausura, es decir, tenemos que probar que $\forall g,g' \in G,\ g \in C(a) \land g' \in C(a) \implies gg' \in C(a)$. Sale solo $(gg')a\inv{gg'} = gg'a\inv{(g')}\inv{g} = ga\inv{g} = a \in C(a)$.
\end{proof}

\begin{pro}
	\label{pro:cardinalcajas}
	$|\{ga\inv{g} \mid g \in G\}| = [G:C(a)] = r$ (el número de elementos de una clase de equivalencia es el índice de un representante)
\end{pro}

\begin{proof}
	Fijamos $a \in G$ y definimos $H = C(a) = \{g \in G \mid ga\inv{g} = a\}$.
\end{proof}



\section{Teorema de Cauchy}

\begin{thm}[de Cauchy]
	\label{thm:cauchy}
	Sea $G$ un grupo finito con $|G| = n$. Si $p$ es primo y $p\divides n$ entonces $G$ contiene un elemento de orden $p$.
\end{thm}

\begin{proof}
	Procedemos por casos:
	\begin{itemize}
		\item Si $G$ es abeliano. Descomponemos $|G| = n = p_1^{\alpha_1}p_2^{\alpha_2}\dots p_s^{\alpha_s}$. Por el teorema \ref{thm:noprobado1}, $G \isom \Z/p_1^{\beta_1}\Z \times \Z/p_2^{\beta_2}\Z \times \dots \times \Z/p_s^{\beta_r}\Z$ donde cada $\alpha_i$ es la suma de algunos $\beta_r \qed$.
		
		\item Si $G$ no es abeliano. Particionamos $G$ con la relación de equivalencia dada anteriormente (definición \ref{dfn:elementosconjugados}), $aRb \iff \exists g \in G \mid ga\inv{g} = b$. Recordemos que cada clase de equivalencia es de la forma $\overline{c} = \{gc\inv{g} \mid g \in G\}$. Observamos que si partimos de $e$, el elemento neutro, $eRb \implies \exists g \mid ge\inv{g} = b$ pero $\forall g \in G,\ ge\inv{g} = e$ por lo que $cl(e)$ tiene un único elemento.
		
		Tomemos ahora una clase de equivalencia, la que contenga a $a \in G$. La clase es $cl(a) = \{ga\inv{g} \mid g \in G\}$. Es claro que $a \in \overline{a}$ por la propiedad reflexiva de $R$, luego por lo menos en $cl(a)$ tiene un elemento.
		
		\begin{align*}
		cl(a) = \{ga\inv{g} \mid g \in G\} = \{a\} &\iff ga\inv{g} = a,\ \forall g \in G \\
		&\iff ga = ag,\ \forall g \in G
		\end{align*}
		\begin{align*}
		|cl(a)| = 1 &\iff \overline{a} = 1 \\
		&\iff a \in Z(G)
		\end{align*}
		
		Supongamos que la partición está dada por subconjuntos $cl(a_1), cl(a_2), \dots, cl(a_s)$. Por ser una partición, cualquier elemento vive en una sola caja, luego para saber cuantos elementos tiene $G$ nos vale con sumar los elementos de cada caja:
		\begin{align*}
		|G| = \sum_{i = 1}^{s} |cl(a_i)| = \sum_{i = 1}^n |\{ga_i\inv{g} \mid g \in G\}|
		\end{align*}
		Ahora bien, por la proposición \ref{pro:cardinalcajas} tenemos que $|cl(a_i)| = [G:C(a_i)]$. Por tanto decir que $|cl(a_i)| = 1 \implies [G:C(a_i)] = 1 \implies G = C(a_i)$.
		
		Ahora vamos a dividir el sumatorio en dos: por un lado las cajas de un solo elemento y luego las cajas de varios elementos:
		\begin{align}
		\label{eq:thmcauchy}
		|G| = |Z(G)| + \sum_{i = r + 1}^{s} [G : C(a_i)] \text{ donde } |Z(G)| = r \text{ y } [G : C(a_i)] \geq 2, \forall i = r+1,\dots, s
		\end{align}
		
		Ahora para probar el teorema de Cauchy procedemos por inducción en $n = |G| = [G:C(a_i)]\cdot |C(a_i)|$.
		
		\begin{enumerate}
			\item Caso $n = 1$. $G = \{e\}$ que es obvio.
			\item Caso $n = 2$. Son grupos cíclicos por lo que $\exists \alpha \in G \mid o(\alpha) = 2$.
			\item Caso $n \implies n+1$. Pueden pasar dos cosas:
			\begin{itemize}
				\item o bien $p \divides |C(a_i)|$ para algún $i = r+1, \dots, s$ entonces, por hipótesis inductiva, $C(a_i)$ contiene algún elemento de orden $p$. Pues ya está: $C(a_i) < G$ porque $\alpha \in C(a_i) \mid o(\alpha) = p \implies \alpha \in G$ también). \qedsymbol
				
				\item o bien $p \not\divides |C(a_i)|,\ \forall i = r+1,\dots,s$. No podemos proceder por inducción. Por hipótesis $|G| = [G:C(a_i)]\cdot |C(a_i)| \land p \divides |G| \implies p \divides [G: C(a_i)],\ \forall i = r+1,\dots, s$.
				
				Como $|G| = |Z(G)| + \sum_{i = r + 1}^{s} [G : C(a_i)]$ y por hipótesis $p \divides |G| \land p \divides [G : C(a_i)], \forall i = r+1,\dots,s \implies p \divides |Z(G)| \implies |Z(G)|$ es múltiplo de $p$. Como $Z(G)$ es abeliano, $\exists \alpha \in Z(G) \mid o(\alpha) = p$. Luego se reduce al caso abeliano y ya estaría \qedhere
			\end{itemize}
		\end{enumerate}
	\end{itemize}
\end{proof}

\begin{ej}
	Sea $G$ tal que $|G| = pq$. Entonces por le teorema de Cauchy $\exists a,b \in G \mid o(a) = p \land o(b) = q$. Como $p$ y $q$ son primos los ordenes de $\langle a \rangle$ y $\langle b \rangle$ son coprimos y por tanto $\langle a \rangle \cap \langle b \rangle = \{e\}$. Por el teorema del orden de conjunto\footnote{No sabemos si alguno es normal, luego no tenemos garantías de que el producto sea un grupo} producto libre (\ref{thm:cardinalidadproductolibre}), $|\langle a \rangle \langle b \rangle| = pq$. Lo que si que sabemos es que $G = \{a^ib^j \mid 0 \leq i < p -1 \land 0 \leq j < q - 1\} = \langle a, b \rangle$.
\end{ej}

\begin{ej}
	Sea $G$ tal que $|G| = 2q$. Análogamente al caso anterior llegamos a que $o(a) = 2$. Como $\langle b \rangle$ tiene índice 2 entonces $\langle b \rangle \normsub G$. Esto nos permite saber como operar con las palabras $a^ib^j$ una vez tenemos un isomorfismo que lleva $a b \inv{a} = b^j$ (tiene que ir a algún $b^j$ porque por ser isomorfismo tiene que llevar elementos de orden $q$ en elementos de orden $q$: los $b \in \langle b \rangle$)
\end{ej}

Dada la relación de equivalencia de conjugación (definición \ref{dfn:elementosconjugados}), definimos $C$ como el conjunto de los representantes de las clases de equivalencia. Entonces podemos decir
\begin{align*}
G = \bigcup_{c_i \in C} \{a \in G \mid a R c_i\}
\end{align*}
Observemos que $d \in Z(G) \iff \{a \in G \mid a R d\} = \{gd\inv{g} \mid g \in G\} = \{d\}$. Y por tanto podemos escribir
\begin{align*}
C = Z(G) \cup (C\setminus Z(G))
\end{align*}
que aunque pareza obvio quiere decir que $C$ se puede expresar como la unión disjunta de las cajas que tienen solo un elemento que se corresponden con elementos que están en el centro y las cajas que tienen más de uno. Y por lo visto en la demostración del teorema de Cauchy tenemos que
\begin{align*}
|G| = \sum_{c_i \in C} | \overline{c_i} | = |Z(G)| + \sum_{i = r + 1}^{s} [G : C(a_i)] \text{ donde } [G : C(a_i)] \geq 2
\end{align*}

\section{P-grupos}

\begin{dfn}[P-grupo]
	Sea $p$ primo. Decimos que $G$ es un p-grupo si $|G| = p^r$.
\end{dfn}

Nos interesan sobre todo los p-grupos no abelianos

\begin{thm}
	Si $G$ es un p-grupo entonces $Z(G)$ es no trivial (no es el vacío).
\end{thm}

\begin{proof}
	Podemos escribir sin distinguir entre cajas de uno o varios elementos
	\begin{align*}
	|G| = |C(c_i)||[G:C(c_i)]|
	\end{align*}
	es decir que tenemos una factorización de $|G| = p^r$ luego $|C(c_i)|$ y $|[G:C(c_i)]|$ son ambos potencias de $p$. Y aplicando esto a la expresión \ref{eq:thmcauchy} tenemos que
	\begin{align*}
	\underbrace{|G|}_{\text{múltiplo de p}} = |Z(G)| + \sum_{i = r + 1}^{s} \underbrace{[G : C(a_i)]}_{\text{múltiplo de p}} \text{ donde } [G : C(a_i)] \geq 2
	\end{align*}
	por lo que $|Z(g)|$ tiene que ser múltiplo $p$ por lo que $Z(G)$ no puede ser el trivial.
\end{proof}

\begin{ej}
	Tenemos que $Z(D_4) = \{1,B^2\}$ y $Z(H) = \{1, B^2\}$ donde $H$ es el grupo de cuateriones (ejemplo \ref{ej:grupocuaterniones}) y $D_4$ es el famoso grupo (ejemplo \ref{ej:famosogrupod4}). 
\end{ej}

\begin{pro}
	\label{pro:primocuadradoabeliano}
	Si $p$ es primo y $|G| = p^2$ entonces $G$ es abeliano.
\end{pro}

\begin{proof}
	Por el la demostración del teorema anterior tenemos que o bien $|Z(G)| = p$ o bien $|Z(G)| = p^2$. Afirmamos que $|Z(G)| \neq p$ ya que si fuera así $|G/Z(G)| = p \implies G/Z(G)$ cíclico pero hemos probado (proposición \ref{pro:triplecentro}) que $G/Z(G)$ no puede ser cíclico. Por lo tanto $|Z(G)| = p^2 \implies Z(G) = G \implies G$ es abeliano.
\end{proof}

% ---------------- después del parcial 1
% 20181009

\hr

Sea $\sim$ una relación de equivalencia definida por $a\sim b \iff \exists g \in G \mid ga\inv{g} = b$ para $a,b \in G$. Esta relación da una partición de $G$ en clases de la forma $cl(a) = \{ga\inv{g} \mid g \in G\}$. En el caso abeliano esta relación es la de igualdad, por lo que no nos merece la pena liar este pifostio para saber que $a\sim b \iff a = b$. 

Es muy importante saber cómo contamos los elementos de una clase, es decir, de cuantas formas podemos \textit{mover} el elemento $a$ con $g \in G$. Para ello definimos el centralizador (definición \ref{dfn:centralizador}) como $C(a) = \{h \in G \mid ha\inv{h} = a\} < G$. Queremos probar que $|cl(a)| = [G:C(a)] = r$.

Lo probamos tomando clases laterales a la izquierda (por ejemplo) y partiendo $G$ en $r$ cajas. Las cajas son de la forma $\alpha_iC(a),\ i = 1, \dots, r$. Esta partición no tiene que ver con la partición anterior. Observemos que para cualquier $g \in \alpha_i C(a), g = \alpha_i h$, tenemos que $g a \inv{g} = \alpha_i h a \inv{h} \inv{\alpha_i} = \alpha_i a \inv{\alpha_i}$ es decir que los $g \in C(a)$ no se mueven fuera de la caja. Es decir, que si $\alpha_i \neq \alpha_j$ para $i\neq j$ entonces hay $r$ maneras de mover a $g$ y por tanto $|cl(a)| = r$.

Probaremos que en efecto los $\alpha_i$ son distintos.

Sean $g_1, g_2 \in G$. $g_1a\inv{g_1} = g_2a\inv{g_2} \iff (\inv{g_2}g_1)a(\inv{g_1}g_2) = a \iff (\inv{g_2}g_1)a\inv{(\inv{g_2}g_1)} \iff C(a) \inv{g_2}g_1 \in C(a) \iff g_1 \in g_2C(a)$.

Si $G/\sim$ tiene $N$ elementos, tomamos $\{c_1, \dots, c_N\}$ como el conjunto de los representantes, donde $c_i$ es un representante de cada conjunto de la partición. Entonces pordemos expresar
\begin{align*}
G = \bigcup_{c_i \in C} = cl(c_i)
\end{align*}
donde $|cl(c_i)| = [G:C(c_i)]$. Por tanto decir que $|cl(c_i)| = 1$ es equivalente ($\iff$) a decir que $G = C(c_i) = \{\forall g \in G,\ gc\inv{g} = c\} \iff c \in Z(G)$.

Afirmábamos que
\begin{align*}
|G| = \sum_{c_i \in C} |cl(c_i)| = |Z(G)| + \sum_{c_i \in C\setminus Z(G)} [G:C(c_i)]
\end{align*}
descomponiendo la suma en las clases con solo un elemento y las clases con más de dos elementos.

\hr

\begin{ej}
	Consideramos $D_3$ (ver ejemplo \ref{ej:grupod3}). Nos fijamos en que $B \not\in Z(D_3)$ es decir que en $cl(B)$ hay más de un elemento. En particular por lo visto anteriormente $|cl(B)| = [G:C(B)]$. Ahora bien $C(B) = \{1, B, B^2\}$ luego $|cl(B)| = [G:C(B)] = 2$. La pregunta es ¿quién es el compañero de $B$ en su clase? Es fácil, recordamos que $\phi_g (x) = gx\inv{g}$ (el isomorfismo conjugación) es un isomorfismo y que $\{1, B, B^2\}$ es normal, por lo que $o(B) = o(\phi_g(B)) = 2$. Entonces $\phi_g(B) \neq 1$ porque no coinciden los órdenes, de manera que $\phi_g(B) = B^2$ por necesidad. Luego el otro elemento es el $B^2$.
	
	¿Qué pasa con el elemento $A$? Pues ocurre que $A \in C(A)$ y $\{1, A\} \in C(A)$ y en realidad no puede haber más porque si hubiese un tercero, $\{1, A\}$ es un subgrupo de orden 2 $\implies o(\{1, A\})$ no divide a 3 $\implies$ si hubiese más, $C(A) = D_3$ y eso no puede ser $\implies C(A) = \{1, A\} \implies |cl(A)| = [D_3:C(A)] = 6/2 = 3$. Como las clases son disjuntas los tres elementos sobrantes forman la última caja.   
	
	Para conlcuir queda que la relación $\sim$ parte $D_3$ en 3 cajas, a saber:
	\begin{align*}
	D_3 = \{\underbrace{1}, \underbrace{B, B^2}, \underbrace{A, AB, AB^2}\}
	\end{align*}
\end{ej}


\begin{ej}
	\label{ej:clasesd4}
	El caso del famoso grupo $D_4$ (ver ejemplo \ref{ej:famosogrupod4})es mucho más interesante porque $Z(D_4)$ no es trivial. Elegimos por ejemplo el elemento $B^2$. Probar que $\phi_g(B^2) = gB^2\inv{g} = B^2,\ \forall g \in D_4$ es complicado. Pero fijémonos en que $\phi_B(B^2) = BB^2\inv{B} = B^2$ y que $\phi_A(B^2) = AB^2\inv{A} = B^2$. Entonces cualquier palabra en $A$ y en $B$ no mueve a $B^2$, por ejemplo $AB(B^2)\inv{B}\inv{A} = B^2$. Nos convencemos de que $B^2 \in Z(D_4)$. Con esto ya tenemos que $|Z(D_4)| \geq 2$ (puesto que de momento ya sabemos que $1, B^2 \in Z(G)$. Podría ser entonces $|Z(D_4)| = 4, 8$ (probamos los divisores de $|D_4|$). Como $D_4$ no es abeliano, es claro que $|Z(D_4) \neq 8$. Tampoco puede ser $|Z(D_4) \neq 4$ porque si tuviera 4, el cociente $D_4/Z(G)$ tendría orden $2$ y por tanto sería cíclico. Pero ya hemos probado que $G/Z(G)$ no puede ser cíclico (ver proposición \ref{pro:triplecentro}). Luego ya sabemos que $Z(D_4) = \{1, B^2\}$.
	
	Vamos a seguir sacando cajas. Veamos $cl(B)$. Claramente $B \in C(B)$ y por alguna razón que me falta $C(B) = \{1, B, B^2, B^3\}$. Por la fórmula tenemos que $|cl(B)| = [D_4:C(B)] = 2$. Tenemos una vez más que utilizar el isomorfismo de conjugación. Sabemos que $cl(B) = \{ga\inv{g} \mid g \in G\}$. Pero al ser $\phi_g$ isomorfismo y $\langle B \rangle$ normal, tenemos que $\phi_g : \langle b \rangle \to \langle b \rangle$ también es isomorfismo y por tanto lleva elementos de orden $n$ en elementos de orden $n$. Por tanto $\phi_g(B) = gB\inv{g}$ solo puede ser $B^3$ (a parte de $B$). Luego ya tenemos que $cl(B) = \{B, B^3\}$.
	
	¿Qué pasa con $A$? Pues es claro que $C(A) \supset \{1, A, B^2, AB^2\}$ ya que $B^2 \in Z(G)$ por lo que está en todos los $C(c_i)$.
	
	
	Segundo intento.
	
	\begin{enumerate}
		\item Como siempre $cl(e) = \{e\}$
		\item Veamos $cl(B)$. Queremos ver cuántos elementos tiene. Sabemos que $|cl(B)| = [D_4:C(B)]$. Veamos quién es $C(B)$. En primer lugar $B \in C(B) \implies \gen{B} \in C(B)$. Así ya tenemos que $|C(B)| \geq 4$. ¿Puede haber algún elemento más en $C(B)$? No, porque si hubiera uno más, su orden ya sería $|C(B)| = 8$ pues $C(B) < D_4$. Así concluímos que $|cl(B)| = [D_4:C(B)] = 8 / 4 = 2$. Además sabemos que $[D_4:C(B)] = 2 \implies C(B) \normsub D_4 \iff gC(B)\inv{g} = C(B) \forall g \in D_4 \implies gB\inv{g} \in C(B)$. Además como $gx\inv{g}$ es un isomorfismo que lleva elementos de orden $n$ en elementos de orden $n$ obtenemos que $o(gB\inv{g}) = o(B) = 2$. Sabemos que $B \in cl(B) \land cl(B) = \{gB\inv{g} \mid g \in D_4\} \land gB\inv{g} \in C(B) \implies gB\inv{g} = B^3$. Por tanto $cl(B) = \{B, B^3\}$.
		\item Veamos $cl(A)$. Queremos ver cuántos elementos tiene. Sabemos que $|cl(A)| = [D_4:C(A)]$. Veamos quién es $C(A)$. En primer lugar $A \in C(A) \implies \gen{A} \subset C(A)$. Si $B \in C(A)$ entonces $C(A) = G$ pues $B$ y $A$ generan. Esto no puede ser porque $C(A) = G \implies A$ conjuga con todos los demás elementos pero sabemos que $AB \neq BA$. Ocurre lo mismo con $B^3$. Probamos con $B^2$. $B^2AB^2 = BBAB^2 = BAB^3B^2 = BAB = AB^3B = A$ luego $B^2 \in C(A)$. Como $C(A) < D_4$ sabemos que es cerrado y por tanto $AB^2 \in C(A)$. Ya no puede haber más elementos porque si hubiera más, entonces $|C(A)| = 8$ y eso no puede ser. Por tanto $|cl(A)| = [D_4:C(A)] = 8 / 4 = 2$. Sabemos que $A \in cl(A)$. ¿Quién es el otro elemento? Como antes, $[D_4:C(A)] = 2 \implies C(A) \normsub D_4 \iff gC(A)\inv{g} = A$. Como $gx\inv{g}$ es un isomorfismo mantiene el orden y por tanto los conjugados de $A$ pueden ser $B^2$ o $AB^2$ (los únicos de orden 2 en $C(A)$
	\end{enumerate}
\end{ej}


\hr

% 20181011

Vez pasada tomábamos $a \in G$ y teníamos $cl(a) = \{g a \inv{g} \mid g \in G\} = \{a=a_1, a_2, \dots, a_r\}$ y $C(a) = \{g \in G \mid ha\inv{h} = a \}$. Concluíamos que $|cl(a)| = [G:C(a)]$.

Vamos a generalizar al caso $S \subset G,\ S \neq \emptyset$. Consideramos la familia de subconjuntos siguiente:
\begin{align*}
\{gS\inv{g} \mid g \in G\} = \{S = S_1, S_2, \dots, S_r\}
\end{align*}
que tiene $r$ subconjuntos distintos.

Recordemos que la conjugación dada $\phi_g(x) = gx\inv{g}$ (el isomorfismo conjugación) es un isomorfismo\footnote{A veces tomate frito llama a este isomorfismo $\gamma_g$}, y por tanto una biyección entre subconjuntos $S_i \subset G$. Por tanto $|S| = \phi_g(S)$.

\begin{dfn}[Normalizador de un subgrupo]
	\label{dfn:normalizador}
	Fijado $S \subset G$, definimos el normalizador de $S$:
	\begin{align}
	N(S) = \{g \in G \mid gS\inv{g} = S\}
	\end{align} 
\end{dfn}

Se parece mucho a la definición de centralizador de un elemento (\ref{dfn:centralizador}). En el caso en que $S = \{a\}$ tenemos que $N(S) = \{g \in G \mid ga\inv{g} = a\} = C(a)$.

Ojo, decir que $gS\inv{g} = S$ no significa que $\forall b_i \in S,\ gb_i\inv{g} = b_i$, sino que $gb_i\inv{g} \in S$ (no mandamos cada elemento a él mismo, sino que todos quedan dentro del subconjunto). Es decir que \textit{$N(S)$ es el conjunto de la totalidad de elementos para los que $\phi_g$ manda el subconjunto $S$ en sí mismo.}

\begin{pro}
	Dado $S \subset G,\ N(S)$ es un subgrupo.
\end{pro}

\begin{proof}$ $\newline
	Como $G$ es finito, $N(S)$ es subgrupo $\iff S \neq \emptyset \land N(S)$ es cerrado por la operación.
	\begin{itemize}
		\item Es claro que $e \in N(S)$ pues $eS\inv{e} = S$, luego $N(S) \neq \emptyset$.
		\item Tenemos que probar la clausura. Si $h_1S\inv{h_1} = S \land h_2S\inv{h_2} = S$ tenemos que $\underbrace{(h_2S\inv{h_2})}_{\in S}\inv{h_1} = S \implies h_1h_2 \in N(S)$.
	\end{itemize}
\end{proof}

\begin{pro}
	\label{pro:propiedad2Ns}
	$\{gS\inv{g} \mid g \in G\} = \{S = S_1, S_2, \dots, S_r\}$ son $r$ subconjuntos distintos. Es decir que $r = [G: N(S)]$.
\end{pro}

\begin{proof}
	A la izquierda del lector.\footnote{Left to the reader.}
\end{proof}

Supongamos ahora que en vez de ser $S \subset G$, tomamos $S < G$. Recordemos que dado $g\in G$, $\phi_g$ es un isomorfismo por tanto manda elementos de un subgrupo en otro subgrupo (si el subgrupo es normal, manda elementos de un subgrupo en sí mismo).

\begin{pro}
	$H < N(H)$
\end{pro}

\begin{proof}
	Si tomamos $h \in G$, tenemos que $hH\inv{h} = H$ y también $\inv{h}H\inv{(\inv{h})} = H$ (todo elemento de $H$ tambéin tiene a su inverso en $H$).
\end{proof}

\begin{thm}
	Sea $G$ grupo, $H < G$. Entonces $H \normsub N(H)$ y $N(H)$ es el mayor subgrupo de $G$ con esta propiedad, es decir, $H \normsub H' \implies H' < N(H)$.
\end{thm}

\begin{proof}$ $\newline
	\begin{itemize}
		\item Para probar que $N\normsub N(H)$ tiene sentido olivdarse del grupo $G$. Tenemos que $h \in N(H) \iff hH\inv{h} = H, \forall h \in G$. En particular, tenemos que $hH\inv{h} = H,\ \forall h \in N(H) \implies H$ es normal en $N(H)$.
		
		\item Para porbar que $N(H)$ es el mayor subgrupo con esta propiedad demostraremos que si $H < H'$ y $H \normsub H'$ entonces $H' \subseteq N(H)$. La demostración es casi una tautología. Tenemos que $\forall h' \in H',\ h'H\inv{h'} = H \implies \forall h' \in H',\ h' \in N(H) \implies H' \subset N(H)$.
	\end{itemize}
\end{proof}

\begin{cor}
	$H \normsub G \iff N(H) = G$
\end{cor}

\begin{proof}
	Sabemos que $H\normsub H = \{gH\inv{g} \mid g \in G\}$ y dicho conjunto tiene $[G:N(H)] = 1$ elementos, luego $N(H) = G$. En otras palabras, el normalizador de un subgrupo $H < G$ normal es todo el grupo $G$.
\end{proof}

\begin{pro}
	$Z(G) < N(H)$
\end{pro}

\begin{proof}
	Por definición de $Z(G)$ tenemos que los elementos $g \in Z(G)$ fijan no solo los elementos dentro de subconjuntos, sino que los fijan uno a uno. Por lo que es claro que $Z(G) < N(H)$. 
\end{proof}

\begin{ej}
	Vamos a empezar por $G = S_3$. En $S_3$ tenemos los subgrupos $\langle (12) \rangle, \langle (13) \rangle, \langle (23) \rangle$ de orden 2 y el subgrupo $\langle (123) \rangle = \{(1), (123), (132)\}$ de orden 3. %TODO pintar retículo
	\begin{itemize}
		\item En el caso de este último $g\langle (123) \rangle \inv{g} = \langle (123) \rangle$ porque es el único subgrupo de orden 3. Por tanto $\langle (123) \rangle \normsub S_3$ y entonces $N(\langle (123) \rangle) = S_3$.
		\item Sin encambio en el caso de los subgrupos de orden $2$ es posible que $g\langle (12) \rangle \neq \langle (12) \rangle$, porque hay más de un subgrupo de orden 2. Observemos por ejemplo que $(13)(12)\inv{(13)} = (32) = (23)$, luego $\langle (12) \rangle$ no es normal en $S_3$, ya que hemos encontrado $g = (13) \in G$ que lo mueve. Pero ¿quién es el normalizador $N(\langle (12) \rangle)$? Pues ya sabemos que es un subgrupo propio, porque no puede dar todo $S_3$. Evidentemente $\langle (12) \rangle \subset N(\langle (12) \rangle)$. Luego tiene que ser que $N(\langle (12) \rangle) = \langle (12) \rangle$\footnote{No tiene gracia que $\langle (12) \rangle$ sea normal en sí mismo, lo que tiene gracia es que $\langle (12) \rangle$ es el mayor grupo donde $\langle (12) \rangle$ es normal.} 
	\end{itemize}
\end{ej}

%20181011

\begin{ej}
	Seguimos por \nameref{ej:famosogrupod4}). Vimos anteriormente (\autoref{ej:clasesd4}) que $Z(D_4) = \{1, B^2\}$. Tenemos su retículo en \autoref{fig:reticuloD4}. Queremos ver de entre los subgrupos de $D_4$, cuáles son los que conmutan.
	\begin{itemize}
		\item Empecemos por $\langle B \rangle = \{1, B, B^2, B^3\}$. Observamos que $\langle b \rangle$ es normal puesto que tiene índice 2, es decir que $\{g\langle B \rangle \inv{g} \mid g \in G\} = \{\langle B \rangle\}$ y tiene sentido que $[G:N(\langle B \rangle)] = 1$. Es decir que como $\langle B \rangle$ es normal tenemos que $N(\langle B \rangle) = D_4$.
		\item Seguimos por $H = \{1, A, B^2, AB^2\}$. Ocurre lo mismo, luego $N(H) = D_4$.
		\item Con el caso de $\langle B^2 \rangle$ tenemos también que $N(\langle B^2 \rangle) = D_4$ por ser normal.
		\item Agotados los subgrupos normales, nos quedan los más difíciles. Consideramos ahora $\langle A \rangle$. Una vez más nos preguntamos quién es el normalizador de $\langle A \rangle$.
		\begin{enumerate}
			\item Es claro que $\langle A \rangle$ conjugará con otros subgrupos de orden 2.
			\item También es claro que $\langle A \rangle \subset N(\langle A \rangle)$ y que $\langle B^2 \rangle \subset N(\langle A \rangle)$. Luego $N(\langle A \rangle)$ tiene al menos 2 elementos.
			\item También sabemos que $N(\langle A \rangle) \subsetneq G$ puesto que $\langle A \rangle$ no es normal, por lo que no puede tener 8 elementos. Por esto y porque $N(\langle A \rangle) < G$, concluimos que $|N(\langle A \rangle)| = 4$.
			\item ¿Cuáles mueven al $\langle A \rangle$? Sabemos que no puede haber más de dos, pues el normalizador tiene 4 elementos. Pues mirando la presentación nos damos cuenta de que $BA = A\inv{B} \iff BA\inv{B} = AB^2$. Luego nos damos cuenta de que $A$ se mueve a $AB^2$.
			\item Análogamente nos damos cuenta de que $AB$ se mueve a $AB^3$.
			\item Ya tenemos los dos elementos que se mueven.
		\end{enumerate}
	\end{itemize}
\end{ej}

\begin{ej}
	Vamos ahora con el grupo de cuaterniones $H$ descrito en el \autoref{ej:grupocuaterniones}.
	
	
	
	\begin{enumerate}
		\item Nos dibujamos el retículo.
		\item Primeramente nos damos cuenta de que $\langle A \rangle \cap \langle b \rangle \supsetneq \{e\}$ porque $H$ tiene 8 elementos y por la fórmula del producto libre (\autoref{thm:cardinalidadproductolibre}) y porque todo producto directo de subgrupos está contenido en el grupo aunque no sea subgrupo.
		\item Ocurre lo mismo con los demás subgrupos de orden 4 ($\langle A \rangle,\ \langle AB \rangle$). Tiene que tener intersección no vacía. En concreto la intersección es el subgrupo generado $\langle A^2 = B^2 = (AB)^2 \rangle$.
		\item En $H$ todos los subgrupos son normales, por lo que no tienen "órbitas" de modo que es muy aburrido.
	\end{enumerate}
\end{ej}

\begin{ej}
	Consideramos ahora $D_5$ que funciona como el $D_4$:
	\begin{align*}
	D_5 = \{1, B, B^2, B^3, B^4, A, AB, AB^2, AB^3, AB^4\} \\
	o(B) = 5
	\end{align*}
	\begin{itemize}
		\item Primera observación. Como $o(B) = 5$ que es primo, tenemos que $o(B^k) = 5,\ k = 1, \dots, 4$. Luego cualquier subgrupo generado por $\langle B^k \rangle = \langle B \rangle$. Aquí falta algo.
		\item Observemos que los subgrupos propios pueden ser de 2 o 5 elementos.
		\item No puede haber subgrupos generados por dos elementos de $D_5$ (por qué?)
		\item Los únicos subgrupos son $\langle B \rangle$ y los generados por $A, AB, AB^2, AB^3, AB^4$.
		\item Afirmamos que $\{gA\inv{g} \mid g \in G\} = \{\langle A \rangle, \langle AB \rangle, \langle AB^2 \rangle, \langle AB^3 \rangle, \langle AB^4 \rangle \}$. Vamos a probarlo.
		
		\begin{enumerate}
			\item Primero nos damos cuenta de que $\{1, A\} \in N(\langle A \rangle)$.
			\item Además tenemos que no puede haber otro grupo por encima de $\langle A \rangle$ y $D_5$ por lo que tenemos que $N(A) = \langle A \rangle$.
			\item Por tanto en la órbita de $A$ tenemos $[D_5:\langle A \rangle] = 5$ grupos.
		\end{enumerate}
		
	\end{itemize}
\end{ej}

\hr

Sea $X$ conjunto. Consideramos
\begin{align*}
Biy(X) = \{f \mid f: X \to X \text{ biyección}\}
\end{align*}
En el caso en que $|X| = n$, por ejemplo $X = \{1, 2, 3, \dots, n\}$ tenemos que $Biy(X) = S_n$. Como $f:X \to X$ si $f$ es inyectiva entonces automáticamente es sobre y por tanto biyectiva.

En general, tiene sentido pensar en $Biy(X)$ aunque $|X| = \infty$. Además, en dicho conjunto viven la biyección identidad y la biyección inversa para cada biyección. Por tanto, tiene sentido pensar en $(Biy(X), \circ)$ como un grupo (la composición de biyecciones da una biyección).

Nos concentramos en el caso en el que $|X| = n$ que nos da $Biy(X) = S_n$. Ya hemos visto que $S_2 = \{1, \sigma\} \implies |S_2| = 2$ y para $S_3$ tenemos $|S_3| = 3!$ y en general $|S_n| = n!$.

Fijamos un conjunto $X$ y un homomorfismo de grupos $\alpha: X \to Biy(X)$. A partir de estos datos definimos una relación de equivalencia que nos da una partición de $X$, es decir, vamos a partir $X$ en conjuntos disjuntos.

\begin{ej}
	Supongamos\footnote{Por qué cojones cambia ahora la letrita?} $G = X,\ |G| = n$ y consideramos $\rho: G \to \autom{G} \subset Biy(X)$. Definimos la relación en $X = G$
	\begin{align*}
	aRb \iff \exists g \in G \mid \phi_g(a) = b,\ \phi_g(x) = gx\inv{g}
	\end{align*}
	que es la relación de conjugación dada por el isomorfismo de conjugación de toda la vida.
	
	Ahora, en lugar de pensar en $G = X$ pensamos en $X = \{H < G\}$ (los subgrupos de $G$). Para cualquier isomorfismo de grupos $\beta: G \to G$, tenemos que si $H < G$ entonces $\beta(H) < G$.
	
	Lo que hemos hecho aquí es un caso particular de lo que viene ahora.
\end{ej}

\begin{pro}
	Sea $\alpha: G \to Biy(X),\ g \mapsto \alpha(g)$ un homomorfismo de grupos. Definimos la relación de equivalencia
	\begin{align}
	aRb \iff \exists g \in G \mid \alpha(g)(a) = b
	\end{align}
	Afirmamos que la relación es de equivalencia y que nos divide $G$ en subconjuntos disjuntos (nos particiona $G$).
\end{pro}

\begin{proof}Probamos las 3 propiedades de las relaciones de equivalencia.
	\begin{enumerate}
		\item Reflexiva: $\forall x \in X, a R a$. Por ser $\alpha$ homomorfismo tenemos que $\alpha(e_G) = id_X$ y por tanto $\alpha(e_G)(a) = a$.
		\item Simétrica: $aRb \implies bRa$. Partimos de que $\exists g \in G \mid \alpha(g)(a) = b$. Tomamos $\inv{g} \in G$ y por ser $\alpha$ homomorfismo de grupos tenemos que $\alpha(\inv{g})(b) = \inv{(\alpha(g))}(b) = a$.
		\item Transitiva: $aRb \land bRc \implies aRc$. Partimos de que $\exists g, g' \in G \mid \alpha(g)(a) = b \land \alpha(g')(b) = c$. Tomamos $g'g \in C$ y tenemos que $\alpha(g'g)(a) = \alpha(g')(\alpha(g)(a)) = \alpha(g')(b) = c$ por composición de biyecciones.
	\end{enumerate}
\end{proof}

¿Cómo son las clases que da la partición?

Pues tenemos que $cl(a) = \{\alpha(g)(a) \mid g \in G\}$ para un $a \in G$. Definimos $H_a = \{g \in G \mid \alpha(g)(a) = a\}$. Tenemos por lo visto anteriormente que $H_a < G \land |cl(a)| = [G:H_a]$. Entonces tenemos lo siguiente:
\begin{itemize}
	\item En el caso en que $X = G$ tenemos que $H_a = C(a)$ donde $C(a)$ es el centralizador de $a$ (definición \ref{dfn:centralizador}).
	\item En el caso en que $X = \{H < G\}$ tenemos que $H_a = N(a)$ donde $N(a)$ es el normalizador de $a$ (definición \ref{dfn:normalizador}).
\end{itemize}
Veremos que se pueden dar más casos.

\begin{ej}
	Fijamos $\sigma \in S_n$ y $G = \langle \sigma \rangle$ subgrupo genereado por $\sigma$ en $S_n$. Entonces $G = \langle \sigma \rangle \to S_n = Biy(X)$ algo pasó. Si $X = \{1, 2, \dots, n\}$ definimos $\sigma(1) = 2,\ \sigma(2) = 1,\ \sigma(i) = i+1, i = 3,\dots, n-2,\sigma(n-1) = 3$. La clase $cl(i) = \{\sigma^k(i) \mid k \in \Z\}$ en particular contiene a la identidad ya que $\sigma^{n!} = id$ y $n! \in \Z$. Nos quedan dos clases
	% TODO: dibujito de sigma
	\begin{align*}
	cl(1) &= \{1, 2\} \\
	cl(3) &= \{3, 4, 5, \dots, n - 1\}
	\end{align*}
\end{ej}

Vemos que si fijamos $\sigma$ se define una partición en $\{1, \dots, n\}$ de subconjuntos disjuntos
\begin{align*}
F_1 \cup F_2 \cup \dots \cup F_n
\end{align*}

Si $r = |F_i| > 1$, $F_i = \{i_0, i_1, \dots, i_r\}$ tal que $\sigma(i_0) = i_1, \sigma(i_1) = i_2, \dots, \sigma(i_r) = i_0$.

\begin{dfn}[Ciclo]
	\label{dfn:ciclo}
	Diremos que $\sigma$ es un ciclo de longitud $r$ si en la partición definida
	\begin{align*}
	F_1 \cup F_2 \cup \dots \cup F_n
	\end{align*}
	todas las cajas $F_j,\ j < r$ tienen un único elemento y $F_r$ tiene $r$ elementos.
\end{dfn}

\begin{pro}
	Toda biyección $\sigma \in S_7$ se puede descomponer como composición de ciclos.
\end{pro}

\begin{ej}
	Consideramos\footnote{Utilizamos la notación de biyecciones de \cite{dor96}.}
	\begin{align*}
	\sigma = \left(\begin{array}{ccccccc}
	1 & 2 & 3 & 4 & 5 & 6 & 7 \\
	2 & 1 & 4 & 5 & 6 & 3 & 7
	\end{array}\right)
	\end{align*}
	que nos divide $X = \{1, 2, 3, 4, 5, 6, 7\}$ en tres subconjuntos disjuntos $\{1, 2\},\ \{3, 4, 5, 6\},\ \{7\}$. Por tanto podemos decir
	\begin{align*}
	\sigma = (12)(3456)(7) = (12)(3456) = (3456)(12)
	\end{align*}
	(podemos conmutar porque al ser ciclos disjuntos lo que toque uno no lo toca el otro).
\end{ej}

Proximamente vermos que a partir de la descomposición en ciclos disjuntos es fácil obtener el orden de $\sigma$.
