% !TeX root = ../ecuaciones-diferenciales.tex

\chapter{Notaci\'{o}n}
\begin{multicols}{2}
    \begin{itemize}
        \item $ \mathcal{P} \equiv f(x) = g(x)$, $\mathcal{P}$ es una ecuación.
        \item $\mbf{y} = f(x)$, $\mbf{y}$ es una variable dependiente.
        \item $\dd{y}{x} = y' = y_x$ es la derivada de y respecto de x.
        \item $X, Y, Z, A, B, ...$ son vectores en $\R^n$ o funciones $:(\alpha, \beta) \to \R^n$
        \item $\mathbb{A}, \mathbb{B}, \mathbb{C}$ son matrices (típicamente cuadradas $\R^n \times \R^n$) o funciones $:(\alpha, \beta) \to \R^n \times \R^n$.
        \item Sea el sistema:
        $$
            \begin{cases}
                x' = 2tx - e^ty + \sin(t)\\
                y' = x - \cos(t) y
            \end{cases}
        $$
        lo escribimos:
        $$
            \left[
            \begin{matrix}
                x\\
                y
            \end{matrix}\right]' =
            \left[
            \begin{matrix}
                2t & -e^t\\
                1 & -\cos(t)
            \end{matrix}\right] \cdot
            \left[
            \begin{matrix}
                x\\
                y
            \end{matrix}
            \right]
            +
            \left[
            \begin{matrix}
                \sin(t)\\
                0
            \end{matrix}
            \right]
        $$
        o de la misma forma:
        $$
        X' = \mathbb{A}(t) X + B(t)
        $$
        \item La matriz diagonal:
            $$ \diag(a_1, a_2, a_3, a_n) =
                \mx{
                    a_1 & 0  & 0  & 0  \\
                     0  &a_2 & 0  & 0  \\
                     0  & 0  & a_3& 0  \\
                     0  & 0  & 0  & a_n
                }
            $$
        \item La matriz formada por los vectores columna $X_1, \ldots X_n$:
        $$
        \mx{\uparrow && \uparrow\\ X_1(t_0) & \cdots & X_n(t_0) \\ \downarrow && \downarrow}
        $$
        \item $f_n \to_{pp} f$, $f_n$ converge puntualmente (o punto a punto) a $f$.
        \item $f_n \to_{unif} f$, $f_n$ converge uniformemente a $f$.
    \end{itemize}
\end{multicols}
